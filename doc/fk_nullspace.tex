\section{Null space of FK}
The FK tables can be regarded as a linear map from the space of PDFs to the space of 
data:
\begin{align}
  \FKtab : \mathbb{R}^{\textrm{PDF}} \rightarrow \mathbb{R}^{\textrm{data}} \,.
\end{align}
Note that the matrix corresponding to this linear map is not square, and hence $\FKtab 
\neq \FKtabT$. We can define the null space of the FK tables
\begin{equation}
  \ker \FKtab \equiv K_{\FKtab} = 
  \left\{
    \pmb{u}_K \in \mathbb{R}^{\PDF} \; : \; \FKtab \, \pmb{u}_K = 0
  \right\} \,,
  \label{eq:ker_FK}
\end{equation}
together with its orthogonal space
\begin{equation}
  R_{\FKtab} \equiv K_{\FKtab}^{\bot} = 
  \left\{
    \pmb{u}_{\bot} \in \mathbb{R}^{\PDF} \; : \; \pmb{u}_{\bot} \cdot \pmb{u}_K = 0,
    \hspace{2mm} \forall \pmb{u}_K \in K_{\FKtab}
  \right\}\,.
\end{equation}
For each of these two subspaces we can define a basis
\begin{align}
  & \B_{K} = \left\{ \pmb{u}^{(i)}_{K}\,, \hspace{2mm}  i=1,\dots, \dim K_{\FKtab} \right\} \,,\\
  & \B_{\bot} = \left\{ \pmb{u}^{(i)}_{\bot}\,, \hspace{2mm}  i=1,\dots, \dim R_{\FKtab} \right\} \,.
  \label{eq:FK_basis}
\end{align}
Remember that $K_{\FKtab} \bigoplus R_{\FKtab} = \mathbb{R}^{\PDF}$ and thus the basis $\B = \B_{K}
\bigoplus \B_{\bot}$ is a basis for $\mathbb{R}^{\PDF}$. Henceforth, when decomposing a vector in $\RPDF$, I 
will use the ordering $\left\{ \B_{K}, \B_{\bot} \right\}$. Hence, given $\pmb{f}\in \RPDF$, we can
write
\begin{equation}
  \pmb{f} = \pmb{f}_{K} + \pmb{f}_{\bot} 
          = \bpmat 
              f_K \\[2pt]
              f_{\bot}
            \epmat \,.
\end{equation}
Finally, note that $\FKtab$ is not symmetric (not even square). Thus, the right null space is not the same
as the left null space, in particular
\begin{align}
  \FKtab \pmb{u}_K = 0  \; \nRightarrow \; \pmb{u}_K^T \FKtab = 0 \,.
\end{align}

With the two bases in eqs.~\eqref{eq:FK_basis}, we can decompose the $\FKtab$ as follows
\begin{equation}
  \FKtab =
  \bpmat
    \FKtab_{KK}     & \FKtab_{K\bot} \\[3pt]
    \FKtab_{\bot K} & \FKtab_{\bot\bot}
  \epmat \,,
\end{equation}
where
\begin{align}
  \FKtab_{B_1 B_2} = \sum_{i=1}^{\dim\B_1} \sum_{j=1}^{\dim\B_2}
  \bra{u_{\B_1}^{(i)}} M \ket{u_{\B_2}^{(j)}} \;
  \pmb{u}_{\B_1}^{(i)} \pmb{u}_{\B_2}^{(j)}\,,
  \hspace{5mm}
  \B_1, \B_2 = K_{\FKtab}, R_{\FKtab}\,.
\end{align}
By definition of the kernel, we also have
\begin{align}
  \FKtab_{KK} = \FKtab_{\bot K} = 0 \,,
\end{align}
while $\FKtab_{K\bot}$ would be zero only if $\FKtab$ was diagonal. Thus, in the basis
$\B$, the FK tables can be expressed as
\begin{align}
  \FKtab =
  \bpmat
    0  & \FKtab_{K\bot} \\[3pt]
    0 & \FKtab_{\bot\bot}
  \epmat \,.
\end{align}
The product with a vector $\pmb{f} \in \RPDF$ becomes
\begin{equation}
  \FKtab \pmb{f} =
    \bpmat
      0  & \FKtab_{K\bot} \\[3pt]
      0  & \FKtab_{\bot\bot}
    \epmat
    \bpmat 
      f_K \\[2pt]
      f_{\bot}
    \epmat
    = \biggl(\FKtab_{K\bot} + \FKtab_{\bot\bot}\biggr) f_{\bot} \,.
\end{equation}

What are the implications for the matrix $M$? ...