\section{Conclusions}
\label{sec:conclusions}

Our present age is marked by unprecedented advancements in machine learning
techniques, whose applications span many scientific domains -- PDF determination
being one of them. However, we believe that we ought to move from mere witnesses
to active contributions. In fact, it is of paramount importance to understand
how these new techniques behave when applied to complex problems such as PDF
determination.

In this work, we have taken a step forward in this direction. We have
investigated a novel treatment of the learning process in the context of PDF
fitting by exploring the training dynamics in the functional space of the neural
network. In fact, the NTK can be used to unravel complex dynamics obfuscated by
the training algorithms commonly employed in PDF fits. We have shown that the
properties of the NTK are highly entangled with the fitting results, and that a
proper understanding of its structure can provide precious insights on the
learning process. Notably, we have developed, under certain assumptions, an
analytical description of how the neural network evolves during training,
enabling us to better understand the NNPDF methodology and its dependence on the
underlying model architecture. To the best of our knowledge, such description
has not yet been attempted in the context of ill-defined inverse problems.

Yet, this work is far from being conclusive. As a pioneering study, we believe
that the most important contribution of this work has been to open the
discussion on the role of the NTK in PDF fits, identifying a set of relevant
diagnostic metrics that can be explored in future works. Indeed, many aspects
merit further investigations. First of all, we are unable to anticipate the
outcome of a similar analysis when applied to a more complex and realistic
framework, including multiple fitted flavors and real data. Furthermore,
further insights could be gained by exploring different architectures and,
possibly, parametrisations beyond neural networks.

In spite of the simplified framework adopted in the present study, our findings
highlight the complexity and richness of the learning process, confirming how
little is understood regarding the origin of PDF uncertainties. This poses a
significant challenge in light of the improved precision of forthcoming
measurements. We believe that the tools presented here can help address this
gap. 