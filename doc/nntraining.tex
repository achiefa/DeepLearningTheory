\documentclass[11pt]{article}

\include{init}

\newcommand{\ldd}[1]{\textcolor{red}{\textbf{Luigi: #1}}}
\newcommand{\ac}[1]{\textcolor{red}{\textbf{Amedeo: #1}}}

\title{Parton Distributions from Neural Networks: Analytical Results}
\author{Amedeo Chiefa}
\author{Luigi Del Debbio}
\author{Richard Kenway}
\affil{Higgs Centre for Theoretical Physics, School of Physics and Astronomy,
Peter~Guthrie~Tait~Road, Edinburgh EH9 3FD, United Kingdom.}

\date{\today}
\makeindex

\begin{document}

\maketitle

\begin{abstract}
  Parton Distribution Functions (PDFs) play a crucial role in describing
  experimental data at hadron colliders and provide insight into proton
  structure. As the LHC enters an era of high-precision measurements, a robust
  PDF determination with a reliable uncertainty quantification has become
  increasingly important to match the experimental precision. The NNPDF
  collaboration has pioneered the use of Machine Learning (ML) techniques for PDF
  determination. In this work, we develop a theoretical framework based on the
  Neural Tangent Kernel (NTK) to analyze the training dynamics of Neural
  Networks. This approach allows us to derive, under certain assumptions, an
  analytical description of how the neural network evolves during training,
  enabling us to better understand the NNPDF methodology and its dependence on
  the underlying model architecture. Notably, we demonstrate that our results
  contrast, to some extent, with the standard picture of the \textit{lazy training}
  regime commonly discussed in the ML community.
\end{abstract}

\section{Introduction}
\label{sec:intro}


A first paragraph on precision physics at the LHC and the need for robust determinations of PDFs.

The extraction of PDFs from experimental data is a classic example of an inverse problem,
namely the reconstruction of a function $f(x)$ from a finite set of data points
$Y_I$, where the index $I=1, \ldots, \ndat$.~\footnote{When omitting the data index $I$, we will always
assume $Y \in \mathbb{R}^{\ndat}$.} In particular, for this study, we will focus
on DIS data, which depend
linearly on the function $f(x)$. The theoretical prediction for the data point $Y_I$ is denoted
\begin{equation}
    \label{eq:TheoryPred}
    T_I[f] = \sum_{i=1}^{\nflav} \int dx\, C_{Ii}(x) f_{i}(x)\, ,
\end{equation}
where $C_{Ii}(x)$ is a coefficient function, known to some given order in perturbation theory,
$i$ labels the parton flavor, and $f_i(x)$
is the PDF (or set of PDFs) that we want to determine.

Trying to determine a function $f$ in an infinite dimensional space of solutions with a finite
set of data leads to an ill-defined problem, whose solution will depend on assumptions made.
In particular, the choice of a parametrization for $f$ leads to a bias in the space
of solutions that can be obtained. Together with the fit methodology, the parametrization also
determines the propagation of the error on the data to the error on the fitted solution. Understanding
the bias and the variance of the fitted PDF is therefore a major challenge for precision physics.

Following the ideas highlighted in Refs.~\cite{DelDebbio:2021whr,Candido:2024hjt}, the solution
of the inverse problem is conveniently phrased in
a Bayesian framework. The functions $f_i$ are promoted to stochastic processes; for any grid
of points $x_{\alpha}$, $\alpha=1, \ldots, \ngrid$, the vector $f_{i\alpha}=f_{i}(x_{\alpha})$ is a
vector of $\nflav\times\ngrid$ stochastic variables, for which we introduce a prior distribution $p(f)$.~\footnote{Following
the same convention used for the data, when omitting the grid index $\alpha$, and/or the flavor index $i$, we will always refer to a
vector $f \in \mathbb{R}^{\nflav\times\ngrid}$.}

Any fitting procedure is interpreted as a recipe that yields the posterior distribution
$\tilde{p}(f)$.

In this study, probability distributions are represented by ensembles of i.i.d. replicas.
So, for instance, the prior distribution $p(f)$ is described by an ensemble
\begin{equation}
    \label{eq:RepDef}
    \left\{f^{(k)} \in \mathbb{R}^{\nflav\times\ngrid}; k=1, \ldots, \nreps\right\}\, ,
\end{equation}
drawn from the distribution $p$, so that
\begin{equation}
    \label{eq:ReplicaEnsemble}
    \mathbb{E}_{p}[O(f)] = \frac{1}{\nreps} \sum_{k=1}^{\nreps} O(f^{(k)})\, ,
\end{equation}
for any observable $O$ that is built from the PDFs.

The prior distribution $p(f)$ is defined by initializing a set of replicas using a Glorot-Normal 
initializer~\cite{glorot2010understanding}.
The result of this initialization is discussed below in Sec.~\ref{sec:Init}.
For each replica, a new set of data $Y^{(k)}$ is generated from an $\ndat$ dimensional Gaussian distribution
centred at the experimental central value $Y$, with the covariance given by the experimental covariance
matrix $C_Y$,
\begin{equation}
    \label{eq:ExpReplicaDistr}
    Y^{(k)} \sim \mathcal{N}\left(Y, C_Y\right)\, .
\end{equation}
Each replica $f^{(k)}$ is trained on its corresponding data set $Y^{(k)}$. We denote the replicas at training time $t$,
$f^{(k)}_{t} \in \mathbb{R}^{\nflav\times\ngrid}$. Stopping the training at time $\bar{t}$, the posterior probability
distribution is represented by the set of replicas
$\left\{f^{(k)}_{\bar{t}}\in \mathbb{R}^{\nflav\times\ngrid}; k=1, \ldots, \nreps\right\}$, so that averages over the posterior
distribution are computed as
\begin{equation}
    \label{eq:PostEnsemble}
    \mathbb{E}_{\tilde{p}}[O(f)] = \frac{1}{\nreps} \sum_{k=1}^{\nreps}
        O\left(f^{(k)}_{\bar{t}}\right)\, .
\end{equation}
All knowledge about the solution of the inverse problem, $f$, is encoded in the posterior
$\tilde{p}$ and is expressed as expectation values of observables $O$ using
Eq.~\eqref{eq:PostEnsemble}.

\FloatBarrier



\section{Large Neural Networks at Initialization}
\label{sec:Init}
When initializing a neural network, the weights and biases -- which we denote collectively as the {\em parameters}\ 
of the network -- are drawn from some probability distribution. In 
the NNPDF formalism, the set of netowrk parameters at initialization for each replica is an instance 
of i.i.d. stochastic variables. The probability distribution
of the network parameters induces a probability distribution for the output of the neural networks at inizialization. 
It is well known that the probability distribution of these outputs becomes approximately gaussian when the size of
the hidden layers is increased. We call this limit the {\em large-network} limit. 
In this section, we review some basic results for the large-network limit at inizialization, and 
compare these theoretical 
predictions with the actual results obtained in numerical experiments performed with the typical networks 
used by the NNPDF collaboration. 
This preliminary study of the properties of networks at initialization also allows us to introduce the 
notation used for the networks in the rest of the paper. 

As detailed in Ref.~\cite{NNPDF:2021njg}, the NNs used for the NNPDF fit have a 2-25-20-8 architecture,
a $\tanh$ activation function,
and are initialized using a Glorot normal distribution~\cite{glorot2010understanding}. The preactivation
function of a neuron is denoted as $\phi^{(\ell)}_{i,\alpha} = \phi^{(\ell)}_i(x_\alpha)$, where $\ell$
denotes the layer of the neuron, $i$ identifies the neuron within the layer\footnote{We refer
to $i$ as the {\em neuron}\ index.}, and $x_{\alpha}$ is a point in the interval $[0,1]$.
A grid of $\ngrid=50$ points is used to compute observables in the NNPDF formalism and in this work 
we focus on the vakue of $f$ at those values of $x_\alpha$. For completeness, we list the values of $x_\alpha$ in
Tab.~\ref{tab:Xvals}.

\begin{table}[ht]
    \centering
    \begin{tabular}{|c|c|c|c|c|c|c|c|c|c|}
    \hline
    $\alpha$ & $x_\alpha$ & $\alpha$ & $x_\alpha$ & $\alpha$ & $x_\alpha$ & $\alpha$ & $x_\alpha$ & $\alpha$ & $x_\alpha$ \\
    \hline
    $1$  & $2.00 \times 10^{-7}$ & $11$ & $1.29 \times 10^{-5}$ & $21$ & $8.31 \times 10^{-4}$ & $31$ & $0.0434$ & $41$ & $0.422$ \\
    $2$  & $3.03 \times 10^{-7}$ & $12$ & $1.96 \times 10^{-5}$ & $22$ & $1.26 \times 10^{-3}$ & $32$ & $0.0605$ & $42$ & $0.480$ \\
    $3$  & $4.60 \times 10^{-7}$ & $13$ & $2.97 \times 10^{-5}$ & $23$ & $1.90 \times 10^{-3}$ & $33$ & $0.0823$ & $43$ & $0.540$ \\
    $4$  & $6.98 \times 10^{-7}$ & $14$ & $4.51 \times 10^{-5}$ & $24$ & $2.87 \times 10^{-3}$ & $34$ & $0.109$ & $44$ & $0.601$ \\
    $5$  & $1.06 \times 10^{-6}$ & $15$ & $6.84 \times 10^{-5}$ & $25$ & $4.33 \times 10^{-3}$ & $35$ & $0.141$ & $45$ & $0.665$ \\
    $6$  & $1.61 \times 10^{-6}$ & $16$ & $1.04 \times 10^{-4}$ & $26$ & $6.50 \times 10^{-3}$ & $36$ & $0.178$ & $46$ & $0.730$ \\
    $7$  & $2.44 \times 10^{-6}$ & $17$ & $1.57 \times 10^{-4}$ & $27$ & $9.70 \times 10^{-3}$ & $37$ & $0.220$ & $47$ & $0.796$ \\
    $8$  & $3.70 \times 10^{-6}$ & $18$ & $2.39 \times 10^{-4}$ & $28$ & $0.0144$ & $38$ & $0.265$ & $48$ & $0.863$ \\
    $9$  & $5.61 \times 10^{-6}$ & $19$ & $3.62 \times 10^{-4}$ & $29$ & $0.0211$ & $39$ & $0.314$ & $49$ & $0.931$ \\
    $10$ & $8.52 \times 10^{-6}$ & $20$ & $5.49 \times 10^{-4}$ & $30$ & $0.0305$ & $40$ & $0.367$ & $50$ & $1.00$ \\
    \hline
\end{tabular}

    \caption{Values of $x_\alpha$ used in the NNPDF grids for the computation of
    observables. The points are equally spaced on a logarithmic scale
    for $\alpha = 1, \ldots, XXX$, and linearly spacing for $\alpha > XXX$.
    \ac{Maybe we need to rethink the layout of this table...}
    \label{tab:Xvals}}
\end{table}

The output of the neuron identified by the pair $(\ell,i)$ is
$\rho^{(\ell)}_{i\alpha} = \tanh\left(\phi^{(\ell)}_{i\alpha}\right)$.
The parameters of the NN are the weights $w^{(\ell)_{ij}}$ and the biases $b^{(\ell)}_i$, which are
collectively denoted as $\theta_\mu$, where $\mu = 1, \ldots, P$ and the total number of parameters
is
\begin{equation}
    \label{eq:TotPar}
    P = \sum_{\ell=1}^{L} \left(n_{\ell} n_{\ell-1} + n_\ell\right)\, .
\end{equation}
The preactivation function in layer $(\ell+1)$ is a weighted average of the outputs of the neurons on 
the previous layer, namely
\begin{align}
    \label{eq:RecursionNN}
    \phi^{(\ell+1)}_{i\alpha} = \sum_{j=1}^{n_\ell} w^{(\ell+1)}_{ij} \rho^{(\ell)}_{i\alpha} + b^{(\ell+1)}_{i}\, .
\end{align}
The PDFs in the
so-called evolution basis are parametrized by the preactivation functions of the output layer $L$,
$x_\alpha f_i(x_\alpha)=A_i \phi^{(L)}_{i,\alpha}$, where $i=1, \ldots, 8$ labels the flavors.
\footnote{For simplicity, we ignore the preprocessing function $x^{-\alpha_i} (1-x)^{\beta_i}$ that
is currently used in the NNPDF fits. While the preprocessing may be useful in speeding the training
it does not affect the current discussion.}
The input layer is identified by $\ell=0$ and the activation
function for that specific layer is the identity, so that
\begin{equation}
    \label{eq:InitLayerPhi}
    \rho^{(0)}_{i,\alpha} = \phi^{(0)}_{i,\alpha} = x_{i,\alpha} =
    \begin{cases}
        x_\alpha\, , \quad &\text{for}\ i=1\, ;\\
        \log\left(x_\alpha\right)\, , \quad &\text{for}\ i=2\, .
    \end{cases}
\end{equation}
In the following we refer to the preactivation functions as {\em fields}.

The Glorot normal initialiser draws each weight and bias of the NN from independent Gaussian
distributions, denoted $p_w$ and $p_b$ respectively, centred at zero and with variances
rescaled by the number of nodes in adjacent layers,
\begin{equation}
    \label{eq:RescaledGlorotVariances}
    \frac{C^{(\ell)}_{w}}{\sqrt{n_{\ell-1} + n_{\ell}}}\, ,
    \quad \frac{C^{(\ell)}_{b}}{\sqrt{n_{\ell-1} + n_{\ell}}}\, .
\end{equation}
Following the NNPDF prescription, we have $C^{(\ell)_{w}=C^(\ell)}_{b}=1$. 
The probability distribution of the NN parameters induces a probability distribution for the
preactivations,
\begin{align}
    \label{eq:PreactAtInit}
    p\left(\phi^{(\ell)}\right)
      &= \int \mathcal{D}w\, p_w(w)\,
        \mathcal{D}b\, p_b(b)\, \prod_{i,\alpha}
        \delta\left(
          \phi^{(\ell)}_{i\alpha} - \sum_{j} w^{(\ell)}_{ij}
          \rho\left(\phi^{(\ell-1)}_{j\alpha}\right)
          - b^{(\ell)}_i
          \right)\, .
\end{align}
Note that, here and in what follows, $p(\phi^{(\ell)})$ denotes the joint probability for all the
$n_{\ell}\times\ngrid$ components of $\phi^{(\ell)}$,
\begin{align}
    \label{eq:ExplIndices}
    p\left(\phi^{(\ell)}\right) = p\left(\phi^{(\ell)}_{1,\alpha_1}, \phi^{(\ell)}_{2,\alpha_1}, \ldots,
        \phi^{(\ell)}_{n_\ell,\alpha_1}, \phi^{(\ell)}_{1,\alpha_2}, \ldots, \phi^{(\ell)}_{n_\ell,\alpha_2},
        \ldots,
        \phi^{(\ell)}_{n_\ell,\ngrid}\right)\, .
\end{align}
This duality between parameter-space and function-space provides a powerful framework to study
the behaviour of an ensemble of NNs, and in particular the symmetry properties of the distribution
$p(\phi^{(\ell)})$, see \eg~\cite{Maiti:2021fpy}. Working in parameter space, \ie\ computing the
expectation values of correlators of fields as integrals over the NN parameter, one can readily
show that
\begin{align}
    \label{eq:NeurRotInv}
    \mathbb{E}\left[
        R_{i_1j_1} \phi^{(n_\ell)}_{j_1 \alpha_1} \ldots
        R_{i_nj_n} \phi^{(n_\ell)}_{j_n \alpha_n}
    \right] =
    \mathbb{E}\left[
        \phi^{(n_\ell)}_{i_1 \alpha_1} \ldots
        \phi^{(n_\ell)}_{i_n \alpha_n}
    \right]\, ,
\end{align}
where $R$ is an orthogonal matrix in $\text{SO}(n_{\ell})$. Eq.\eqref{eq:NeurRotInv} implies
that the probability distribution in Eq.~\eqref{eq:PreactAtInit} is also invariant under rotations,
and therefore it can only be a function of $\text{SO}(n_{\ell})$ invariants. Therefore
\begin{align}
    \label{eq:PriorAction}
    p\left(\phi^{(n_\ell)}\right) =
        \frac{1}{Z^{(\ell)}} \exp\left(-S\left[\phi^{(\ell)}_{\alpha_1}
            \cdot \phi^{(\ell)}_{\alpha_2}\right]\right)\, ,
\end{align}
where
\begin{align}
    \label{eq:PhiInvariant}
    \phi^{(\ell)}_{\alpha_1}
            \cdot \phi^{(\ell)}_{\alpha_2} =
    \sum_{i=1}^{n_\ell} \phi^{(\ell)}_{i \alpha_1} \phi^{(\ell)}_{i \alpha_2}\, .
\end{align}
The action can be expanded in powers of the invariant bilinear,
\begin{align}
    \label{eq:ExpandAction}
    S\left[\phi^{(\ell)}_{\alpha_1}
            \cdot \phi^{(\ell)}_{\alpha_2}\right] =
        \frac12 \gamma^{(\ell)}_{\alpha_1\alpha_2}
            \phi^{(\ell)}_{\alpha_1} \cdot \phi^{(\ell)}_{\alpha_2} +
            \frac{1}{8 n_{\ell-1}} \gamma^{(\ell)}_{\alpha_1\alpha_2,\alpha_3\alpha_4}
            \phi^{(\ell)}_{\alpha_1} \cdot \phi^{(\ell)}_{\alpha_2} \,
            \phi^{(\ell)}_{\alpha_3} \cdot \phi^{(\ell)}_{\alpha_4} + O(1/n_{\ell-1}^2)\, ,
\end{align}
so that the probability distribution is fully determined by the couplings 
$\gamma^{(\ell)}$.\footnote{
    We have denoted {\em all}\ couplings by $\gamma^{{(\ell)}}$. Different couplings 
    are indentified by the number of indices, so that $\gamma^{(\ell)}_{\alpha_1\alpha_2}$ 
    is a two-point coupling, $\gamma^{(\ell)}_{\alpha_1\alpha_2,\alpha_3\alpha_4}$ is a four-point 
    coupling, etc. 
} 
In
Eq.~\eqref{eq:ExpandAction}, we have factored out inverse powers of $n_\ell$ for each coupling.
With this convention, and with the scaling of the parameters variances in
Eq.~\eqref{eq:RescaledGlorotVariances}, the couplings in the action are all $O(1)$
in the limit where $n_\ell\to\infty$.
As a consequence, the probability distribution at initialization is a multidimensional Gaussian at
leading order in $1/n_\ell$, with quartic corrections that are $O(1/n_\ell)$, while higher powers
of the invariant bilinear are suppressed by higher powers of the width of the layer. This power counting
defines an effective field theory, where deviations from Gaussianity can be computed in perturbation
theory to any given order in $1/n_\ell$, see \eg\ Ref.~\cite{Roberts:2021fes} for a detailed
presentation of these ideas. While the actual calculations become rapidly cumbersome, the
conceptual framework is straightforward.

At leading order, the second and fourth cumulant are respectively
\begin{align}
    &\langle \phi^{(\ell)}_{i_1,\alpha_1} \phi^{(\ell)}_{i_2,\alpha_2}\rangle
      = \delta_{i_1 i_2} K^{(\ell)}_{\alpha_1\alpha_2} + O(1/n_{\ell-1})\, , \\
    &\langle \phi^{(\ell)}_{i_1,\alpha_1} \phi^{(\ell)}_{i_2,\alpha_2}
      \phi^{(\ell)}_{i_3,\alpha_3} \phi^{(\ell)}_{i_4,\alpha_4}\rangle_c
      = O(1/n_{\ell-1})\, ,
\end{align}
where
\begin{equation}
    \label{eq:DefineKmat}
    K^{(\ell)}_{\alpha_1\alpha_2} = \left(\gamma^{(\ell)}\right)^{-1}_{\alpha_1\alpha_2}\, .
\end{equation}
The ``evolution'' of the couplings as we go deep in the NN, \ie\ the dependence of the couplings on
$\ell$, is governed by Renormalization Group (RG) equations, which preserve the power counting in
powers of $1/n_{\ell}$. At leading order,
\begin{align}
    K^{(\ell+1)}_{\alpha_1\alpha_2} &=
      \left.
      C_b^{(\ell+1)} + C_w^{(\ell+1)} \frac{1}{n_\ell}
      \langle \vec{\rho}^{\,(\ell)}_{\alpha_1} \cdot
      \vec{\rho}^{\,(\ell)}_{\alpha_2} \rangle
      \right|_{O(1)} \\
      \label{eq:RecursionForK}
      &= C_b^{(\ell+1)} + C_w^{(\ell+1)} \frac{1}{n_\ell}
      \langle \vec{\rho}^{\,(\ell)}_{\alpha_1} \cdot
      \vec{\rho}^{\,(\ell)}_{\alpha_2} \rangle_{K^{(\ell)}}\, ,
\end{align}
where
\begin{align*}
    \frac{1}{n_\ell}
      \langle \vec{\rho}^{\,(\ell)}_{\alpha_1} \cdot
      \vec{\rho}^{\,(\ell)}_{\alpha_2} \rangle_{K^{(\ell)}} =
    \int \prod_{\alpha}d\phi_\alpha\,
      \frac{e^{-\frac12 \left(K^{(\ell)}\right)^{-1}_{\beta_1\beta_2}
        \phi_{\beta_1} \phi_{\beta_2}}}
        {\left|2\pi K^{(\ell)}\right|^{1/2}}\,
        \rho(\phi_{\alpha_1}) \rho(\phi_{\alpha_2})\, .
\end{align*}
Eq.~\eqref{eq:RecursionForK} can be solved for the NNPDF architecture leading to the
covariance matrices for the output of the NNs displayed in
Figs.~\ref{Fig:KRecursionOne} and~\ref{Fig:KRecursionTwo}.
\begin{figure}[t!]
    \centering
    \includegraphics[scale=0.4]{figs/K1_correlations.pdf}
    \caption{The empirical (left) and analytical (right) covariance matrices $K^{(1)}$ of the first layer
    of the NNPDF architecture. The covariance in the left panel is computed ``bootstrapping'' over an
    ensemble of 100 replicas, initialised using the Glorot normal distribution. The covariance in the right
    panel is obtained by solving Eq.~\eqref{eq:RecursionForK} numerically.
    \label{Fig:KRecursionOne}
    }
\end{figure}

\begin{figure}[t!]
    \centering
    \includegraphics[scale=0.4]{figs/K2_correlations.pdf}
    \includegraphics[scale=0.4]{figs/K3_correlations.pdf}
    \caption{Same as Fig.~\ref{Fig:KRecursionOne}, but for the second (top) and third (bottom) layers of the
    NNPDF architecture.
    \ac{Here the four indices of the covariance $K_{i_1i_2, \alpha_1\alpha_2}$ are flattened into
    two indices for the sake of graphical representation. Maybe we should group the labels into groups of
    $N_{\rm grid}$ ticks on the axes.}
    \label{Fig:KRecursionTwo}
    }
\end{figure}

As a consequence of the symmetry of the probability distribution, the mean value of the fields at
initialization needs to vanish, while their variance at each point $x_\alpha$ is given by the
diagonal matrix elements of $K^{(\ell)}$. The central value and the variance of the
parametrized singlet ($\Sigma$) and gluon ($g$) at initialization are shown in
Fig.~\ref{fig:SingletGluonInit} for an ensemble of $\nreps=100$. The central value is computed as
discussed above in Eq.~\eqref{eq:ReplicaEnsemble},
\begin{align}
    \label{eq:MeanValAtInit}
    \bar{f}_{i\alpha} = \bar{f}_{i}(x_\alpha) = \frac{1}{\nreps} \sum_{k=1}^{\nreps} f^{(k)}_i(x_\alpha)\, ,
\end{align}
and the variance $\sigma^2_{i\alpha}$ is computed using the same formula with
\begin{align}
    \label{eq:VarAtInit}
    O(f) = \frac{\nreps}{\nreps-1} \left(f_i(x_\alpha) - \bar{f}_{i}(x_\alpha)\right)^2\, .
\end{align}

\begin{figure}
    \centering
    \includegraphics[scale=0.5]{plots/UoECentredLogo282v1160215.png}
    \caption{\ldd{Plot of replicas at init}}        
    \label{fig:SingletGluonInit} 
\end{figure}

\ldd{List of plots that we still need to do:
comparison analytical vs empirical, 
comparison analytical vs Gaussian processes}

\paragraph{Dependence on the architecture.}
Having analytical expressions for the variance at initialization allows us to investigate the
impact of the NN architecture on the prior that is imposed on the PDFs. Iterating
Eq.~\eqref{eq:RecursionForK} yields the covariance at initialization for various depths.
Do we get something interesting? worth mentioning?

\ldd{We should also look at the recursion relations with other activations. Again, check whether we
get something interesting...}

\FloatBarrier



\section{Training}
\label{sec:Training}
\subsection{Gradient Flow}
\label{sec:GradFlow}

The parameter-space/function-space duality described in Sec.~\ref{sec:Init} yields intersting insight
on the training process. Our main aim in this paper is understanding the dynamics driving the
training process. Therefore, we work in a simplified setting where we consider standard gradient descent
and data that depend linearly on the unknwon PDFs, as shown in Eq.~\eqref{eq:TheoryPred}. The generalization
to other minimizers and non-linear data is left to future investigations, but is expected to yield
qualitatively similar results. The results
in this subsection apply to any generic parametrization of the unknown function, they are {\em not}\
specific to the case of using NNs for the parametrization.

Gradient descent is described as a continuous flow of the parameters $\theta$ in training time $t$
along the negative gradient of the loss function $\mathcal{L}$. The parameters and the fields during
training are labelled by adding an index $t$, so that \eg\ $\theta_{t,\mu}$ identifies the parameter
$\theta_\mu$ at time $t$.
The gradient flow is given by
\begin{align}
    \label{eq:GradientFlowDef}
    \ddt &\theta_{t,\mu} = -\nabla_\mu \mathcal{L}_t\, .
\end{align}
We focus here on quadratic loss functions that are obtained as the negative logarithm of Gaussian
data distributions around their theoretical predictions,
\begin{align}
    \label{eq:QuadLoss}
    \mathcal{L}_t = \frac12 \left(Y - T[f_t]\right)^T C_Y^{-1} \left(Y - T[f_t]\right)\, ,
\end{align}
where $C_Y$ is the covariance of the data, which includes statistical and systematic errors given by
the experiments and also any theoretical error, like \eg\ missing higher orders in the theoretical
predictions. Indices that are summed over are suppressed to improve the clarity of the equations.
Note that the loss function at training time $t$ is computed using the theoretical prediction $T[f_t]$,
\ie\ the result of Eq.~\eqref{eq:TheoryPred} computed using the fields at training time $t$. For a quadratic
loss, the gradient is
\begin{align}
    \nabla_\mu \mathcal{L}_t = - \left(\nabla_\mu f_t\right)^T \left(\frac{\partial T}{\partial f}\right)_t
      C_Y^{-1} \epsilon_t\, ,
\end{align}
where, writing explicitly the data index,
\begin{align}
    \label{eq:EpsDef}
    \epsilon_{t,I} = Y_I - T_I[f_t]\, , \quad I=1, \ldots, \ndat\, .
\end{align}
For the specific case of a quadratic loss function, the gradient is proportional to $\epsilon_t$, which
is the difference between the theoretical prediction and the data at training time $t$. If at some point
during the training the theoretical predictions reproduce all the data, the training process ends.
A further simplification is obtained in the case of data that depend linearly on the unknown function $f$.
In the specific case of NNPDF fits, the integrals in Eq.~\eqref{eq:TheoryPred} are approximated by
a Riemann sum over the grid of $x$ points,
\begin{align}
    \label{eq:FKTabDef}
    T_I[f] \approx \sum_{i=1}^{\nflav}\sum_{\alpha=1}^{\ngrid} \FKtab_{Ii\alpha} f_{i\alpha}\, ,
\end{align}
and hence
\begin{align}
    \label{eq:dTbydf}
    \left(\frac{\partial T_I}{\partial f_{i\alpha}}\right)_t =
        \FKtab_{Ii\alpha}\, ,
\end{align}
and is independent of $t$. The flow of parameters $\theta$ translates into a flow for the fields,
\begin{align}
    \label{eq:NTKFlow}
    \ddt &f_{t,i_1\alpha_1} = (\nabla_\mu f_{t,i_1\alpha_1}) \ddt \theta_\mu =
      \Theta_{t,i_1\alpha_1i_2\alpha_2}
      \FKtabT_{i_2\alpha_2I} \left(C_Y^{-1}\right)_{IJ} \epsilon_{t,J}\, ,
\end{align}
where
\begin{align}
    \label{eq:NTKDef}
    \Theta_{t,i_1\alpha_1i_2\alpha_2} = \sum_\mu
    \nabla_\mu f_{t,i_1\alpha_1} \nabla_\mu f_{t,i_2\alpha_2}\, .
\end{align}
For clarity, we often omit indices and write
\begin{align}
    \label{eq:dTdfForLinearObs}
    \left(\frac{\partial T}{\partial f}\right)_t
        &= \FKtab\, , \\
    \label{eq:NTKDefNoIndices}
    \Theta_t
        &= \left(\nabla_\mu f_t\right) \left(\nabla_\mu f_t\right)^T\, , \\
    \label{eq:FlowEquationNoIndices}
    \ddt f_t
        &= \Theta_t \FKtabT C_Y^{-1} \epsilon_t\, .
\end{align}
Note that these equations do not refer to a specific parametrization and remain valid when some
explicit functional form is chosen to parametrize the PDFs, as \eg\ in
Refs.~\cite{Bailey:2020ooq,Hou:2019efy}.

\subsection{Lazy Training for the Flow Equation}
\label{sec:Lazy}

The large-$n_{\ell}$ effective theory discussed in Sect.~\ref{sec:Init} also predicts that
the NTK remains constant along training, up to corrections that are $O(1/n_{\ell})$, see
Ref.~\cite{DBLP:journals/corr/abs-1806-07572} and references therein for a derivation of this result.
This regime is sometimes referred to as {\em lazy kernel training}, and allows an analytical
solution of the flow equation.

We start by rewriting Eq.~\eqref{eq:FlowEquationNoIndices} as
\begin{align}
    \label{eq:FlowEqTwo}
    \ddt f_t = -\Theta M f_t + b\, ,
\end{align}
where
\begin{align}
    M &= \FKtabT C_Y^{-1} \FKtab\, , \quad b = \Theta \FKtabT C_Y^{-1} Y\, .
\end{align}
The eigenvectors of $\Theta$,
\begin{align}
    \label{eq:ThetaEigensystem}
    \Theta z^{(k)} = \lambda^{(k)} z^{(k)}\, ,
\end{align}
provide a basis for expanding Eq.~\eqref{eq:FlowEqTwo}. It is necessary at this stage to distinguish
the components of $f_t$ that are in the kernel of $\Theta$ from the ones that are in the orthogonal
complement, hence we introduce the notation
~\footnote{
    The scalar product is defined as
    \[
        \left(f'_{t'}, f_t\right)
            = \sum_{i,\alpha} f'_{t',i\alpha} f_{t,i\alpha}\, .
    \]
}
\begin{align}
    \label{eq:ParallelCompnents}
    &f^\parallel_{t,k} = \left(z^{(k)}, f_t\right)\, , \quad \text{if}\ \lambda_{(k)} = 0\, , \\
    \label{eq:OrthogonalComponents}
    &f^\perp_{t,k} = \frac{1}{\sqrt{\lambda^{(k)}}} \left(z^{(k)}, f_t\right)\, , \quad
        \text{if}\ \lambda^{(k)} \neq 0\, .
\end{align}
One can readily see that the components in the kernel of $\Theta$, $\text{ker}\ \Theta$,
do not evolve during the flow,
\begin{align}
    \label{eq:FlowParallel}
    \ddt f^\parallel_{t,k} = 0
        \quad \Longrightarrow \quad f^\parallel_{t,k} = f^\parallel_{0,k}\, .
\end{align}
The flow equation for the orthogonal components can be written as
\begin{align}
    \label{eq:FlowPerp}
    \ddt f^\perp_{t,k} = - H^\perp_{kk'} f^\perp_{t,k'}
        + B^\perp_{k}\, ,
\end{align}
where the indices on quantities that have a $\perp$ suffix only span the space orthogonal to the kernel
of $\Theta$, while the indices on quantities that have a $\parallel$ suffix span the kernel.
In Eq.~\eqref{eq:FlowPerp}, we introduced
\begin{align}
    H^\perp_{kk'} &= \sqrt{\lambda^{(k)}} \left(z^{(k)}, M z^{(k')}\right) \sqrt{\lambda^{(k')}}\, ,\\
    B^\perp_k &= -\sqrt{\lambda^{(k)}} \left[\left(z^{(k)}, M z^{(k')}\right) f^\parallel_{0,k'}
        - \left(z^{(k)}, \FKtabT C_Y^{-1} Y\right)\right]\, .
\end{align}
We refer to $H^\perp$ as the flow (or training) Hamiltonian; we see explicitly in the definition above that
the flow dynamics is determined by a combination of the architecture of the NN, encoded in the NTK, and the
data, on which $M$ depends. More specifically, the matrix elements of $M$ can be written as
\begin{align}
    \label{eq:MMatElems}
    \left(z^{(k)}, M z^{(k')}\right) = T^{(k)T} C_Y^{-1} T^{(k')}\, ,
\end{align}
where $T^{(k)} = T[z^{(k)}]$ is the vector of theory predictions for the data obtained using $z^{(k)}$ as the
input PDF. Similarly, we have
\begin{align}
    \label{eq:BMatElems}
    \left(z^{(k)}, \FKtabT C_Y^{-1} Y\right) = T^{(k)T} C_Y^{-1} Y\, .
\end{align}
Denoting by $d^\perp$ the dimension of the subspace orthogonal to $\text{ker}\ \Theta$, $H^\perp$ is
a $d^\perp\times d^\perp$ symmetric matrix, whose eigenvalues and eigenvectors satisfy
\begin{align}
    H^\perp_{kk'} w^{(i)}_{k'} = h^{(i)} w^{(i)}_{k}\, .
\end{align}
The solution to Eq.~\eqref{eq:FlowPerp} can be written as the sum of the solution of the
homogeneous equation, $\hat{f}^{\perp}_{t,k}$, and a particular solution of the full equation.
The solution of the homogeneous equation is
\begin{align}
    \label{eq:HomoSoln}
    \hat{f}^{\perp}_{t,k} = \sum_{i=1}^{d^\perp} f^{\perp}_{0,i} e^{-h^{(i)}t} w^{(i)}_k\, ,
\end{align}
where
% ~\footnote{
%     Note that here the scalar product is computed in the subspace orthogonal to the kernel of $\Theta$,
%     \[
%         \left(w^{(i)}, f^\perp_0\right) = \sum_{k=1}^{d_\perp} w^{(i)}_{k} f^\perp_{0,k}
%     \]
% }
\begin{align}
    \label{eq:InitialCi}
    f^{\perp}_{0,i} = \sum_{k=1}^{d_\perp} w^{(i)}_k f^\perp_{0,k}\, ,
        %= \left(w^{(i)}, f^\perp_0\right)\, ,
\end{align}
guarantees that the initial condition $\hat{f}^\perp_{t,k}=f^\perp_{0,k}$ is
satisfied. Similarly, if we define
\begin{align}
    \label{eq:BiDef}
    \Upsilon^{(i)} = \sum_{k=1}^{d_\perp} w^{(i)}_k B^\perp_{k}\, ,
        %= \left(w^{(i)}, B^\perp\right)\, ,
\end{align}
then
\begin{align}
    \label{eq:PartSol}
    \check{f}^\perp_{t,k} = \sideset{}{'}\sum_{i} \frac{1}{h^{(i)}} \Upsilon^{(i)}
        \left(1 - e^{-h^{(i)}t}\right) w^{(i)}_k\, ,
\end{align}
where the sum only involves the non-zero modes of $H^\perp$,
is a particular solution of the inhomogeneous equation, which satisfies the boundary
condition $\check{f}^{\perp}_{0,k}=0$. Finally, the solution of the flow equation in the subspace orthogonal to
$\text{ker}\ \Theta$ is
\begin{align}
    f^\perp_{t,k}
    \label{eq:FlowSolution}
        &= \hat{f}^\perp_{t,k} + \check{f}^\perp_{t,k}
        % &= \sum_{i=1}^{d^\perp}  \left(w^{(i)}, f^\perp_0\right) e^{-h^{(i)}t} w^{(i)}_k
        %     + \sideset{}{'}\sum_{i=1}  \frac{1}{h^{(i)}} \left(w^{(i)}, B^\perp\right)
        %         \left(1 - e^{-h^{(i)}t}\right) w^{(i)}_k
        \, .
\end{align}
Collecting all terms yields a simple (and useful!) expression,
\begin{align}
    \label{eq:AnalyticSol}
    f_{t,\alpha}
        = U(t)_{\alpha\alpha'} f_{0,\alpha'} + V(t)_{\alpha I} Y_{I}\, .
\end{align}
The two evolution operators $U(t)$ and $V(t)$ have lengthy, yet explicit, expressions, which we
summarise here: 
\begin{align}
    U(t)_{\alpha\alpha'} = \hat{U}^\perp(t)_{\alpha\alpha'}
        + \check{U}^\perp(t)_{\alpha\alpha'} + U^\parallel_{\alpha\alpha'}\, ,
\end{align}
where
\begin{align}
    \hat{U}^\perp(t)_{\alpha\alpha'}
        = \sum_{k,k'\in\perp} \sqrt{\lambda^{(k)}} z^{(k)}_\alpha 
            \left[\sum_i w^{(i)}_{k} e^{-h^{(i)}t} w^{(i)}_{k'}\right]
            z^{(k')}_{\alpha'} \frac{1}{\sqrt{\lambda^{(k')}}}\, ,
\end{align}
and
\begin{align}
    U^\parallel_{\alpha\alpha'}
        = \sum_{k''\in\parallel} z^{(k)}_\alpha z^{(k)}_{\alpha'} \, .
\end{align}
The contributions from $\check{U}^\perp(t)$ and $V(t)$ are more easily expressed by
introducing the operator
\begin{align}
    \label{eq:MOperatorDef}
    \mathcal{M}(t)_{\alpha\alpha'} 
        = \sum_{k,k'\in\perp} \sqrt{\lambda^{(k)}} z^{(k)}_\alpha 
            \left[\sideset{}{'}\sum_{i} w^{(i)}_{k} \frac{1}{h^{(i)}}\, 
            \left( 1- e^{-h^{(i)}t}\right) w^{(i)}_{k'}\right]
            z^{(k')}_{\alpha'} \frac{1}{\sqrt{\lambda^{(k')}}}\,. 
\end{align}
Then, we can write
\begin{align}
    \label{eq:UperpCheck}
    \check{U}^\perp(t)
        = - \mathcal{M}(t)\; \FKtabT C_Y^{-1} \FKtab 
            \left[\sum_{k''\in\parallel} z^{(k'')} z^{(k'') T}\right]\, ,
\end{align}
and
\begin{align}
    V(t) = \mathcal{M}(t)\; \FKtabT C_Y^{-1}\, ,
\end{align}
where we note that the term in the bracket in Eq.~\eqref{eq:UperpCheck} is simply the projector on the 
kernel of the NTK. 

\paragraph{Physical Interpretation.} The four terms that appear in the analytical solution have a clear physical interpretation. 
\begin{itemize}
    \item The first term $\hat{U}^\perp(t)$ suppresses the components of the initial condition that lie in the subspace orthogonal 
    to the kernel of the NTK. These are the components that are learned by the network during training. While the trained solution
    still depends on its value at initialization, that dependence is suppressed during training. This suppression 
    is exponential in the training time, and the rates are given by the eigenvalues of 
    $H^{\perp}$.
    \item The contribution from $U^\parallel$ yields the component of the initial condition that lies in the kernel of the NTK. 
    As such, those components remain unchanged during training and are part of the trained field at all times $t$. 
    \item The two remaining contributions are best understood by combining them together,
    \begin{align}
        \label{eq:DataCorrectedInference}
        \check{U}^{\perp}(t) f_{0} + V(t) Y 
            = \mathcal{M}(t)\; \FKtabT C_Y^{-1} \left[Y - \FKtab f_{0}^{\parallel}\right]\, .
    \end{align}
    The parallel component of the initial condition $f_{0}^{\parallel}$ does not evolve during training, and therefore it yields
    a contribution $\FKtab f_{0}^{\parallel}$ to the theoretical prediction of the data points at all times $t$. This is 
    taken into account by subtracting this
    contribution from the data, before the inference is performed.
\end{itemize}

% Collecting all terms yields a simple (and useful!) expression,
% \begin{align}
%     \label{eq:AnalyticSol}
%     f_{t,\alpha}
%         = U(t)_{\alpha\alpha'} f_{0,\alpha'} + V(t)_{\alpha I} Y_{I}\, .
% \end{align}
% The two evolution operators $U(t)$ and $V(t)$ have lengthy, yet explicit, expressions, which we
% summarise here or move to an appendix: \ac{$U^\parallel$ should not be time-dependent, right?}
% \begin{align}
%     U(t)_{\alpha\alpha'} = \hat{U}^\perp(t)_{\alpha\alpha'}
%         + \check{U}^\perp(t)_{\alpha\alpha'} + U^\parallel_{\alpha\alpha'}\, ,
% \end{align}
% where
% \begin{align}
%     \hat{U}^\perp(t)_{\alpha\alpha'}
%         &= \sum_i Z^{(i)}_{\alpha} e^{-h^{(i)}t} Z^{(i)}_{\alpha'}\, , \\
%     Z^{(i)}_{\alpha}
%         &= \sum_{k\in\perp} \sqrt{\lambda^{(k)}} z^{(k)}_\alpha w^{(i)}_{k}\, ,
% \end{align}
% and
% \begin{align}
%     \check{U}^\perp(t)_{\alpha\alpha'}
%         &= \sideset{}{'}\sum_{i} Z^{(i)}_{\alpha} \frac{1}{h^{(i)}} \left(1 - e^{-h^{(i)}t}\right) \tilde{Z}^{(i)}_{\alpha'}\, , \\
%     \tilde{Z}^{(i)}_{\alpha}
%         &= -\sum_{k'\in\perp} \sum_{k''\in\parallel} w^{(i)}_{k'} \sqrt{\lambda^{(k')}}
%             T^{(k') T} C_Y^{-1} T^{(k'')} z^{(k'')}_{\alpha}\, ,
% \end{align}
% and
% \begin{align}
%     U^\parallel_{\alpha\alpha'}
%         = \sum_{k\in\parallel} z^{(k)}_\alpha z^{(k)}_{\alpha'} \, ,
% \end{align}
% and
% \begin{align}
%     V(t)_{\alpha I} = \sideset{}{'}\sum_{i} Z^{(i)}_{\alpha} \frac{1}{h^{(i)}} \left(1 - e^{-h^{(i)}t}\right)
%         \tilde{T}^{(i)}_{I}\, ,
% \end{align}
% and
% \begin{align}
%     \tilde{T}^{(i)}_{I} = \sum_{k'\in\perp} w^{(i)}_{k'} \sqrt{\lambda^{(k')}}
%         T^{(k')}_J \left(C_Y^{-1}\right)_{JI}\, .
% \end{align}

\subsection{Behaviour of the solution}
\label{sec:BehaviourOfSolution}
The solution in Eq.~\eqref{eq:AnalyticSol} is the main result of this section. It shows that the
training process can be described as the sum of a linear transformation of the initial fields $f_{0,\alpha}$,
which are the preactivations of the output layer at initialization, and a linear transformation
of the data $Y_I$. The two transformations depend on the flow time $t$ and are given by the evolution
operators $U(t)$ and $V(t)$. Eq.~\eqref{eq:AnalyticSol} encode the information on the central value
and the variance of the trained fields, and any other quantity that is derived from the PDFs. 

\subsubsection{Central value of the trained fields.}
\label{sec:CentralValue}
The central values of the trained fields is obtained by taking the expectation value of
Eq.~\eqref{eq:AnalyticSol} over the initial fields, which are approximately Gaussian distributed at 
initialization, and over the fluctuations of the NTK,
\begin{align}
    \label{eq:MeanValAtT}
    \bar{f}_{t,\alpha} = \mathbb{E}\left[f_{t,\alpha}\right]
        = \mathbb{E}\left[U(t)_{\alpha\alpha'} f_{0,\alpha'}\right]
            + \mathbb{E}\left[V(t)_{\alpha I} Y_I\right] \, .
\end{align}
Note that the first term on the right-hand side of Eq.~\eqref{eq:MeanValAtT} can only be non-zero because of the
correlations between $U(t)$ and $f_0$. In the absence of such correlations, the first term would be given by the product
of the expectation values and hence would vanish up to corrections of order $\mathcal{O}(1/n)$, since the expectation
value of the fields at initialization vanishes in the limit of infinitely wide networks.
Assuming that the correlations between the initial fields and the evolution operators vanish, we can write
\begin{align}
    \label{eq:MeanUt}
    \bar{U}(t)
        &= \mathbb{E}\left[U(t)\right]\, , \\
    \label{eq:MeanVt}
    \bar{V}(t)
        &= \mathbb{E}\left[V(t)\right]\, ,
\end{align}
and
\begin{equation}
    \label{eq:MeanValAtTNoCorr}
    \bar{f}_{t,\alpha} = \bar{U}(t)_{\alpha\alpha'} \bar{f}_{0,\alpha'}
        + \bar{V}(t)_{\alpha I} Y_I = \bar{V}(t)_{\alpha I} Y_I \, .
\end{equation}

The second term in Eq.~\eqref{eq:MeanValAtT}, or equivalently Eq.~\eqref{eq:MeanValAtTNoCorr}, explicitly shows the contribution
of each data point to the central value of the
trained fields at each value of $x_{\alpha}$. It is worthwhile remarking that in this limit, the central value
from the set of trained networks is a linear combination of the data points, with coefficients given by the
evolution operator $V(t)_{\alpha I}$. In this respect, the trained NNs behave like other well-known linear methods for 
solving inverse problems, like \eg\ Backus-Gilbert or Gaussian Processes. It is interesting to compare the matrix $V(t)$ with 
the corresponding linear operators that enter in Backus-Gilbert or Gaussian Processes solutions. 

\paragraph{Central value at infinite training time.}

In the limit of infinite training time, the evolution operators $U(t)$ and $V(t)$ simplify and yield an 
elegant interpretation of the minimum of the cost function. For large training times, we have
\begin{align}
    \label{eq:UhatInfty}
    \hat{U}^\perp_{\infty, \alpha\alpha'}
        &= \lim_{t\to\infty}\hat{U}^\perp(t)_{\alpha\alpha'} = 0\, \\
    \label{eq:MOperatorInfty}
    \mathcal{M}_{\infty, \alpha\alpha'} 
        &= \lim_{t\to\infty}\mathcal{M}(t)_{\alpha\alpha'} = \sum_{k,k'\in\perp} \sqrt{\lambda^{(k)}} z^{(k)}_\alpha 
        \left[\sideset{}{'}\sum_{i} w^{(i)}_{k} \frac{1}{h^{(i)}}\, 
        w^{(i)}_{k'}\right] z^{(k')}_{\alpha'} \frac{1}{\sqrt{\lambda^{(k')}}}\, ,
\end{align}
and explicit expressions for $\check{U}^\perp_{\infty}$ and $V_{\infty}$ are obtained from $\mathcal{M}_{\infty}$.
The term in the square bracket in Eq.~\eqref{eq:MOperatorInfty} is the spectral decomposition of the pseudoinverse 
of $H^\perp$ in $d_\perp$ orthogonal subspace. So, the operator $\mathcal{M}_{\infty}$ acts as follow on a field 
$f_{\alpha}$:
\begin{enumerate}
    \item The term on the right of the square bracket computes the coordinate $f_k$ introduced in 
    Eq.~\eqref{eq:OrthogonalComponents}. The $f_k$ are a set of coordinates for the component $f^\perp$ 
    of the field that evolves during training, 
    \begin{align}
        \label{eq:RightOfTheBracket}
        f^\perp = \sum_{k\in\perp} \sqrt{\lambda^{(k)}} f_k\, z^{(k)}\,  .
    \end{align}
    \item The term in the square bracket applies the pseudoinverse to the coordinates $f_k$, 
    \begin{align}
        \label{eq:ApplyPseudoInv}
        f'_k = \left(H^\perp\right)^+_{kk'} f_{k'}\, .
    \end{align}
    \item The final term on the left of the square bracket reconstructs the full field corresponding to the modified 
    $f'_{k}$,
    \begin{align}
        \label{eq:LeftOfTheBracket}
        f^{'\perp} = \sum_{k\in\perp} \sqrt{\lambda^{(k)}} f'_{k}\, z^{(k)}\, .
    \end{align}
    
\end{enumerate}

As discussed at the end of Sect.~\ref{sec:Lazy} it is convenient to 
combine the contributions of $\check{U}^\perp_{\infty}$ and $V_{\infty}$,
\begin{align}
    \label{eq:DataCorrectedInferenceAtInfty}
    \check{U}^{\perp}_{\infty} f_{0} + V_{\infty} Y 
        = \mathcal{M}_{\infty}\; \FKtabT C_Y^{-1} \left[Y - \FKtab f_{0}^{\parallel}\right]\, .
\end{align}
The contribution to the observables from the parallel components of $f$ does not change during training, 
therefore that contribujtion is subtracted to the data and the orthogonal components of $f$ are adjusted to 
minimize the $\chi^2$ of the corrected data. The minimum of the $\chi^2$ in the orthogonal subspace is found
applying $\mathcal{M}_{\infty}$, \ie by projecting in the orthogonal subspace, applying the pseudoinverse and 
finally recompute the full field as detailed above. 

\subsubsection{Covariance of the trained fields.}
\label{sec:Covariance}
For the covariance we have 
\begin{align}
    \cov[f_t,f_t^T]
        &= \mathbb{E}\left[U(t) f_0 f_0^T U(t)^T\right] 
            - \mathbb{E}\left[U(t) f_0\right] \mathbb{E}\left[f_0^T U(t)^T\right]  \nonumber \\
        &\quad + \mathbb{E}\left[U(t) f_0 Y^T V(t)^T\right] 
            - \mathbb{E}\left[U(t) f_0\right] \mathbb{E}\left[Y^T V(t)^T\right] \nonumber \\
        &\quad + \mathbb{E}\left[V(t) Y f_0^T U(t)^T\right]
            - \mathbb{E}\left[V(t) Y\right] \mathbb{E}\left[f_0^T U(t)^T\right] \nonumber \\
    \label{eq:CovAtT}
        &\quad + \mathbb{E}\left[V(t) Y Y^T V(t)^T\right]
            - \mathbb{E}\left[V(t) Y\right] \mathbb{E}\left[Y^T V(t)^T\right] \, .
\end{align}
Note that the first and the fourth lines above yield symmetric matrices, while the third line is just the 
transpose of the second, thereby ensuring that the whole covariance matrix is the sum of three symmetric 
matrices and therefore is symmetric, 
\begin{align}
    \label{eq:SumOfCovariances}
    \cov[f_t,f_t^T] = C^{(00)} + C^{(0Y)} + C^{(YY)}\, ,
\end{align}
where
\begin{align}
    \label{eq:C00term}
    C^{(00)} 
        &= \mathbb{E}\left[U(t) f_0 f_0^T U(t)^T\right] 
        - \mathbb{E}\left[U(t) f_0\right] \mathbb{E}\left[f_0^T U(t)^T\right]\, ,\\
    C^{(0Y)}
        &= \mathbb{E}\left[U(t) f_0 Y^T V(t)^T\right] 
        - \mathbb{E}\left[U(t) f_0\right] \mathbb{E}\left[Y^T V(t)^T\right] \nonumber \\
        \label{eq:C0Yterm}
        &\quad + \mathbb{E}\left[V(t) Y f_0^T U(t)^T\right]
            - \mathbb{E}\left[V(t) Y\right] \mathbb{E}\left[f_0^T U(t)^T\right] \, ,\\
    C^{(YY)}
        &= \mathbb{E}\left[V(t) Y Y^T V(t)^T\right]
        - \mathbb{E}\left[V(t) Y\right] \mathbb{E}\left[Y^T V(t)^T\right]\, .
\end{align}
We report the values of $C^{(00)}$, $C^{(0Y)}$ and $C^{(YY)}$ for different values of $t$ in 
Fig.\ref{fig:CovarianceContribs}.
\ldd{Here we need plots at many values of $t$, in order to figure out what is actually going on with these three 
contributions along training. Hoepfully we see explicitly that the fluctuations of the fields at initialization yield a 
contribution to the statistical uncertainty of the trained fields at early times, and that contribution
decreases.}

\begin{figure}
    \label{fig:CovarianceContribs}
    \caption{The various contributions to the covariance as described in the text above. 
    \ldd{Caption will change once we have the plot.}}
\end{figure}

% % wrong limit
% In the limit where we ignore the fluctuations 
% of $U(t)$ and $V(t)$, we obtain
% \begin{align}
%     \cov[f_t,f_t^T]
%         &= \bar{U}(t) K \bar{U}(t)^T 
%             + \bar{V}(t) C_Y \bar{V}(t)^T\, ,
% \end{align}
% where we used the fact that the fluctuations of the data $Y$ are statistically independent from 
% the fluctations of the fields at initialization. 

\ldd{To be rewritten when we have the plots mentioned above.}

An important criterion that we want to put forward at this point is that the covariance of the trained
fields should be dominated by the statistical error on the data. This is the case towards the end of
training as shown in Fig.~\ref{fig:xT3_exp_val}. In this way we ensure that the quoted error 
on the PDFs is actually dominated by the statistical error on the data, and not by the fluctuations of the
initial fields. This is a crucial point, since it guarantees that the ensemble of trained PDFs is not biased by the
choice of prior that is made by selecting a given architecture, activation function, and probability distribution 
for the biases and weights at initialization. \ldd{More plots needed. To be discussed.}

\FloatBarrier


\section{Behaviour of the NTK}
\label{sec:NTKPheno}
\section{Behaviour of the NTK}
\label{sec:NTKPheno}

\ac{General comments: the paging of the figuers is all messed up, but I will fix it once we
know what we want to keep (and discard).}
The NTK, introduced in Sec.~\ref{sec:GradFlow}, provides a powerful framework
for understanding neural network dynamics during training. Originally developed
by Jacot et al.~\cite{jacot2018neural} to analyze infinite-width feed-forward
networks, the NTK theory has since been extended to diverse architectures
including convolutional networks~\cite{arora2019exact} and recurrent
networks~\cite{alemohammad2021recurrent}. This theoretical framework has proven
invaluable for characterizing learning dynamics and generalization properties
across various network designs.

From Eq.~\eqref{eq:FlowEquationNoIndices}, we observe that the NTK encodes the
dependence on the architecture of the network and governs its training dynamics.
The analysis of the NTK properties is thus crucial for understanding the network
behaviour during training. We first show the properties of the NTK at
initialisation, before moving to the training phase, where we provide a detailed
study of the NTK in the context of the NNPDF methodology. To this end, we
performed a fit of $T_3$ using the NNPDF methodology with the dataset described
in Sec.~???. We initialized an ensemble of $N_{\rm Rep} = 100$ replicas with
identical architecture, training each independently using gradient descent (GD)
optimization. As our focus here is on NTK properties rather than physical
predictions, we employ synthetic data with controlled noise characteristics,
namely L0, L1, and L2 data generations. Throughout training, we track the
evolution of the NTK to understand how the network's effective dynamics change
as it learns the target function.

% ===================================
\begin{figure}[t!]
  \centering
  \includegraphics[width=0.90\textwidth]{../plots/ntk_initialization_with_uncertainty.pdf}
  \caption{Frobenius norm of the NTK at initialisation, $\lVert \Theta_0
  \rVert$, in function of the width of the network. On the left, central values
  and uncertainty bands are obtained as mean and one-sigma deviation of the
  ensemble of networks (left). The relative uncertainty is also shown (right)}
  \label{fig:NTKInit}
\end{figure}
% ===================================

% ===================================
\begin{figure}[t!]
  \centering
  \includegraphics[width=0.90\textwidth]{plots/ntk_eigenvalue_spectrum.pdf}
  \caption{Spectrum of the NTK for the first three architectures displayed in
  Fig.~\ref{fig:NTKInit}. \ac{I'm thinking of removing uncertainty bands.}}
  \label{fig:NTKSpectrum}
\end{figure}
% ===================================
We first discuss the properties of the NTK at initialisation, that is when the
network is not blind to data. We remind that, at this stage, the NTK depends on
the $x$-grid of input and on the architecture. It is argued in the literature
that, in the large-width limit, the variance of the NTK tends to zero with the
size of the architecture (see, \textit{e.g.}, \cite{Roberts:2021fes}). In order
to assess this property of the NTK, we computed the Frobenius norm of the NTK
over an ensemble of networks for different architectures. For each architecture,
we took mean value and standard deviation as statistical estimators of the
ensemble. The result is displayed in Fig.~\ref{fig:NTKInit}. From this figure,
we can confirm that the variance of the NTK becomes smaller with the size of the
network. Note that, in addition to the scaling $\mathcal{O}(1/n)$ theoretically
predicted in the large networks, the uncertainty bands include bootstrap errors
due to the finite size of the ensemble. \ac{I can say something more here.}

Another feature of the NTK is shown in Fig.~\ref{fig:NTKSpectrum}, where the
spectrum of the NTK is shown for four different architectures. As debated in the
literature, the spectrum of the NTK is heavily hierarchical, and only few
eigenvalues are actually non-zero\footnote{Note that, due to the large
difference in magnitude of the eigenvalues, the relative precision of the
machine introduces noise noise in the decomposition, so that small eigenvalues
should be effectively considered zero. We will discuss the cut-off tolerance
later, when will discuss the training process in more details.}. This means that
only a small subset of active directions can inform the network during training,
as it will be discussed later. Note that, at least at initialisation, these
observations do not depend on the architecture.

% ===================================
\begin{figure}[h!]
  \centering
  \includegraphics[width=0.6\textwidth]{../plots/delta_ntk.pdf}
  \caption{Variation of the NTK during training for L0, L1, and L2 data. Error
  bands correspond to one-sigma uncertainties over the ensemble of networks.}
  \label{fig:NTKTime}
\end{figure}
% ===================================
Having established the properties of the NTK at initialisation, we now discuss
its behaviour during training. In the machine learning literature, it is argued
that the NTK remains constant during training provided that the width of the
network is large enough. Here we show that this is not the case, at least for
the NNPDF methodology. In Fig.~\ref{fig:NTKTime}, we show the variation of the
Frobenius norm of the NTK during training for three different datasets, L0, L1,
and L2. We observe that the NTK does not remain constant during training, but
rather it tends to change with time. In the figure, we can identify two
different phases. The first one covers the initial part of the training. From
Fig.~\ref{fig:NTKTime}, we see that the is rather sensitive to the evolution, in
strong contrast with the observations argued in the machine learning literature.
Note also that this initial peak is more pronounced for L2 data. This is
consistent with the fact that the NTK (\textit{i.e.}, the architecture) needs to
accommodate the noise in the data, thus leading to a larger variation of the
NTK. On the other hand, after this initial phase, the NTK tends to stabilize. We
will refer to this second phase as the \textit{lazy training}, in keeping with
the terminology adopted in the literature. We conclude that, in this phase, the
NTK does not change significantly. As a consequence, this suggests that the
theory of the infinite-width networks during training can be applied only after
the initial phase, when the NTK has stabilized, as discussed later in this
article.

% ===================================
\begin{figure}[h!]
  \centering
  \includegraphics[width=0.48\textwidth]{../plots/eigval_1.pdf}
  \includegraphics[width=0.48\textwidth]{../plots/eigval_2.pdf}
  \includegraphics[width=0.48\textwidth]{../plots/eigval_3.pdf}
  \includegraphics[width=0.48\textwidth]{../plots/eigval_4.pdf}
  \includegraphics[width=0.48\textwidth]{../plots/eigval_5.pdf}
  \vspace{0.5cm}
  \caption{The first five eigenvalues of the NTK for L0, L1, and L2 data. Error
  bands correspond to one-sigma uncertainties over the ensemble of networks.}
  \label{fig:EigvalsComparison}
\end{figure}
% ===================================
One may ask how the eigenvalues of the NTK contribute to the variation of the
NTK. In Fig.~\ref{fig:EigvalsComparison}, the first five eigenvalues of the NTK
are displayed for L0, L1, and L2 data. We can make a few observations upon
inspecting these plots. First, we notice that the way in which data is generated
has an impact on the eigenvalues of the NTK. In general, the uncertainty bands
for L2 data are larger than those for L1 and L0 data, indicating that the NTK is
more sensitive to the noise in the data. This is consistent with the observation
made in Fig.~\ref{fig:NTKTime}. Second, we observe that the initial hierarchy of
the eigenvalues, Fig.~\ref{fig:NTKSpectrum}, is not preserved during training.
While the first eigenvalue remains dominant, the other eigenvalues grow with
time. This fact, combined with the analysis of Eq.~\eqref{eq:FlowSolution}
\ac{(this reference will be likely changed)}, suggests that more ``physical''
features become learnable during training. Most importantly, being these values
non-zero, they can be learned in a finite time and not require
$T\rightarrow\infty$ to be learned, which is unpractical.

% ===================================
\begin{figure}[h!]
  \centering
  \includegraphics[width=0.6\textwidth]{../plots/delta_ntk_arch.pdf}
  \caption{Comparison of the variation of the NTK during training for two
  different architectures with sizes $[28,20]$ and $[100,100]$ respectively. In
  both cases, L1 data are used. Error bands correspond to one-sigma
  uncertainties over the ensemble of networks.}
  \label{fig:NTKTimeDiffArch}
\end{figure}
% ===================================

% ===================================
\begin{figure}[h!]
  \centering
  \includegraphics[width=0.48\textwidth]{../plots/eigval_1_arch_L0.pdf}
  \includegraphics[width=0.48\textwidth]{../plots/eigval_1_arch_L1.pdf}
  \includegraphics[width=0.48\textwidth]{../plots/eigval_1_arch_L2.pdf}
  \vspace{0.5cm}
  \caption{Comparison of the first eigenvalue of the NTK obtained using two
  different architectures in the case of L0 (upper-left), L1 (upper-right), and
  L2 (bottom) generated data. Error bands correspond to one-sigma uncertainties
  over the ensemble of networks.}
  \label{fig:NTKvalsDiffArch}
\end{figure}
% ===================================
Then there is the dependence on the architecture...


% ===================================
\begin{figure}[h!]
  \centering
  \includegraphics[width=0.23\textwidth]{plots/overlap_epoch_0.pdf}
  \includegraphics[width=0.23\textwidth]{plots/overlap_epoch_10000.pdf}
  \includegraphics[width=0.23\textwidth]{plots/overlap_epoch_30000.pdf}
  \includegraphics[width=0.23\textwidth]{plots/overlap_epoch_50000.pdf}
  \vspace{0.5cm}
  \caption{Matrix $A$ as defined in Eq.~\eqref{eq:MatrixA} for L2 data and for a single
  replica of the NTK. The matrix is shown at different epochs of the training process,
  indicated in the top of each panel.}
  \label{fig:NtkMAlign}
\end{figure}
% ===================================
It has been argued before that there is a non-trivial interplay between the eigenspace of
the NTK and that of the matrix $M$. Indeed, the former encodes the model dependence, while
the latter brings physical information. Of course the two matrices are independent at
initialisation, and we do not expect any alignment patter between the two. However, this picture
might change during training, as the NTK evolves and the model learns the target function.
To quantify this alignment, we define the matrix $A$ as
\begin{equation}
  A_{kk'} = \left( \left< z^{(k)}, v^{(k')}\right> \right)^2 = \cos^2(\theta_{kk'}) \;,
  \label{eq:MatrixA}
\end{equation}
where $z^{(k)}$ and $v^{(k')}$ are the $k$-th and $k'$-th eigenvectors of the NTK and
$M$, respectively. The matrix $A$ is thus a measure of the alignment between the eigenspaces of
the two matrices. In Fig.~\ref{fig:NtkMAlign}, we show the matrix $A$ at different epochs of
the training for L2 data and for a single NTK replica.


\section{Training and Closure}
\label{sec:TrainClosure}
The analytical solution in \eqref{eq:AnalyticSol} can
shed a new light onto the behaviour of the numerical neural network training.
In order to study the training process, the NNPDF collaboration has successfully developed so-called
{\em closure tests}. A closure test uses synthetic data, generated using a known
set of PDFs, to train the neural network. The PDFs used for genrating the data are called here {\em input}\
PDFs. The results of the training are then compared to the
known input PDFs; the performance of the training algorithm and the NN
architecture are assessed by quantifying the comparison between trained PDFs and input PDFs.
Following the original presentation in Ref.~\cite{NNPDF:2014otw}, we distinguish three
levels of closure tests, which are defined by the complexity of the data used to train the NNs.
We use the standard NNPDF nomenclature and refer to these three levels as level-0 (L0), level-1 (L1),
and level-2 (L2) closure tests and we denote the input PDFs used to generate the data as $\fin$.

Let us start by discussing the case of L0 tests. In this case, the data are given by
\begin{equation}
    \label{eq:DataL0}
    Y_I = T[\fin]_I
        = \sum_{i=1}^{\nflav} \sum_{\alpha=1}^{\ngrid} \FKtab_{Ii\alpha} \fin_{i\alpha}\, ,
\end{equation}
or equivalently, suppressing the indices,
\begin{equation}
    \label{eq:DataL0NoIndices}
    Y = \FKtab \fin\, .
\end{equation}
Using L0 data in the analytical expression for the trained network in Eq.~\eqref{eq:AnalyticSol} allows
a simplification of the second term,
\begin{align}
    \label{eq:L0ClosureTrained}
    V(t) Y = \sumprime_{i} Z^{(i)} \left(1 - e^{-h^{(i)}t}\right) 
        \sum_{k\in\perp} w^{(i)}_k \fin_k\, .
\end{align}
Interestingly, for $t\to\infty$, 
\begin{align}
    \label{eq:L0ClosureInfiniteTraining}
    \lim_{t\to\infty} V(t) Y = \finperp\, ,
\end{align}
and therefore the $V$ component of the trained solution reproduces exactly the 
component of the PDF that lies in  the subspace orthogonal to the kernel of $\Theta$.


Distance between the numerical minimization and the analytical formula as a function of
training time $t$ 

\ldd{Compute distances between numerical and analytical solutions using the standard NNPDF definition of the
distance. Same as the ones that are below, but using L0 data}


\newpage



\begin{figure}[t!]
  \label{fig:xT3_exp_val}
  \centering
  \includegraphics[width=0.65\textwidth]{plots/xT3_exp_val_early.pdf} \\
  \includegraphics[width=0.65\textwidth]{plots/xT3_exp_val_eot.pdf}
  \caption{Expectation value of the product (blue) and product of the
  expectation value for the $U$ contribution. Plots obtained using $t_{\rm ref}
  = 30000$, while $f_0$ is an ensemble of networks at initialisation
  (\textit{i.e.} $\mathbb{E}[f_0]=0$). \ac{These plots will be modified (font
  size, etc...) to match the other figures.}} \ldd{Error bars needed}
\end{figure}
% ===================================


\FloatBarrier


\section{First Training Results}
\label{sec:FirstResults}
Just one more line to test github

 results with real data
  
  
\FloatBarrier


\bibliographystyle{unsrt}
\bibliography{ntk.bib}

\appendix
\appendix
\section{The BCDMS dataset for $T_3$}
\label{app:dataset}

The analysis presented in this work employs a set of synthetic data points
generated using a known underlying law $\fin$ that we seek to reconstruct.
Analogous to Ref.~\cite{Candido:2024hjt}, the pseudo-data are constructed by
convolving $\fin$ with FK tables whose kinematic dependence is specified by
measurements of the structure function $F_2$ on proton and deuterium targets
from the BCDMS collaboration~\cite{Benvenuti:1989fm}. By combining these
measurements, one can construct the observable $F_2^p - F_2^d$ that, at
next-to-next-to-leading order (NNLO) in QCD and under the assumption of
isoscalarity for the deuterium nucleus, provides a clean probe of the
non-singlet triplet PDF combination $T_3 = u^+ - d^+$, where $u^+ = u + \bar{u}$
and $d^+ = d + \bar{d}$. The factorisation formula in Eq.~\eqref{eq:TheoryPred}
then simplifies to 
\begin{equation}
F_2^p - F_2^d = C_{T_3} \otimes T_3,
\end{equation}
where $C_{T_3}$ is the corresponding Wilson coefficient computed in perturbative
QCD, and $\otimes$ is a short-hand notation that denotes the convolution as in
Eq.~\eqref{eq:TheoryPred}. The construction of the FK tables needed to compute
the predictions, as well as the construction of the covariance matrix, are
identical to Ref.~\cite{Candido:2024hjt}, to which the reader is referred for
further details. This yields a total of 333 data points, which then reduce to
248 points after applying the kinematic cuts.

Following the closure test framework developed by the NNPDF
collaboration~\cite{DelDebbio:2021whr,NNPDF:2021njg}, we generate
data with three different levels of noise, labelled as Level 0 (L0), Level 1 (L1)
and Level 2 (L2) data. Furthermore, we use the non-singlet triplet $xT_3$ from the
NNPDF4.0 parton set~\cite{NNPDF:2021njg} as the input law $\fin$. In the following,
we summarise the definition of the different levels of pseudo-data.

\paragraph{Level 0}
The pseudo-data are generated without any experimental noise, \ie, by using
the input function and the FK tables as follows
\begin{equation}
Y_{L0} =  \FKtab \fin.
\end{equation}
In this ideal scenario, the analysis should reproduce the input $\fin$, though
some residual reconstruction error may remain in the kinematic region not
covered by the FK tables. Level 0 assesses the intrinsic bias of the methodology,
as any neural network replica will be trained on the same data points $Y_{L0}$.

\paragraph{Level 1}
In this case, the experimental noise is added on top of the L0 data, by sampling from the
multivariate normal distribution with the full experimental covariance matrix
$C_Y$ provided by the BCDMS collaboration
\begin{equation}
Y_{L1} =  Y_{L0} + \eta, \quad \textrm{where} \quad \eta \sim \mathcal{N}(0, C_Y).
\end{equation}
This case is closer to actual experimental data, where the ``true'' value is
blurred by the presence of noise. Note however that we are not yet propagating
the experimental uncertainties into the uncertainties of the fitted PDF, as the added
noise is fixed over all replicas.

\paragraph{Level 2}
Finally, we generate L2 pseudo-data by adding a different noise realisation to each
replica, sampled from the same multivariate normal distribution
\begin{equation}
Y_{L2}^{(k)} =  Y_{L1} + \xi^{(k)}, \quad \textrm{where} \quad \xi^{(k)} \sim \mathcal{N}(0, C_Y).
\end{equation}
This represents the most realistic scenario where both model and data uncertainties
are present. In this case, each neural network replica will be trained on a
different set of data points $Y_{L2}^{(k)}$.

\end{document}
