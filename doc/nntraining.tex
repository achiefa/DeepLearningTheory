\documentclass[11pt,a4paper]{article}

\usepackage{xcolor}
\usepackage[colorlinks=true, linkcolor=black!50!blue, urlcolor=blue, citecolor=blue, anchorcolor=blue]{hyperref}
\usepackage[font=small,labelfont=bf,margin=0mm,labelsep=period,tableposition=top]{caption}
\usepackage[a4paper,top=3cm,bottom=2.5cm,left=2.5cm,right=2.5cm,bindingoffset=0mm]{geometry}
\usepackage{amsmath}
\usepackage{amsfonts}
\usepackage{amssymb}
\usepackage{authblk}
\usepackage{dsfont}
\usepackage{pifont}
\usepackage{booktabs}
\usepackage{tabularx}
\usepackage{siunitx}
\usepackage{graphicx}
\usepackage{epstopdf}
\usepackage{epsfig}
\usepackage{framed}
\usepackage{makeidx}
\usepackage{simplewick}
\usepackage{placeins}
\usepackage{bbold}
\usepackage{braket}
\graphicspath{{./figs/}}

% barn deprecated by siunitx
% \DeclareSIUnit{\barn}{b}

% General global commands
%\newcommand{\cov}{C}
\newcommand{\posterior}[1]{\tilde{#1}}
\newcommand{\real}{\mathbb{R}}
\renewcommand{\textin}{\text{in}} 

% General inverse Problems
% map operators
\newcommand{\fwdmapop}{G}
\newcommand{\obsop}{O}
\newcommand{\fwdobsop}{\mathcal{\fwdmapop}}
% model space
\newcommand{\nmodel}{N_{\rm model}}
\newcommand{\modelspace}{X}
\newcommand{\modelvec}{u}
\newcommand{\modelpriorcent}{\modelvec_0}
\newcommand{\modelpriorcov}{\cov_X}
\newcommand{\modelpostcent}{\posterior{\modelvec}}
\newcommand{\modelpostcov}{\posterior{\cov}_X}
% observable space
\newcommand{\ndata}{N_{\rm data}}
\newcommand{\obs}{y}
\newcommand{\obspriorcent}{\obs_0}
\newcommand{\obspriorcov}{\cov_{Y}}
\newcommand{\obsnoise}{\eta}
\newcommand{\obspostcent}{\posterior{\obs}}
\newcommand{\obspostcov}{\posterior{\cov}_Y}
% linear map
\newcommand{\linmap}{\fwdobsop}
\newcommand{\vander}{\mathcal{X}}
\newcommand{\nlaw}{N_{\rm law}}

% NNPDF data/model
\newcommand{\law}{f}
\newcommand{\pseudodat}{\mu}
\newcommand{\noise}{\epsilon}
\newcommand{\repind}{(k)}
\newcommand{\modelvecrep}{\modelvec_*^{\repind}}
% likelihood
\newcommand{\likelihood}{\mathcal{L}}
\newcommand{\repchis}{{\chi^2}^{\repind}}

% closure fitting
\newcommand{\lawmodel}{w}
\newcommand{\utrue}{u_\mathrm{true}}
\newcommand{\uest}{u_\mathrm{est}}

% closure estimators
\newcommand{\testset}[1]{ {{#1}^{\prime}} }
\newcommand{\emodel}[1]{ \mathbf{E}_{\{ \modelvec_* \}} \left[ #1 \right] }
\newcommand{\eout}{\mathcal{E}^{\rm out}}

\newcommand{\nfits}{N_{\rm fits}}
\newcommand{\nreps}{N_{\rm rep}}

\newcommand{\bias}{{\rm bias}}
\newcommand{\var}{{\rm variance}}
\newcommand{\covrep}{\testset{\cov}^{(\rm replica)}}
\newcommand{\covcent}{\testset{\cov}^{(\rm central)}}
\newcommand{\biasvarratio}{\mathcal{R}_{bv}}

% quantile estimators
\newcommand{\xisigdat}[1]{\xi^{(\rm data)}_{#1 \testset{\sigma}}}
\newcommand{\xisigdati}[1]{\xi^{(\rm data)}_{#1 {\testset{\sigma}_i}}}
\newcommand{\modelstd}{\hat{\sigma}}
\newcommand{\erf}{{\rm erf}}

% abbreviations
\newcommand{\ie}{{\it i.e.}}
\newcommand{\eg}{{\it e.g.}}
\newcommand{\viz}{{\it viz.}}

% delta chi2 appendix
\newcommand{\ein}{\mathcal{E}^{\rm in}}
\newcommand{\deltachi}{\Delta_{\chi^2}}
\newcommand{\noisecross}{{\rm noise \, cross \, term}}

% -------------------%
% deprecated commands
% -------------------%
\newcommand{\vv}[1]{\mathbf{#1}}

\newcommand{\vecdiffreptwo}{\left( \vv{\model}^{\repind} - \vv{\levtwo}^{\repind}  \right)}
\newcommand{\vecdiffcentone}{\left( \erep{\vv{\model}} - \vv{\levone} \right)}

\newcommand{\coveig}{\sigma^{2}}
\newcommand{\diag}[1]{\hat{#1}}

\newcommand{\levelonediff}{\Delta}
\newcommand{\underlyingdiff}{u}
\newcommand{\repdiff}{v}

\newcommand{\shiftcross}{{\rm shift \, cross \, term}}
\newcommand{\deltaeps}{\Delta_{\epsilon}}
\newcommand{\kldiv}{D_{KL}}

\newcommand{\diffreptwo}{\left( \model^{\repind} - \levtwo^{\repind} \right)}
\newcommand{\diffcentone}{\left( \erep{\model} - \levone \right)}
\newcommand{\diffcentunder}{\left( \erep{\model} - \law \right)}
\newcommand{\diffcentrep}{\left( \erep{\model} - \model^{\repind}\right)}

\newcommand{\invcov}[1]{\cov^{-1}_{#1}}
\newcommand{\erep}[1]{\mathbf{E}_{\noise}\left[ #1 \right]}
\newcommand{\eshift}[1]{\mathbf{E}_{\shift}\left[ #1 \right]}

\newcommand{\model}{\fwdobsop}
\newcommand{\shift}{\obsnoise}
\newcommand{\invcovprime}{C_D^{\prime -1}}
\newcommand{\levone}{z}
\newcommand{\levtwo}{y}

\newcommand{\npoints}{N_{\rm points}}
\newcommand{\nfit}{\texttt{n3fit}}

\newcommand{\ndat}{N_{\mathrm{dat}}}
\newcommand{\ngrid}{N_{\mathrm{grid}}}
\newcommand{\nflav}{N_{f}}
\newcommand{\NThetaPar}{N_{\Theta\parallel}}
\newcommand{\NThetaPerp}{N_{\Theta\perp}}   
%\newcommand{\nmodel}{N_{\mathrm{model}}}
\newcommand{\cov}{\mathrm{Cov}}
\newcommand{\FKtab}{(\mathrm{FK})}
\newcommand{\FKtabT}{(\mathrm{FK})^T}
\newcommand{\GP}{\mathcal{GP}}
\newcommand{\lat}{{\mathrm{lat}}}
\newcommand{\lin}{{\mathrm{lin}}}
\newcommand{\PDF}{\textrm{PDF}}
\newcommand{\B}{\mathcal{B}} % Basis
\newcommand{\RPDF}{\mathbb{R}^{\PDF}}
\newcommand{\RRPDF}{\mathbb{R}^{\PDF \times \PDF}}
\newcommand{\ddt}{\frac{d}{dt}}
\newcommand{\fin}{f^{\rm in}}
\newcommand{\finperp}{f^{\rm in \perp}}
\newcommand{\finpar}{f^{\rm in \parallel}}
\newcommand{\sumprime}{\sideset{}{'}\sum}

% Matrix
\newcommand{\bpmat}{\begin{pmatrix}}
\newcommand{\epmat}{\end{pmatrix}}
\newcommand{\red}[1]{\textcolor{red}{#1}}
\newcommand{\ldd}[1]{\textcolor{red}{\textbf{Luigi: #1}}}
\newcommand{\ac}[1]{\textcolor{red}{\textbf{Amedeo: #1}}}

\begin{document}
\newgeometry{top=1.5cm,bottom=1.5cm,left=1.5cm,right=1.5cm,bindingoffset=0mm}
\vspace{-2.0cm}
\begin{flushright}
Edinburgh 2025/x
\end{flushright}
\vspace{0.3cm}

\begin{center}
  {\Large \bf On the Origin of PDF Uncertainties}
%  {\Large \bf Diagnostic Tools for Parton Distribution Function Fits}
%  {\Large \bf Towards a Quantitative Understanding of PDF Fits and their Uncertainties}
  
  \vspace{1.0cm}

\vspace{1.1cm}

  Amedeo Chiefa, Luigi Del Debbio and Richard Kenway

  \vspace{0.2cm}

  {\it \small
    The Higgs Centre for Theoretical Physics, \\ 
    School of Physics and Astronomy, The University of Edinburgh,\\
    Peter Guthrie Tait Road, Edinburgh EH9 3FD, United Kingdom\\[0.1cm]
  }
  \vspace{0.7cm}
\end{center}

\begin{abstract}
  Parton Distribution Functions (PDFs) play a central role in describing
  experimental data at colliders and provide insight into the 
  structure of nucleons. As the LHC enters an era of high-precision measurements, a robust
  PDF determination with a reliable uncertainty quantification has become
  mandatory in order to match the experimental precision. The NNPDF
  collaboration has pioneered the use of Machine Learning (ML) techniques for PDF
  determinations, using Neural Networks (NNs) to parametrise the unknown PDFs 
  in a flexible and unbiased way. The NNs are then trained on experimental data
  by means of stochastic gradient descent algorithms. The statistical robustness 
  of the results is validated by extensive closure tests using synthetic data.
  In this work, we develop a theoretical framework based on the
  Neural Tangent Kernel (NTK) to analyse the training dynamics of neural
  networks. This approach allows us to derive, under precise assumptions, an
  analytical description of the neural networks evolution during training,
  enabling a quantitative understanding of the training process. 
  Having an analytical handle on the training dynamics allows us to clarify the
  role of the NN architecture and the impact of the experimental data in a 
  transparent way. Similarly, we are able to describe the evolution of the covariance of
  the NN output during training, providing a quantitative description of how
  uncertainties are propagated from the data to the fitted function.
  Interestingly, the methodology developed in this work can be used to understand the 
  minimization of a loss function for any kind of parametrization, thereby 
  providing a unified framework to compare different PDF determinations, like,
  \eg, fits based on a particular functional form. While our results are {\em not}\ 
  a substitute for PDF fitting, they do provide a powerful diagnostic tool to
  assess the robustness of current fitting methodologies.
  Beyond its relevance for particle physics phenomenology, our analysis of 
  PDF determinations provides a testbed to apply theoretical ideas about the 
  learning process developed in the ML community. As seen in applications from 
  other domains, we find that our results deviate from the simple picture 
  of the \textit{lazy training} regime discussed in the ML literature.
\end{abstract}

\newpage

\tableofcontents
\clearpage

% Main Sections
\section{Introduction}
\label{sec:intro}

Parton Distribution Functions (PDFs) play a crucial role in describing
experimental data at hadron colliders and in gaining insights into the internal
structure of the proton. The high-precision era of particle physics that we are
now witnessing calls for equally precise theoretical predictions. Since PDFs are
a key ingredient in these predictions, the need for robust PDF determinations
with reliable uncertainty quantification has become increasingly important for
both Standard Model measurements and searches for new physics.

As non-perturbative objects, PDFs cannot be computed from first principles but
must be extracted from global analyses to experimental data. However, PDF
determination is a classic example of an \textit{inverse problem}, as it
involves inferring a continuous function from a finite set of data points. This
process is inherently ill-defined, and the limited amount of experimental
information prevents us from obtaining a unique solution to the problem. The
solution will inevitably depend on the assumptions made and on the prior
knowledge introduced to regularise the problem, either explicitly stated or
implicitly embedded in the fitting framework.

The complex nature of inverse problems has prompted the development of
sophisticated statistical methods, cutting-edge methodologies, and advanced
tools to tackle them. In general, PDF determinations can be broadly classified
into two main categories, depending on whether a specific functional form is
assumed for the PDFs or whether a non-parametric approach is adopted. Although
the former approach has been widely used in the literature, non-parametric
approaches based on Bayesian inference have been successfully applied to the
problem of PDF determination, even though in a limited
scenario~\cite{DelDebbio:2021whr,Candido:2024hjt,Medrano:2025cmg}.
Bayesian-based approaches are promising tools that ensure a rigorous framework
where prior information and assumptions are spelled out explicitly. Yet, a
global PDF determination based on these methods has not yet been attempted.

On the other hand, state-of-the-art PDF determinations rely on parametric
approaches, where a specific functional form is assumed for the PDFs at a given
initial scale $Q_0$. These functions are typically chosen to be flexible enough
to capture the main features of the PDFs, while their internal parameters are
optimised to reproduce the experimental data. Several
groups~\cite{NNPDF:2021njg,Ablat:2024hbm,Bailey:2020ooq,Alekhin:2017kpj} have
set the standard for PDF determinations through continuous refinement of their
global fits as new data and theoretical advances become available, with an
increasing emphasis on uncertainty quantification. Although these determinations
have been shown to perform incredibly well on a wide range of new experimental
data~\cite{Chiefa:2025loi}, the different methodological frameworks adopted by
the various groups lead to PDF sets whose differences are yet to be fully
understood~\cite{Harland-Lang:2024kvt,PDF4LHCWorkingGroup:2022cjn}. These
differences become significantly visible when considering parameter
determinations that are particularly sensitive to the choice of the PDF set,
thus on the central value and, most importantly, on the associated uncertainty
(see Refs.~\cite{ATLAS:2023lhg,CMS:2024ony,ATLAS:2023fsi} for some recent
examples).

In this work, we build upon the intents of
Refs.~\cite{DelDebbio:2021whr,Candido:2024hjt}, which aim at providing a
transparent statistical and sound framework for PDF determination, with all
underlying assumptions clearly stated. Here we focus on the NNPDF
methodology~\cite{NNPDF:2021njg}, which pioneered the use of machine learning
tools in the context of PDF determination and has been validated through
extensive studies over the
years~\cite{DelDebbio:2021whr,Barontini:2025lnl,Cruz-Martinez:2021rgy}. It
combines a Monte Carlo sampling of the experimental data and a feed-forward
neural network parameterisation of the PDFs. We adopt a simplified but
controlled framework to revisit the training process from first principles,
aiming at providing an explainable reformulation of its key aspects and making
transparent the assumptions that are often implicitly embedded in the fitting
procedure.

We demonstrate that the training dynamics of a neural network can be fully
reformulated in the functional space, thus providing a more interpretable
description of the learning process. We show that the training dynamics is
completely dictated by the Neural Tangent Kernel
(NTK)~\cite{DBLP:journals/corr/abs-1806-07572}, which encodes and factorises the
dependence on the architecture of the neural network and on the parameters. In
fact, the spectral properties of the NTK provide a powerful lens through which
to understand the learning process. At initialisation, the NTK is characterised
by a wide spectrum of eigenvalues, with only few large eigenvalues being
significantly different from zero. As training evolves, the NTK hierarchy is
preserved, but eigenvalues that were initially small or zero grow in magnitude.
Since the only directions that contribute to the learning process are those
associated to the non-zero eigenvalues, with the eigenvalue setting the learning
speed along this direction, this growth indicates that new features in the
functional space emerge during training and the network thus becomes capable of
learning more complex functions.

Another key result of this work is that, after an initial transient phase where
the NTK evolves significantly, the training process enters a second regime where
the NTK becomes approximately constant. This regime is often referred to as
\textit{lazy training} in the Machine Learning
literature~\cite{DBLP:journals/corr/abs-1806-07572}, and it has important
implications for the training dynamics. In this regime, we show that the
training process can be described analytically, allowing us to obtain a single
and clean closed-form expression for the output of the network at any point
during training. Interestingly, this expression decomposes into two
contributions: one that depends on the initial condition and another that
depends on the data, thus making explicit the role of prior information and of
the experimental measurements in shaping the final result. Although applicable
only when the NTK reaches stability, this analytical description serves as a
promising tool to bridge the gap between the parametric regression approach
adopted in NNPDF and the kernel regression methods that are receiving growing
attention in the community.

While derived in a simplified setting -- considering a single PDF flavour
combination with DIS data and vanilla gradient descent optimization -- we
present this study as an exploration of foundational aspects, with the
understanding that further investigations will be needed to extend these ideas
to the full complexity of modern global PDF fits. We particularly emphasise that
the present analysis is not limited to neural networks, but can be extended to
any functional parameterisation that undergoes a gradient-based training
process. It will be interesting to explore the properties of the NTK together
with its spectral structure in more realistic PDF fits, translating the differences
between various fitting methodologies in terms of the NTK. We leave these
studies to future work.

The remainder of this paper is organized as follows. In Section~\ref{sec:Init}
the inverse problem of PDF determination is briefly reviewed in the simplified
case of theoretical predictions that depend linearly on the PDFs. We then review
some fundamental statistical aspects of the Neural Networks at initialisation,
which will be relevant in the rest of the paper. The training dynamics is then
discussed in Section~\ref{sec:Training}, where the learning process of the
neural network is reformulated in functional space by means of the NTK. The
implications of the \textit{lazy training} regime are employed in
Section~\ref{sec:LazyTraining} to derive an analytical description of the
training process. Finally, we present our conclusions and outlook in
Section~\ref{sec:Conclusions}.

\section{On Neural Networks and Inverse Problems}
\label{sec:Init}

In this section, we discuss some non-trivial properties of neural networks at
initialization. Then, we present the framework adopted in this work and
introduce the notation used in the rest of the paper.

\subsection{The Inverse Problem for $xT_3$}
\label{subsec:inverse_problem}

The extraction of PDFs from experimental data is a classic example of an inverse
problem, namely the reconstruction of a function $f(x)$ from a finite set of
data points $Y_I$, where the index $I=1, \ldots, \ndat$.\footnote{When omitting
the data index $I$, we will always assume $Y \in \mathbb{R}^{\ndat}$.} In
particular, for this study, we will focus on DIS data, which depend linearly on
the function $f(x)$. The theoretical prediction for the data point $Y_I$ is
given by
\begin{equation}
    \label{eq:TheoryPred}
    T_I[f] = \sum_{i=1}^{\nflav} \int dx\, C_{Ii}(x) f_{i}(x)\, ,
\end{equation}
where $C_{Ii}(x)$ is a coefficient function, known to some given order in
perturbation theory, $i$ labels the parton flavour, and $f_i(x)$ is the PDF (or
set of PDFs) that we want to determine.

In this work, we will 

Attempting to determine a function $f$ in an infinite dimensional space of
solutions with a finite set of data is inherently ill-posed. The solution will
inevitably depend on assumptions and prior knowledge, conscious or not,
introduced to regularise the problem. The solution of the inverse problem
can be tackled either by using non-parametric methods, as recently proposed
in Refs.~\cite{DelDebbio:2021whr,Candido:2024hjt}, or, more commonly in the
filed of PDF determination, by introducing a parameterization for the function
$f$. Many state-of-the-art PDF determinations belong to the latter, and
they differ, among other things, in the choice of the parameterization and the
fitting methodology.

As mentioned, understanding the difference between the various approaches is
crucial for precision physics. For instance, different parametrisations for
can lead to a bias in the the space of te reconstructed solution. Furthermore,
the interplay between the parameterization and the fitting methodology determines
how the uncertainties on the data propagate to the uncertainties on the fitted
solution. Understanding the bias and variance of the fitted PDF is therefore a
major challenge for precision physics.

Following the ideas highlighted in
Refs.~\cite{DelDebbio:2021whr,Candido:2024hjt}, the solution of the inverse
problem is conveniently phrased in a Bayesian framework. The functions $f_i$ are
promoted to stochastic processes; for any grid of points $x_{\alpha}$,
$\alpha=1, \ldots, \ngrid$, the vector $f_{i\alpha}=f_{i}(x_{\alpha})$ is a
vector of $\nflav\times\ngrid$ stochastic variables, for which we introduce a
prior distribution $p(f)$.~\footnote{Following the same convention used for the
data, when omitting the grid index $\alpha$, and/or the flavor index $i$, we
will always refer to a vector $f \in \mathbb{R}^{\nflav\times\ngrid}$.}

Any fitting procedure is interpreted as a recipe that yields the posterior
distribution $\tilde{p}(f)$.

In this study, probability distributions are represented by ensembles of i.i.d.
replicas. So, for instance, the prior distribution $p(f)$ is described by an
ensemble
\begin{equation}
    \label{eq:RepDef}
    \left\{f^{(k)} \in \mathbb{R}^{\nflav\times\ngrid}; k=1, \ldots, \nreps\right\}\, ,
\end{equation}
drawn from the distribution $p$, so that
\begin{equation}
    \label{eq:ReplicaEnsemble}
    \mathbb{E}_{p}[O(f)] = \frac{1}{\nreps} \sum_{k=1}^{\nreps} O(f^{(k)})\, ,
\end{equation}
for any observable $O$ that is built from the PDFs.

The prior distribution $p(f)$ is defined by initializing a set of replicas using
a Glorot normal initializer~\cite{glorot2010understanding}. The result of this
initialization is discussed below in Sec.~\ref{sec:Init}. For each replica, a
new set of data $Y^{(k)}$ is generated from an $\ndat$ dimensional Gaussian
distribution centred at the experimental central value $Y$, with the covariance
given by the experimental covariance matrix $C_Y$,
\begin{equation}
    \label{eq:ExpReplicaDistr}
    Y^{(k)} \sim \mathcal{N}\left(Y, C_Y\right)\, .
\end{equation}
Each replica $f^{(k)}$ is trained on its corresponding data set $Y^{(k)}$. We
denote the replicas at training time $t$, $f^{(k)}_{t} \in
\mathbb{R}^{\nflav\times\ngrid}$. Stopping the training at time $T$, the
posterior probability distribution is represented by the set of replicas
$\left\{f^{(k)}_{T}\in \mathbb{R}^{\nflav\times\ngrid}; k=1, \ldots,
\nreps\right\}$, so that averages over the posterior distribution are computed
as
\begin{equation}
    \label{eq:PostEnsemble}
    \mathbb{E}_{\tilde{p}}[O(f)] = \frac{1}{\nreps} \sum_{k=1}^{\nreps}
        O\left(f^{(k)}_{T}\right)\, .
\end{equation}
All knowledge about the solution of the inverse problem, $f$, is encoded in the
posterior $\tilde{p}$ and is expressed as expectation values of observables $O$
using Eq.~\eqref{eq:PostEnsemble}.

\subsection{Neural Networks at Initialisation}
% \label{subsec:NNinit}

When initializing a neural network, the weights and biases -- which we denote collectively as the {\em parameters}\ 
of the network -- are drawn from some probability distribution. In 
the NNPDF formalism, the set of network parameters at initialization for each replica is an instance 
of i.i.d. stochastic variables. The probability distribution
of the network parameters induces a probability distribution for the output of the neural networks at inizialization. 
It is well known that the probability distribution of these outputs becomes approximately gaussian when the size of
the hidden layers is increased. We call this limit the {\em large-network} limit. 
In this section, we review some basic results for the large-network limit at inizialization, and 
compare these theoretical 
predictions with the actual results obtained in numerical experiments performed with the typical networks 
used by the NNPDF collaboration. 
This preliminary study of the properties of networks at initialization also allows us to introduce the 
notation used for the networks in the rest of the paper. 

As detailed in Ref.~\cite{NNPDF:2021njg}, the NNs used for the NNPDF fit have a 2-25-20-8 architecture,
a $\tanh$ activation function,
and are initialized using a Glorot normal distribution~\cite{glorot2010understanding}. The preactivation
function of a neuron is denoted as $\phi^{(\ell)}_{i,\alpha} = \phi^{(\ell)}_i(x_\alpha)$, where $\ell$
denotes the layer of the neuron, $i$ identifies the neuron within the layer\footnote{We refer
to $i$ as the {\em neuron}\ index.}, and $x_{\alpha}$ is a point in the interval $[0,1]$.
A grid of $\ngrid=50$ points is used to compute observables in the NNPDF formalism and in this work 
we focus on the vakue of $f$ at those values of $x_\alpha$. For completeness, we list the values of $x_\alpha$ in
Tab.~\ref{tab:Xvals}.

\begin{table}[ht]
    \centering
    \begin{tabular}{|c|c|c|c|c|c|c|c|c|c|}
    \hline
    $\alpha$ & $x_\alpha$ & $\alpha$ & $x_\alpha$ & $\alpha$ & $x_\alpha$ & $\alpha$ & $x_\alpha$ & $\alpha$ & $x_\alpha$ \\
    \hline
    $1$  & $2.00 \times 10^{-7}$ & $11$ & $1.29 \times 10^{-5}$ & $21$ & $8.31 \times 10^{-4}$ & $31$ & $0.0434$ & $41$ & $0.422$ \\
    $2$  & $3.03 \times 10^{-7}$ & $12$ & $1.96 \times 10^{-5}$ & $22$ & $1.26 \times 10^{-3}$ & $32$ & $0.0605$ & $42$ & $0.480$ \\
    $3$  & $4.60 \times 10^{-7}$ & $13$ & $2.97 \times 10^{-5}$ & $23$ & $1.90 \times 10^{-3}$ & $33$ & $0.0823$ & $43$ & $0.540$ \\
    $4$  & $6.98 \times 10^{-7}$ & $14$ & $4.51 \times 10^{-5}$ & $24$ & $2.87 \times 10^{-3}$ & $34$ & $0.109$ & $44$ & $0.601$ \\
    $5$  & $1.06 \times 10^{-6}$ & $15$ & $6.84 \times 10^{-5}$ & $25$ & $4.33 \times 10^{-3}$ & $35$ & $0.141$ & $45$ & $0.665$ \\
    $6$  & $1.61 \times 10^{-6}$ & $16$ & $1.04 \times 10^{-4}$ & $26$ & $6.50 \times 10^{-3}$ & $36$ & $0.178$ & $46$ & $0.730$ \\
    $7$  & $2.44 \times 10^{-6}$ & $17$ & $1.57 \times 10^{-4}$ & $27$ & $9.70 \times 10^{-3}$ & $37$ & $0.220$ & $47$ & $0.796$ \\
    $8$  & $3.70 \times 10^{-6}$ & $18$ & $2.39 \times 10^{-4}$ & $28$ & $0.0144$ & $38$ & $0.265$ & $48$ & $0.863$ \\
    $9$  & $5.61 \times 10^{-6}$ & $19$ & $3.62 \times 10^{-4}$ & $29$ & $0.0211$ & $39$ & $0.314$ & $49$ & $0.931$ \\
    $10$ & $8.52 \times 10^{-6}$ & $20$ & $5.49 \times 10^{-4}$ & $30$ & $0.0305$ & $40$ & $0.367$ & $50$ & $1.00$ \\
    \hline
\end{tabular}

    \caption{Values of $x_\alpha$ used in the NNPDF grids for the computation of
    observables. The points are equally spaced on a logarithmic scale
    for $\alpha = 1, \ldots, XXX$, and linearly spacing for $\alpha > XXX$.
    \ac{Maybe we need to rethink the layout of this table...}
    \label{tab:Xvals}}
\end{table}

The output of the neuron identified by the pair $(\ell,i)$ is
$\rho^{(\ell)}_{i\alpha} = \tanh\left(\phi^{(\ell)}_{i\alpha}\right)$.
The parameters of the NN are the weights $w^{(\ell)_{ij}}$ and the biases $b^{(\ell)}_i$, which are
collectively denoted as $\theta_\mu$, where $\mu = 1, \ldots, P$ and the total number of parameters
is
\begin{equation}
    \label{eq:TotPar}
    P = \sum_{\ell=1}^{L} \left(n_{\ell} n_{\ell-1} + n_\ell\right)\, .
\end{equation}
The preactivation function in layer $(\ell+1)$ is a weighted average of the outputs of the neurons on 
the previous layer, namely
\begin{align}
    \label{eq:RecursionNN}
    \phi^{(\ell+1)}_{i\alpha} = \sum_{j=1}^{n_\ell} w^{(\ell+1)}_{ij} \rho^{(\ell)}_{i\alpha} + b^{(\ell+1)}_{i}\, .
\end{align}
The PDFs in the
so-called evolution basis are parametrized by the preactivation functions of the output layer $L$,
$x_\alpha f_i(x_\alpha)=A_i \phi^{(L)}_{i,\alpha}$, where $i=1, \ldots, 8$ labels the flavors.
\footnote{For simplicity, we ignore the preprocessing function $x^{-\alpha_i} (1-x)^{\beta_i}$ that
is currently used in the NNPDF fits. While the preprocessing may be useful in speeding the training
it does not affect the current discussion.}
The input layer is identified by $\ell=0$ and the activation
function for that specific layer is the identity, so that
\begin{equation}
    \label{eq:InitLayerPhi}
    \rho^{(0)}_{i,\alpha} = \phi^{(0)}_{i,\alpha} = x_{i,\alpha} =
    \begin{cases}
        x_\alpha\, , \quad &\text{for}\ i=1\, ;\\
        \log\left(x_\alpha\right)\, , \quad &\text{for}\ i=2\, .
    \end{cases}
\end{equation}
In the following we refer to the preactivation functions as {\em fields}.

The Glorot normal initialiser draws each weight and bias of the NN from independent Gaussian
distributions, denoted $p_w$ and $p_b$ respectively, centred at zero and with variances
rescaled by the number of nodes in adjacent layers,
\begin{equation}
    \label{eq:RescaledGlorotVariances}
    \frac{C^{(\ell)}_{w}}{\sqrt{n_{\ell-1} + n_{\ell}}}\, ,
    \quad \frac{C^{(\ell)}_{b}}{\sqrt{n_{\ell-1} + n_{\ell}}}\, .
\end{equation}
Following the NNPDF prescription, we have $C^{(\ell)_{w}=C^(\ell)}_{b}=1$. 
The probability distribution of the NN parameters induces a probability distribution for the
preactivations,
\begin{align}
    \label{eq:PreactAtInit}
    p\left(\phi^{(\ell)}\right)
      &= \int \mathcal{D}w\, p_w(w)\,
        \mathcal{D}b\, p_b(b)\, \prod_{i,\alpha}
        \delta\left(
          \phi^{(\ell)}_{i\alpha} - \sum_{j} w^{(\ell)}_{ij}
          \rho\left(\phi^{(\ell-1)}_{j\alpha}\right)
          - b^{(\ell)}_i
          \right)\, .
\end{align}
Note that, here and in what follows, $p(\phi^{(\ell)})$ denotes the joint probability for all the
$n_{\ell}\times\ngrid$ components of $\phi^{(\ell)}$,
\begin{align}
    \label{eq:ExplIndices}
    p\left(\phi^{(\ell)}\right) = p\left(\phi^{(\ell)}_{1,\alpha_1}, \phi^{(\ell)}_{2,\alpha_1}, \ldots,
        \phi^{(\ell)}_{n_\ell,\alpha_1}, \phi^{(\ell)}_{1,\alpha_2}, \ldots, \phi^{(\ell)}_{n_\ell,\alpha_2},
        \ldots,
        \phi^{(\ell)}_{n_\ell,\ngrid}\right)\, .
\end{align}
This duality between parameter-space and function-space provides a powerful framework to study
the behaviour of an ensemble of NNs, and in particular the symmetry properties of the distribution
$p(\phi^{(\ell)})$, see \eg~\cite{Maiti:2021fpy}. Working in parameter space, \ie\ computing the
expectation values of correlators of fields as integrals over the NN parameter, one can readily
show that
\begin{align}
    \label{eq:NeurRotInv}
    \mathbb{E}\left[
        R_{i_1j_1} \phi^{(n_\ell)}_{j_1 \alpha_1} \ldots
        R_{i_nj_n} \phi^{(n_\ell)}_{j_n \alpha_n}
    \right] =
    \mathbb{E}\left[
        \phi^{(n_\ell)}_{i_1 \alpha_1} \ldots
        \phi^{(n_\ell)}_{i_n \alpha_n}
    \right]\, ,
\end{align}
where $R$ is an orthogonal matrix in $\text{SO}(n_{\ell})$. Eq.\eqref{eq:NeurRotInv} implies
that the probability distribution in Eq.~\eqref{eq:PreactAtInit} is also invariant under rotations,
and therefore it can only be a function of $\text{SO}(n_{\ell})$ invariants. Therefore
\begin{align}
    \label{eq:PriorAction}
    p\left(\phi^{(n_\ell)}\right) =
        \frac{1}{Z^{(\ell)}} \exp\left(-S\left[\phi^{(\ell)}_{\alpha_1}
            \cdot \phi^{(\ell)}_{\alpha_2}\right]\right)\, ,
\end{align}
where
\begin{align}
    \label{eq:PhiInvariant}
    \phi^{(\ell)}_{\alpha_1}
            \cdot \phi^{(\ell)}_{\alpha_2} =
    \sum_{i=1}^{n_\ell} \phi^{(\ell)}_{i \alpha_1} \phi^{(\ell)}_{i \alpha_2}\, .
\end{align}
The action can be expanded in powers of the invariant bilinear,
\begin{align}
    \label{eq:ExpandAction}
    S\left[\phi^{(\ell)}_{\alpha_1}
            \cdot \phi^{(\ell)}_{\alpha_2}\right] =
        \frac12 \gamma^{(\ell)}_{\alpha_1\alpha_2}
            \phi^{(\ell)}_{\alpha_1} \cdot \phi^{(\ell)}_{\alpha_2} +
            \frac{1}{8 n_{\ell-1}} \gamma^{(\ell)}_{\alpha_1\alpha_2,\alpha_3\alpha_4}
            \phi^{(\ell)}_{\alpha_1} \cdot \phi^{(\ell)}_{\alpha_2} \,
            \phi^{(\ell)}_{\alpha_3} \cdot \phi^{(\ell)}_{\alpha_4} + O(1/n_{\ell-1}^2)\, ,
\end{align}
so that the probability distribution is fully determined by the couplings 
$\gamma^{(\ell)}$.\footnote{
    We have denoted {\em all}\ couplings by $\gamma^{{(\ell)}}$. Different couplings 
    are indentified by the number of indices, so that $\gamma^{(\ell)}_{\alpha_1\alpha_2}$ 
    is a two-point coupling, $\gamma^{(\ell)}_{\alpha_1\alpha_2,\alpha_3\alpha_4}$ is a four-point 
    coupling, etc. 
} 
In
Eq.~\eqref{eq:ExpandAction}, we have factored out inverse powers of $n_\ell$ for each coupling.
With this convention, and with the scaling of the parameters variances in
Eq.~\eqref{eq:RescaledGlorotVariances}, the couplings in the action are all $O(1)$
in the limit where $n_\ell\to\infty$.
As a consequence, the probability distribution at initialization is a multidimensional Gaussian at
leading order in $1/n_\ell$, with quartic corrections that are $O(1/n_\ell)$, while higher powers
of the invariant bilinear are suppressed by higher powers of the width of the layer. This power counting
defines an effective field theory, where deviations from Gaussianity can be computed in perturbation
theory to any given order in $1/n_\ell$, see \eg\ Ref.~\cite{Roberts:2021fes} for a detailed
presentation of these ideas. While the actual calculations become rapidly cumbersome, the
conceptual framework is straightforward.

At leading order, the second and fourth cumulant are respectively
\begin{align}
    &\langle \phi^{(\ell)}_{i_1,\alpha_1} \phi^{(\ell)}_{i_2,\alpha_2}\rangle
      = \delta_{i_1 i_2} K^{(\ell)}_{\alpha_1\alpha_2} + O(1/n_{\ell-1})\, , \\
    &\langle \phi^{(\ell)}_{i_1,\alpha_1} \phi^{(\ell)}_{i_2,\alpha_2}
      \phi^{(\ell)}_{i_3,\alpha_3} \phi^{(\ell)}_{i_4,\alpha_4}\rangle_c
      = O(1/n_{\ell-1})\, ,
\end{align}
where
\begin{equation}
    \label{eq:DefineKmat}
    K^{(\ell)}_{\alpha_1\alpha_2} = \left(\gamma^{(\ell)}\right)^{-1}_{\alpha_1\alpha_2}\, .
\end{equation}
The ``evolution'' of the couplings as we go deep in the NN, \ie\ the dependence of the couplings on
$\ell$, is governed by Renormalization Group (RG) equations, which preserve the power counting in
powers of $1/n_{\ell}$. At leading order,
\begin{align}
    K^{(\ell+1)}_{\alpha_1\alpha_2} &=
      \left.
      C_b^{(\ell+1)} + C_w^{(\ell+1)} \frac{1}{n_\ell}
      \langle \vec{\rho}^{\,(\ell)}_{\alpha_1} \cdot
      \vec{\rho}^{\,(\ell)}_{\alpha_2} \rangle
      \right|_{O(1)} \\
      \label{eq:RecursionForK}
      &= C_b^{(\ell+1)} + C_w^{(\ell+1)} \frac{1}{n_\ell}
      \langle \vec{\rho}^{\,(\ell)}_{\alpha_1} \cdot
      \vec{\rho}^{\,(\ell)}_{\alpha_2} \rangle_{K^{(\ell)}}\, ,
\end{align}
where
\begin{align*}
    \frac{1}{n_\ell}
      \langle \vec{\rho}^{\,(\ell)}_{\alpha_1} \cdot
      \vec{\rho}^{\,(\ell)}_{\alpha_2} \rangle_{K^{(\ell)}} =
    \int \prod_{\alpha}d\phi_\alpha\,
      \frac{e^{-\frac12 \left(K^{(\ell)}\right)^{-1}_{\beta_1\beta_2}
        \phi_{\beta_1} \phi_{\beta_2}}}
        {\left|2\pi K^{(\ell)}\right|^{1/2}}\,
        \rho(\phi_{\alpha_1}) \rho(\phi_{\alpha_2})\, .
\end{align*}
Eq.~\eqref{eq:RecursionForK} can be solved for the NNPDF architecture leading to the
covariance matrices for the output of the NNs displayed in
Figs.~\ref{Fig:KRecursionOne} and~\ref{Fig:KRecursionTwo}.
\begin{figure}[t!]
    \centering
    \includegraphics[scale=0.4]{figs/K1_correlations.pdf}
    \caption{The empirical (left) and analytical (right) covariance matrices $K^{(1)}$ of the first layer
    of the NNPDF architecture. The covariance in the left panel is computed ``bootstrapping'' over an
    ensemble of 100 replicas, initialised using the Glorot normal distribution. The covariance in the right
    panel is obtained by solving Eq.~\eqref{eq:RecursionForK} numerically.
    \label{Fig:KRecursionOne}
    }
\end{figure}

\begin{figure}[t!]
    \centering
    \includegraphics[scale=0.4]{figs/K2_correlations.pdf}
    \includegraphics[scale=0.4]{figs/K3_correlations.pdf}
    \caption{Same as Fig.~\ref{Fig:KRecursionOne}, but for the second (top) and third (bottom) layers of the
    NNPDF architecture.
    \ac{Here the four indices of the covariance $K_{i_1i_2, \alpha_1\alpha_2}$ are flattened into
    two indices for the sake of graphical representation. Maybe we should group the labels into groups of
    $N_{\rm grid}$ ticks on the axes.}
    \label{Fig:KRecursionTwo}
    }
\end{figure}

As a consequence of the symmetry of the probability distribution, the mean value of the fields at
initialization needs to vanish, while their variance at each point $x_\alpha$ is given by the
diagonal matrix elements of $K^{(\ell)}$. The central value and the variance of the
parametrized singlet ($\Sigma$) and gluon ($g$) at initialization are shown in
Fig.~\ref{fig:SingletGluonInit} for an ensemble of $\nreps=100$. The central value is computed as
discussed above in Eq.~\eqref{eq:ReplicaEnsemble},
\begin{align}
    \label{eq:MeanValAtInit}
    \bar{f}_{i\alpha} = \bar{f}_{i}(x_\alpha) = \frac{1}{\nreps} \sum_{k=1}^{\nreps} f^{(k)}_i(x_\alpha)\, ,
\end{align}
and the variance $\sigma^2_{i\alpha}$ is computed using the same formula with
\begin{align}
    \label{eq:VarAtInit}
    O(f) = \frac{\nreps}{\nreps-1} \left(f_i(x_\alpha) - \bar{f}_{i}(x_\alpha)\right)^2\, .
\end{align}

\begin{figure}
    \centering
    \includegraphics[scale=0.5]{plots/UoECentredLogo282v1160215.png}
    \caption{\ldd{Plot of replicas at init}}        
    \label{fig:SingletGluonInit} 
\end{figure}

\ldd{List of plots that we still need to do:
comparison analytical vs empirical, 
comparison analytical vs Gaussian processes}

% \paragraph{Dependence on the architecture.}
Having analytical expressions for the variance at initialization allows us to investigate the
impact of the NN architecture on the prior that is imposed on the PDFs. Iterating
Eq.~\eqref{eq:RecursionForK} yields the covariance at initialization for various depths.
Do we get something interesting? worth mentioning?

\ldd{We should also look at the recursion relations with other activations. Again, check whether we
get something interesting...}

\FloatBarrier

\section{Training Dynamics and the Neural Tangent Kernel}
\label{sec:Training}

Having defined the physical context of this work and established some properties
of the neural network at initialisation, we now turn to the optimisation
process. In the context of machine learning, specifically when dealing with
neural networks, optimisation is an iterative algorithm that updates the
parameters of the network in order to minimise a figure of merit defined
appropriately. Due to the large number of parameters that characterise a neural
network, the figure of merit (also known as \textit{error function},
\textit{loss function}, or simply \textit{loss}) is a non-convex
high-dimensional function, posing a challenge in the minimisation task. In
addition, in order to avoid \textit{over-} and \textit{under-learning}, these
training algorithms are paired with the so-called \textit{stopping criterion},
which specifies the optimal condition to end the training process.

In practice, this task is tackled by using gradient methods where the direction
towards the minimum is defined by the gradient of the loss function. These
methods are usually improved by including, for instance, stochasticity and
information on previous iterations. A detailed overview of these extended
gradient methods is beyond the scope of this work. In the context of PDF
determinations, the NNPDF collaboration makes intensive use of these tools and
the reader is encouraged to refer to Ref.~\cite{NNPDF:2021njg} for an extensive
discussion.

Our main aim in this paper is understanding the dynamics driving the training
process. Indeed, while these algorithms have achieved remarkable empirical
success, a theoretical understanding of the optimization process remains
elusive. Therefore, we work in a simplified setting where we consider the
simplest gradient method, \textit{i.e.} Gradient Descent (GD), and and data that
depend linearly on the unknwon PDFs, as shown in Eq.~\eqref{eq:TheoryPred}.
Furthermore, we do not split the dataset between training and
validation\footnote{We are aware that such a framework has limited applicability
in the context of PDF determination, and we are far from assuming this as an
optimal choice. Again, the focus remain on the methodological aspects rather
than the underlying physics.} The generalization to other minimizers and
non-linear data is left to future investigations, but is expected to yield
qualitatively similar results. Finally, we remark that the results in this
section, although obtained having in mind neural networks, apply to any generic
parametrization of the unknown function, whether it is a polynomial or a
kernel~\cite{Costantini:2025wxp}.

\subsection{Training in Functional Space}
\label{sec:GradFlow}

Gradient descent is described as a continuous flow of the parameters $\theta$ in
training time $t$ along the negative gradient of the loss function
$\mathcal{L}$. Following from the parameter-space/function-space duality
introduced in Sec.~\ref{sec:Init}, we aim at rephrasing the optimisation process
of GD in the space of the network output $f$. To ease the mathematical
tractability, we employ the continuous version of GD, which has been
shown~\cite{barrett2022igr} to match the discretised version as long as the
learning rate is small enough. The continuous Gradient Flow (GF) is then given
by
\begin{align}
    \label{eq:GradientFlowDef}
    \ddt &\theta_{t,\mu} = -\nabla_\mu \mathcal{L}_t\, ,
\end{align}
where $\theta_{t,\mu}$ and $\mathcal{L}_t$ identify respectively the parameter
and the loss function at training time $t$. We focus here on quadratic loss
functions that are obtained as the negative logarithm of Gaussian data
distributions around their theoretical predictions,
\begin{align}
    \label{eq:QuadLoss}
    \mathcal{L}_t = \frac12 \left(Y - T[f_t]\right)^T C_Y^{-1} \left(Y - T[f_t]\right)\, ,
\end{align}
where $f_t$ is the output of the network at training time $t$, which follows
from the time-dependence of the internal parameters. Here $C_Y$ is the
covariance of the data, which includes statistical and systematic errors given
by the experiments and also any theoretical error, like \eg\ missing higher
orders in the theoretical predictions. Indices that are summed over are
suppressed to improve the clarity of the equations. Note that the loss function
at training time $t$ is computed using the theoretical prediction $T[f_t]$, \ie\
the result of Eq.~\eqref{eq:TheoryPred} computed using the fields at training
time $t$. For a quadratic loss, the gradient is
\begin{align}
    \nabla_\mu \mathcal{L}_t = - \left(\nabla_\mu f_t\right)^T \left(\frac{\partial T}{\partial f}\right)_t
      C_Y^{-1} \epsilon_t\, ,
\end{align}
where, writing explicitly the data index,
\begin{align}
    \label{eq:EpsDef}
    \epsilon_{t,I} = Y_I - T_I[f_t]\, , \quad I=1, \ldots, \ndat\, .
\end{align}
For the specific case of a quadratic loss function, the gradient is proportional
to $\epsilon_t$, which is the difference between the theoretical prediction and
the data at training time $t$. If at some point during the training the
theoretical predictions reproduce all the data, the training process ends. A
further simplification is obtained in the case of data that depend linearly on
the unknown function $f$. In the specific case of NNPDF fits, the integrals in
Eq.~\eqref{eq:TheoryPred} are approximated by a Riemann sum over the grid of $x$
points,
\begin{align}
    \label{eq:FKTabDef}
    T_I[f] \approx \sum_{i=1}^{\nflav}\sum_{\alpha=1}^{\ngrid} \FKtab_{Ii\alpha} f_{i\alpha}\, ,
\end{align}
and hence
\begin{align}
    \label{eq:dTbydf}
    \left(\frac{\partial T_I}{\partial f_{i\alpha}}\right)_t =
        \FKtab_{Ii\alpha}\, ,
\end{align}
which is independent of $t$. With simple algebraic steps, the flow of parameters
$\theta$ can be translated into a flow for the fields,
\begin{align}
    \label{eq:NTKFlow}
    \ddt &f_{t,i_1\alpha_1} = (\nabla_\mu f_{t,i_1\alpha_1}) \ddt \theta_\mu =
      \Theta_{t,i_1\alpha_1i_2\alpha_2}
      \FKtabT_{i_2\alpha_2I} \left(C_Y^{-1}\right)_{IJ} \epsilon_{t,J}\, ,
\end{align}
where we have defined the Neural Tangent Kernel~\cite{jacot2018neural}
\begin{align}
    \label{eq:NTKDef}
    \Theta_{t,i_1\alpha_1i_2\alpha_2} = \sum_\mu
    \nabla_\mu f_{t,i_1\alpha_1} \nabla_\mu f_{t,i_2\alpha_2}\, .
\end{align}

In order to facilitate the discussion in Sec.~\ref{sec:Lazy},
Eq.~\eqref{eq:NTKFlow} can be rewritten in a more compact form. We first omit
the indices such that, for instance,
\begin{align}
  \left(\frac{\partial T}{\partial f}\right)_t = \FKtab\, , \quad
  \Theta_t = \left(\nabla_\mu f_t\right) \left(\nabla_\mu f_t\right)^T\, .
  \label{eq:dTdfForLinearObs}
\end{align}
Then, using the definition of the error in Eq.~\eqref{eq:EpsDef}, we can rewrite
Eq.~\eqref{eq:NTKFlow} as follows
\begin{align}
    \label{eq:FlowEquationNoIndices}
    \ddt f_t = -\Theta M f_t + b\, ,
\end{align}
where
\begin{align}
    M &= \FKtabT C_Y^{-1} \FKtab\, , \quad b = \Theta \FKtabT C_Y^{-1} Y\, .
\end{align}
Here $M$ is a positive-semidefinite matrix that depends only on the data and the
theoretical predictions, while $b$ is a vector that depends also on the data.

Before moving to the next subsection, a few comments are due. First, although
derived in the context of neural networks, these equations do not refer to a
specific parameterization. Indeed, these remain valid even when an explicit
functional form to parametrize the PDFs is chosen, as \eg in
Refs.~\cite{Bailey:2020ooq,Hou:2019efy,Costantini:2025wxp}. Second, it is
interesting to observe that the flow equation,
Eq.~\eqref{eq:FlowEquationNoIndices}, depends on two matrices, $\Theta$ and $M$.
The former encodes the model dependence, while the latter brings physical
information. The interplay between these two matrices is crucial for
understanding the training dynamics, as it will be discussed in
Sec.~\ref{sec:NTKAlign}. Finally, the NTK derived in Eq.~\ref{eq:NTKDef} is
inherently time-dependent in a complex way, which precludes any attempt in
integrating Eq.~\ref{eq:FlowEquationNoIndices} analytically. We will come back
to this point in Sec.~\ref{sec:Lazy}, and we now turn to discussing the
properties of the NTK during training.

\subsection{Inside the Training Dynamics: an NTK perspective}

The NTK introduced above provides a powerful framework for understanding neural
network dynamics during training. Originally developed by Jacot et
al.~\cite{jacot2018neural} to analyse infinite-width feed-forward networks, the
NTK theory has since been extended to diverse architectures including
convolutional networks~\cite{arora2019exact} and recurrent
networks~\cite{alemohammad2021recurrent}. This theoretical framework has proven
invaluable for characterizing learning dynamics and generalization properties
across various network designs.

From Eqs.~\eqref{eq:NTKDef} and~\eqref{eq:FlowEquationNoIndices}, we observe
that the NTK encodes the dependence on the architecture of the network and
governs its training dynamics. The analysis of the NTK properties is thus
crucial for understanding the behaviour of the network during training. We first
discuss the properties of the NTK at initialisation, before moving to the
training phase, where we provide a detailed study of the NTK in the context of
the NNPDF methodology.

% ===================================
\begin{figure}[t!]
  \centering
  \includegraphics[width=0.90\textwidth]{figs/section_3/ntk_initialization_with_uncertainty.pdf}
  \caption{Frobenius norm of the NTK at initialisation, $\lVert \Theta_0
  \rVert$, in function of the width of the network. On the left, the central
  values and uncertainty bands are obtained as the mean and one-sigma deviation
  of the ensemble of networks. The plot on the right shows the relative
  uncertainty.}
  \label{fig:NTKInit}
\end{figure}
% ===================================

\subsubsection{NTK at Initialization}
\label{sec:NTKAtInit}

Before training, the NTK is blind to data and depends, in addition to the
architecture, on the $x$-grid of input and on the architecture, as it can be
seen from Eq.~\eqref{eq:NTKDef}. The NTK is a function of the fields $f$, which
are stochastic variables described by their joint probability distribution as
discussed in Sect.~\ref{sec:Init}. Therefore the NTK is also a stochastic
variable, with its own probability distribution, which we represent as usual as
a set of replicas. 

It is argued in the literature that, in the large-width limit, the variance of
the NTK over the set of replicas tends to zero with the width of the hidden
layers (see, \textit{e.g.}, \cite{Roberts:2021fes}). In order to quantify the
variation of the NTK, we start by computing the Frobenius norm of the NTK over
an ensemble of networks for different architectures. For each architecture, we
consider the mean value and standard deviation of the norm as statistical
estimators of the variations of the NTK. The result is displayed in
Fig.~\ref{fig:NTKInit}. Even though the Frobenius norm is a coaarse indicator of
the variations of the NTK, the figure shows clearly that the variance of the
norm becomes smaller with the size of the network, which is consistent with the
theoretical expectation that the NTK should not fluctuate for infinite-width
networks\footnote{Note that, in addition to the scaling $\mathcal{O}(1/n)$
theoretically predicted for large networks, the uncertainty bands include
bootstrap errors due to the finite size of the ensemble. This amounts to $\sim
10\%$ of the total error quoted in the plots, as explicitly checked using
bootstrap over the ensemble.}.

% ===================================
\begin{figure}[t]
  \centering
  \includegraphics[width=0.45\textwidth]{figs/section_3/ntk_initialization_arch.pdf}
  \caption{Spectrum of the NTK at initialization for the architectures shown in
  Fig.~\ref{fig:NTKInit}. Error bands correspond to one-sigma uncertainties over
  the ensemble of networks.}
  \label{fig:NTKSpectrum}
\end{figure}
% ===================================

In order to get a more quantitative description of the NTK at initialization,
its spectrum is shown in Fig.~\ref{fig:NTKSpectrum} for four different
architectures. As debated in the literature~\cite{XXX}, the spectrum of the NTK
is heavily hierarchical, and only few eigenvalues are actually
non-zero\footnote{Note that, due to the large difference in magnitude of the
eigenvalues, the finite precision used in our codes introduces noise in the
decomposition, so that small eigenvalues should be effectively considered zero.
We discuss the cut-off tolerance later, when we discuss the training process in
more details.}. This means that only a small subset of active directions can
inform the network during training, as it will be discussed later. Note that, at
least at initialization, these observations do not depend on the architecture.
The eigenvalues in Fig.~\ref{fig:NTKSpectrum} are mostly independent of the size
of the network. There is a downward fluctuation of the third eigenvalue for the
largest architecture that we considered, but we do not have any evidence that
this drop is a physical feature of the system, rather than a fluctuation. the
variance of the set of eigenvalues over replicas decreases with increasing size,
as expected. 

% \FloatBarrier

\subsubsection{NTK During Training}
\label{sec:NTKDuringTraining}

% In the machine learning literature, it is argued that the NTK remains constant
% during training provided that the width of the network is large enough. Here
% we show that this is not the case, at least for the architectures used in the
% standard NNPDF methodology. 

Having established the properties of the NTK at initialisation, we now discuss
its behaviour during training. TO do so, we performed a fit of $T_3$ using the
NNPDF methodology with the dataset described in App.~\ref{app:dataset}. We
initialized an ensemble of $\nreps = 100$ replicas with identical architecture,
training each replica independently using GD optimization. As our focus here is
on NTK properties rather than physical predictions, we use generate three sets
of data with controlled noise characteristics -- L0, L1, and L2 -- following the
prescription described in App.~\ref{app:dataset}. Throughout the training
process, we track the evolution of the NTK to understand how the network's
effective dynamics change as it learns the target function.

\paragraph{Onset of Lazy Training} 

As a first estimator of the variation of the NTK, we show in the left panel of
Fig.~\ref{fig:NTKTime} the Frobenius norm of the variation during training,
normalized by the Frobenius norm of the NTK itself, 
\begin{equation}
\delta \Theta_t = \frac{\lVert \Theta_{t+1} - \Theta_t \rVert}{\lVert \Theta_t \rVert} \;,
\label{eq:DeltaNTK}
\end{equation}
for three different datasets, L0, L1, and L2. 

% ===================================
\begin{figure}[t]
  \centering
  \includegraphics[width=0.45\textwidth]{section_3/delta_ntk.pdf}
  \includegraphics[width=0.45\textwidth]{section_3/ntk_eigvals_single_plot_L0.pdf} 
  \caption{(Left) Relative variation of the NTK during training for L0, L1, and
  L2 data. Error bands correspond to one-sigma uncertainties over the ensemble
  of networks. (Right) Evolution during training of the first five eigenvalues
  of the NTK using L0 data.}
  \label{fig:NTKTime}
\end{figure}
% ===================================

It is clear from our data that the NTK does not remain constant during training.
Two different phases can be distinguished in the figure. The first one covers
the initial part of the training. From the left panel of Fig.~\ref{fig:NTKTime},
we see that the norm of the NTK varies significantly in the early stages of the
evolution, in strong contrast with the predictions obtained in the
infinite-width limit. Note also that this initial peak is more pronounced for L2
data. This is consistent with the fact that the NTK (\ie\ the architecture)
needs to accommodate the noise in the data, thus leading to a larger variation
of the NTK. On the other hand, after this initial phase -- corresponding
approximately to the first 20,000 epochs in our experiment -- the NTK tends to
stabilize. As discussed in the previous Section, we refer to this second phase
as the \textit{lazy training}, in keeping with the terminology adopted in the
literature. We conclude that, in this phase, the NTK does not change
significantly. As a consequence, this suggests that a description of the
training using a constant NTK, as predicted by the theory of the infinite-width
networks, can only be applied after the initial phase, \ie\ after the NTK has
stabilized. 

\FloatBarrier

\paragraph{Eigenvalues During Training}

Further insight on the evolution of the NTK can be obtained by studying its
eigensystem as a function of the training time. In the right panel of
Fig.~\ref{fig:NTKTime} we report the variation of the first five eigenvalues of
the NTK, using the standard NNPDF architecture and L0 data. We see that the
hierachical structure observed at initialization is preserved, but the size of
the subdominant eigenvalues increases significantly in the early stages of
training -- by one or two orders of magnitude depending on the specific
eigenvalues. 

% ===================================
\begin{figure}[t]
  \centering
  \includegraphics[width=0.30\textwidth]{figs/section_3/ntk_eigvals_L0_L1_L2_n_1.pdf}
  \includegraphics[width=0.30\textwidth]{figs/section_3/ntk_eigvals_L0_L1_L2_n_2.pdf}
  \includegraphics[width=0.30\textwidth]{figs/section_3/ntk_eigvals_L0_L1_L2_n_3.pdf}
  \includegraphics[width=0.30\textwidth]{figs/section_3/ntk_eigvals_L0_L1_L2_n_4.pdf}
  \includegraphics[width=0.30\textwidth]{figs/section_3/ntk_eigvals_L0_L1_L2_n_5.pdf}
  \caption{The first five eigenvalues of the NTK for L0, L1, and L2 data. Error
  bands correspond to one-sigma uncertainties over the ensemble of networks.}
  \label{fig:EigvalsComparison}
\end{figure}
% ===================================

In Fig.~\ref{fig:EigvalsComparison}, the same first five eigenvalues of the NTK
are displayed for L0, L1, and L2 data. We can make a few observations upon
inspecting these plots. First, we notice that the way in which data is generated
has an impact on the eigenvalues of the NTK. In general, the uncertainty bands
for L2 data are larger than those for L1 and L0 data, indicating that the NTK is
more sensitive to the noise in the data. This is consistent with the observation
made in Fig.~\ref{fig:NTKTime}. The eigenvalues reach a plateau and do not
change significantly once the network enters the lazy training regime. The
increase of the subdominant eigenvalues, combined with the analysis of
Eqs.~\eqref{eq:FlowParallel} and~\eqref{eq:FlowPerp}, suggests that more
``physical'' features become learnable before lazy training sets in.

% ===================================
\begin{figure}[t]
  \centering
  \includegraphics[width=0.45\textwidth]{section_3/loss_and_eigvals_vs_epochs.pdf}  
  \caption{Variation of the loss function overlaid with the first five
  eigenvalues for a selected replica over the ensemble. Left scale refers to the
  loss, while the right scale refers to the eigenvalues.}
  \label{fig:Loss}
\end{figure}
% ===================================

Finally, in Fig.~\ref{fig:Loss} we show the variation of the loss function
during training, overlaid with the first five eigenvalues of the NTK, for a
selected replica over the ensemble. It is interesting to see that in
correspondence with the sudden variation of the subdominant eigenvalues, the
loss function drops significantly, at the cost of an instability localised in
the descent. We interpret this as the network learning new features, changing
its internal representation to accommodate the new information. After this
initial phase, the eigenvalues stabilize and the loss function decreases
smoothly, as expected in the lazy training regime.

As it will be extensively discussed later in Sec.~\ref{sec:LazyTraining}, the
eigenvalues and eigenvectors of the NTK play a special role. Indeed, the output
$f$ can be decomposed into the basis of eigenvectors of the NTK. Hence the
eigenvectors corresponding to the larger eigenvalues can be interpreted as {\em
learnable}\ features, while the small (or zero) eigenvalues, correspond to
directions in which the field $f$ never evolves during training.

% \FloatBarrier

\subsubsection{Eigenvectors and Alignment of the NTK}
\label{sec:NTKAlign}

It has been argued before that there is a non-trivial interplay between the
eigenspace of the NTK and that of the matrix $M$. Indeed, the former encodes the
model dependence, while the latter brings physical information. Of course the
two matrices are independent at initialisation, and we do not expect any
alignment patter between the two. However, this picture does change during
training, as the NTK evolves and the model learns the target function. To
quantify this alignment, we define the matrix $A$, 
\begin{equation}
  \label{eq:MatrixA}
  A_{kk'} = \left( \left< z^{(k)}, v^{(k')}\right> \right)^2 = \cos^2(\theta_{kk'}) \;,
\end{equation}
where $z^{(k)}$ and $v^{(k')}$ are the $k$-th and $k'$-th eigenvectors of the
NTK and $M$, respectively. The matrix $A$ is thus a measure of the alignment
between the eigenspaces of the two matrices. The rows of the matrix correspond
to the eigenvectors of the NTK, ordered by the value of the corresponding
eigenvalues, with the eigenvectors corresponding to the larger eigenvalues at
the top of the matrix. The columns correspond to eigenvectors of the matrix $M$,
also ordered by the values of the corresponding eigenvalues, with the largest
eigenvalues to the left in this case. In Fig.~\ref{fig:NtkMAlign}, we show the
matrix $A$ at different epochs of the training for L2 data and a single NTK
replica. 
% ===================================
\begin{figure}[ht!]
  \centering
  \includegraphics[width=1\textwidth]{section_3/ntk_alignment_L2.pdf}
  \caption{Matrix $A$ as defined in Eq.~\eqref{eq:MatrixA} for L2 data and for a
  single replica of the NTK. The matrix is shown at different epochs of the
  training process, indicated in the top of each panel. The white dashed line
  indicates the cut-off tolerance that we imposed to the eigenvalues of the NTK
  (see Appendix...?).}
  \label{fig:NtkMAlign}
\end{figure}
% ===================================

% ===================================
\begin{figure}[ht!]
  \centering
  \includegraphics[width=0.45\textwidth]{section_3/ntk_alignment_fin_L2_1}
  \includegraphics[width=0.45\textwidth]{section_3/ntk_alignment_fin_L2_2}
  \caption{Alignment of the eigenvectors of the NTK with the input function
  $f^{\rm(in)}$ used to generate the L2 data, measured in terms of $\cos
  \theta^{(k)} = (z^{(k)}, f^{\rm{(in)}})/ \Vert f^{\rm{(in)}}\Vert$. In the
  left panel, the first five eigenvectors of the NTK are shown, while the right
  panel shows the remaining eigenvectors up to $k=7$. \ac{Is this plot telling
  us something?}}
  \label{fig:NTKAlignFin}
\end{figure}
% ===================================
% ===================================
\begin{figure}[ht!]
  \centering
  \includegraphics[width=0.90\textwidth]{section_3/q_directions_sign.pdf}
  \caption{First five eigenvectors of the combined matrix $H=\Theta M$, as in
  Eq.~\eqref{eq:FlowEquationNoIndices}, at different training time and as
  function of the input $x$-grid. We also show the output of the network at the
  same training time, which is displayed in gray.}
  \label{fig:NTKMEigVecs}
\end{figure}
% ===================================

The blue rectangle in the top right corner of the matrix shows that the
eigenvectors of the NTK corresponding to the largest eigenvalues are orthogonal
to the eigenvectors of $M$ that are in the kernel of $M$, \ie\ the directions
that do not contribute to the observables. It is useful to remember that the
largest eigenvalues of the NTK correspond to the directions that are orthogonal
to $\ker\Theta$, \ie\ the directions that are learnable during the training
process. In order to have a robust training process, we expect these learnable
directions to align with the directions that actually contribute to the loss
functions, \ie\ the ones corresponding to the largest eigenvalues of $M$.
Consistently with this intuition, we see that the size of this blue rectangle
increases with training time. In particular, it is clear from our plot that it
becomes deeper by the onset of the lazy training regime: more of the learnable
directions -- the {\it features}\ that the network can learn -- are aligned with
the directions that contribute most to the observables.

A similar analysis can also be performed by studying the alignment of the
eigenvectors of the NTK with the input function used to generate the data. In
Fig.~\ref{fig:NTKAlignFin}, we show the cosine of the angle between the two
vectors for two subsects of eigenvectors, in the right and left panel
respectively. In these two plots, different pattern can be observed. First, the
first four angles vary considerably during training. These observations lead us
to conclude that other than changing the dominant eigenvectors, the NTK is also
activating some others that were asleep in the first stage of training. This in
accordance to what has been previously observed in
Figs.~\ref{fig:EigvalsComparison}-\ref{fig:NtkMAlign}. 

A complementary picture is displayed in Fig.~\ref{fig:NTKMEigVecs}. Here, we
show the eigenvectors of the matrix $H = \Theta M$, labelled with $q^{(i)}$, at
different training times and as functions of the $x$-grid. Together with the
eigenvectors, we also show the output of the trained neural network at the
corresponding training time. From these plots, we see that as the training
progress, the shape of the eigenvectors become more structured in order to
reproduce the output function. Again, this conclusion supports the observations
made previously in various occasions, that during training the neural network is
changing its internal representation and the NTK encodes this information.

\FloatBarrier

\section{Lazy Training in NNPDF (NOT FINAL)}
\label{sec:LazyTraining}

\begin{center}
  \red{Comments}\\
  \textcolor{blue}{Might be worth citing these two works for the NTH
  \cite{huang2019ddnn}\cite{Dyer:2019uzd}?}
\end{center}

\noindent In the previous section we presented an empirical study of the
training dynamics through the lens of the NTK. We observed that the NTK is able
to capture the main features of the training process, and that its time
evolution is characterised by a rapid initial transient, followed by a slower
evolution during the rest of the training. We now turn our attention on this
last stage of the training, where the NTK has stabilised and becomes
approximately constant. In doing so, we will build upon the results presented in
Refs.~\cite{jacot2018neural,lee2019wide} and extend them to the case of NNPDF.
In the following, we derive the analytical solution of the flow equation, which
allows us to write an explicit expression for the trained field as a function of
the field at initialisation and the data. We will discuss the phenomenological
implications of these results in Sec.~\ref{sec:TrainClosure}.

\subsection{Solution to the Flow Equation}
\label{sec:Lazy}

The lazy training regime is characterised by a slow-evolving NTK. We denote as
$T_{\rm ref}$ the time at which the onset of this regime occurs. The NTK is then
\textit{frozen} to its value at $T_{\rm ref}$, and from this time onward the NTK
is taken to be constant
\begin{equation}
  \Theta_t = \Theta_{T_{\rm ref}} \equiv \Theta, \quad \textrm{for } t \geq T_{\rm ref}.
\end{equation} 
The flow equation can then be written as
\begin{align}
  \ddt f_t = -\Theta M f_t + b\, ,
  \label{eq:FlowEqTwo}
\end{align}
where $M$ and $b$ are defined as in Eq.~\eqref{eq:MandBDef}. Note that now where
now neither $\Theta$ nor $b$ depend on the training time $t$, as a consequence
of the NTK. In order to solve this first-order linear differential equation, we
observe that the eigenvectors of $\Theta$,
\begin{align}
    \label{eq:ThetaEigensystem}
    \Theta z^{(k)} = \lambda^{(k)} z^{(k)}\, ,
\end{align}
provide a basis for expanding Eq.~\eqref{eq:FlowEqTwo}. It is necessary at this
stage to distinguish the components of $f_t$ that are in the kernel of $\Theta$
from the ones that are in the orthogonal complement, hence we introduce the
notation
\begin{align}
    \label{eq:ParallelCompnents}
    &f^\parallel_{t,k} = \left(z^{(k)}, f_t\right)\, , \quad \text{if}\ \lambda^{(k)} = 0\, , \\
    \label{eq:OrthogonalComponents}
    &f^\perp_{t,k} = \frac{1}{\sqrt{\lambda^{(k)}}} \left(z^{(k)}, f_t\right)\, , \quad
        \text{if}\ \lambda^{(k)} \neq 0\, ,
\end{align}
where the scalar product has been defined as
\begin{equation}
  \left(f'_{t'}, f_t\right) = \sum_{i,\alpha} f'_{t',i\alpha} f_{t,i\alpha}\,.
\end{equation}
One can readily see that the components in the kernel of $\Theta$, $\text{ker}\
\Theta$, do not evolve during the flow,\footnote{ Despite this result has been
obtained using the frozen NTK, it is worth mentioning that at any time during
training the kernel of the NTK is always defined and non-empty. Hence, also in
the initial stage, there will be a component that is completely determined by
the initial condition, \ie by the prior distribution in functional space.
\ac{This footnote might be expanded in the pheno section of NTK...}}
\begin{align}
    \label{eq:FlowParallel}
    \ddt f^\parallel_{t,k} = 0
        \quad \Longrightarrow \quad f^\parallel_{t,k} = f^\parallel_{0,k}\, .
\end{align}
This means that the final solution will be affected by an irreducible noise that
is purely dictated by the initial condition. The flow equation for the
orthogonal components can be written as
\begin{align}
    \label{eq:FlowPerp}
    \ddt f^\perp_{t,k} = - H^\perp_{kk'} f^\perp_{t,k'}
        + B^\perp_{k}\, ,
\end{align}
where  we introduced
\begin{align}
    H^\perp_{kk'} &= \sqrt{\lambda^{(k)}} \left(z^{(k)}, M z^{(k')}\right) \sqrt{\lambda^{(k')}}\, ,\\
    B^\perp_k &= -\sqrt{\lambda^{(k)}} \left[\left(z^{(k)}, M z^{(k')}\right) f^\parallel_{0,k'}
        - \left(z^{(k)}, \FKtabT C_Y^{-1} Y\right)\right]\, .
\end{align}
Here the indices on quantities that have a $\perp$ suffix only span the space
orthogonal to the kernel of $\Theta$, while the indices on quantities that have
a $\parallel$ suffix span the kernel. We refer to $H^\perp$ as the flow (or
training) Hamiltonian; we see explicitly in the definition above that the flow
dynamics is determined by a combination of the architecture of the NN, encoded
in the NTK, and the data, on which $M$ depends. More specifically, the matrix
elements of $M$ can be written as
\begin{align}
    \label{eq:MMatElems}
    \left(z^{(k)}, M z^{(k')}\right) = T^{(k)T} C_Y^{-1} T^{(k')}\, ,
\end{align}
where $T^{(k)} = T[z^{(k)}]$ is the vector of theory predictions for the data
obtained using $z^{(k)}$ as the input PDF. Similarly, we have
\begin{align}
    \label{eq:BMatElems}
    \left(z^{(k)}, \FKtabT C_Y^{-1} Y\right) = T^{(k)T} C_Y^{-1} Y\, .
\end{align}
Denoting by $d^\perp$ the dimension of the subspace orthogonal to $\text{ker}\
\Theta$, $H^\perp$ is a $d^\perp\times d^\perp$ symmetric matrix, whose
eigenvalues and eigenvectors satisfy
\begin{align}
    H^\perp_{kk'} w^{(i)}_{k'} = h^{(i)} w^{(i)}_{k}\, .
\end{align}
The solution to Eq.~\eqref{eq:FlowPerp} can be written as the sum of the
solution of the homogeneous equation, $\hat{f}^{\perp}_{t,k}$, and a particular
solution of the full equation. The solution of the homogeneous equation is
\begin{align}
    \label{eq:HomoSoln}
    \hat{f}^{\perp}_{t,k} = \sum_{i=1}^{d^\perp} f^{\perp}_{0,i} e^{-h^{(i)}t} w^{(i)}_k\, ,
\end{align}
where
% ~\footnote{ Note that here the scalar product is computed in the subspace
%     orthogonal to the kernel of $\Theta$,
%     \[
%         \left(w^{(i)}, f^\perp_0\right) = \sum_{k=1}^{d_\perp} w^{(i)}_{k} f^\perp_{0,k}
%     \]
% }
\begin{align}
    \label{eq:InitialCi}
    f^{\perp}_{0,i} = \sum_{k=1}^{d_\perp} w^{(i)}_k f^\perp_{0,k}\, ,
        %= \left(w^{(i)}, f^\perp_0\right)\, ,
\end{align}
guarantees that the initial condition $\hat{f}^\perp_{t,k}=f^\perp_{0,k}$ is
satisfied. Similarly, if we define
\begin{align}
    \label{eq:BiDef}
    \Upsilon^{(i)} = \sum_{k=1}^{d_\perp} w^{(i)}_k B^\perp_{k}\, ,
        %= \left(w^{(i)}, B^\perp\right)\, ,
\end{align}
then
\begin{align}
    \label{eq:PartSol}
    \check{f}^\perp_{t,k} = \sideset{}{'}\sum_{i} \frac{1}{h^{(i)}} \Upsilon^{(i)}
        \left(1 - e^{-h^{(i)}t}\right) w^{(i)}_k\, ,
\end{align}
where the sum only involves the non-zero modes of $H^\perp$, is a particular
solution of the inhomogeneous equation, which satisfies the boundary condition
$\check{f}^{\perp}_{0,k}=0$. Finally, the solution of the flow equation in the
subspace orthogonal to $\text{ker}\ \Theta$ is
\begin{align}
    f^\perp_{t,k}
    \label{eq:FlowSolution}
        &= \hat{f}^\perp_{t,k} + \check{f}^\perp_{t,k}
        % &= \sum_{i=1}^{d^\perp}  \left(w^{(i)}, f^\perp_0\right) e^{-h^{(i)}t} w^{(i)}_k
        %     + \sideset{}{'}\sum_{i=1}  \frac{1}{h^{(i)}} \left(w^{(i)}, B^\perp\right)
        %         \left(1 - e^{-h^{(i)}t}\right) w^{(i)}_k
        \, .
\end{align}
Finally, collecting the parallel contribution, Eq.~\eqref{eq:FlowParallel}, and
the solution of the orthogonal component, Eq.~\eqref{eq:FlowSolution}, yields a
simple expression,
\begin{align}
    \label{eq:AnalyticSol}
    f_{t,\alpha}
        = U(t)_{\alpha\alpha'} f_{0,\alpha'} + V(t)_{\alpha I} Y_{I}\, .
\end{align}
The two evolution operators $U(t)$ and $V(t)$ have lengthy, yet explicit,
expressions, which we summarise here: 
\begin{align}
    U(t)_{\alpha\alpha'} = \hat{U}^\perp(t)_{\alpha\alpha'}
        + \check{U}^\perp(t)_{\alpha\alpha'} + U^\parallel_{\alpha\alpha'}\, ,
\end{align}
where
\begin{align}
    \hat{U}^\perp(t)_{\alpha\alpha'}
        = \sum_{k,k'\in\perp} \sqrt{\lambda^{(k)}} z^{(k)}_\alpha 
            \left[\sum_i w^{(i)}_{k} e^{-h^{(i)}t} w^{(i)}_{k'}\right]
            z^{(k')}_{\alpha'} \frac{1}{\sqrt{\lambda^{(k')}}}\, ,
\end{align}
and
\begin{align}
    U^\parallel_{\alpha\alpha'}
        = \sum_{k''\in\parallel} z^{(k)}_\alpha z^{(k)}_{\alpha'} \, .
\end{align}
The contributions from $\check{U}^\perp(t)$ and $V(t)$ are more easily expressed
by introducing the operator
\begin{align}
    \label{eq:MOperatorDef}
    \mathcal{M}(t)_{\alpha\alpha'} 
        = \sum_{k,k'\in\perp} \sqrt{\lambda^{(k)}} z^{(k)}_\alpha 
            \left[\sideset{}{'}\sum_{i} w^{(i)}_{k} \frac{1}{h^{(i)}}\, 
            \left( 1- e^{-h^{(i)}t}\right) w^{(i)}_{k'}\right]
            z^{(k')}_{\alpha'} \sqrt{\lambda^{(k')}}\,. 
\end{align}
Then, we can write
\begin{align}
    \label{eq:UperpCheck}
    \check{U}^\perp(t)
        = - \mathcal{M}(t)\; \FKtabT C_Y^{-1} \FKtab 
            \left[\sum_{k''\in\parallel} z^{(k'')} z^{(k'') T}\right]\, ,
\end{align}
and
\begin{align}
    V(t) = \mathcal{M}(t)\; \FKtabT C_Y^{-1}\, ,
\end{align}
where we note that the term in the bracket in Eq.~\eqref{eq:UperpCheck} is
simply the projector on the kernel of the NTK. The four terms that appear in the
analytical solution have a clear physical interpretation:
\begin{itemize}
    \item The first term $\hat{U}^\perp(t)$ suppresses the components of the
    initial condition that lie in the subspace orthogonal to the kernel of the
    NTK. These are the components that are learned by the network during
    training. While the trained solution still depends on its value at
    initialisation, that dependence is suppressed during training. This
    suppression is exponential in the training time, and the rates are given by
    the eigenvalues of $H^{\perp}$.
    \item The contribution from $U^\parallel$ yields the component of the
    initial condition that lies in the kernel of the NTK. As such, those
    components remain unchanged during training and are part of the trained
    field at all times $t$. 
    \item The two remaining contributions are best understood by combining them
    together,
    \begin{align}
        \label{eq:DataCorrectedInference}
        \check{U}^{\perp}(t) f_{0} + V(t) Y 
            = \mathcal{M}(t)\; \FKtabT C_Y^{-1} \left[Y - \FKtab f_{0}^{\parallel}\right]\, .
    \end{align}
    The parallel component of the initial condition $f_{0}^{\parallel}$ does not
    evolve during training, and therefore it yields a contribution $\FKtab
    f_{0}^{\parallel}$ to the theoretical prediction of the data points at all
    times $t$. This is taken into account by subtracting this contribution from
    the data, before the inference is performed.
\end{itemize}
The solution in Eq.~\eqref{eq:AnalyticSol} is the main result of this section.
It shows that the training process can be described as the sum of a linear
transformation of the initial fields $f_{0,\alpha}$, which are the
preactivations of the output layer at initialisation, and a linear
transformation of the data $Y_I$. The two transformations depend on the flow
time $t$ and are given by the evolution operators $U(t)$ and $V(t)$.
Eq.~\eqref{eq:AnalyticSol} encodes the information on the central value and the
variance of the trained fields, and any other quantity that is derived from the
PDFs.

Before moving on, a clarification is in order. We have been using the term
``initial condition'' without specifying what it exactly means in this context.
Indeed, the initial condition can be taken to be the value of the trained
network at the $T_{\rm ref}$, \ie the time at which we pick the NTK. In this
case Eq.~\eqref{eq:AnalyticSol} would provide the continuation of the training
from $T_{\rm ref}$ onward, provided we are in the lazy regime. However, the
initial condition can also be sample from a prior distribution, for instance it
could be a replica of the network ensemble at initialisation.

\subsubsection{Central value and Covariance of the trained fields}
\label{sec:CentralAndCovariance}

\noindent In deriving the analytical solution in Eq.~\eqref{eq:AnalyticSol}, we
implicitly considered a single realization of the frozen NTK over the ensemble.
However, we are interested in the analytical evolution of the ensemble of
trained networks, which can be obtained by considering the different replicas of
the frozen NTK. Thus, the operators $U(t)$ and $V(t)$ that enter
Eq.~\eqref{eq:AnalyticSol} should be considered as random variables distributed
according to the ensemble. Moreover, the initial condition $f_0$ is also a
random variable, while the data fluctuations in the data $Y$ depend on
how the artificial data replicas are constructed.

As a consequence, the analytical solution in Eq.~\eqref{eq:AnalyticSol} becomes
a random variable. We can then characterize its distribution using the mean and
variance across the ensemble of analytical solutions, as shown in
Eq.~\eqref{eq:ReplicaEnsemble}. The central value of the trained field is thus
defined as
\begin{align}
    \label{eq:MeanValAtT}
    \bar{f}_{t,\alpha} = \mathbb{E}\left[f_{t,\alpha}\right]
        = \mathbb{E}\left[U(t)_{\alpha\alpha'} f_{0,\alpha'}\right]
            + \mathbb{E}\left[V(t)_{\alpha I} Y_I\right] \, .
\end{align}
Note that the first term on the right-hand side of Eq.~\eqref{eq:MeanValAtT} can
only be non-zero because of the correlations between $U(t)$ and $f_0$. In the
absence of such correlations, the first term would be given by the product of
the expectation values. In $f_0$ is an ensemble of networks at initialisation,
the we now from the theory in Sec.~\ref{sec:Init} that the expectation value
of the output vanishes.

and hence would vanish up to corrections of order
$\mathcal{O}(1/n)$, since the expectation value of the fields at initialisation
vanishes in the limit of infinitely wide networks. Assuming that the
correlations between the initial fields and the evolution operators vanish, we
can write
\begin{align}
    \label{eq:MeanUt}
    \bar{U}(t)
        &= \mathbb{E}\left[U(t)\right]\, , \\
    \label{eq:MeanVt}
    \bar{V}(t)
        &= \mathbb{E}\left[V(t)\right]\, ,
\end{align}
and
\begin{equation}
    \label{eq:MeanValAtTNoCorr}
    \bar{f}_{t,\alpha} = \bar{U}(t)_{\alpha\alpha'} \bar{f}_{0,\alpha'}
        + \bar{V}(t)_{\alpha I} Y_I = \bar{V}(t)_{\alpha I} Y_I \, .
\end{equation}
The second term in Eq.~\eqref{eq:MeanValAtT}, or equivalently
Eq.~\eqref{eq:MeanValAtTNoCorr}, explicitly shows the contribution of each data
point to the central value of the trained fields at each value of $x_{\alpha}$.
It is worthwhile remarking that in this limit, the central value from the set of
trained networks is a linear combination of the data points, with coefficients
given by the evolution operator $V(t)_{\alpha I}$. It is worth mentioning that
Eq.~\eqref{eq:FlowSolution} resembles the structure of a linear method,
like \eg\ Backus-Gilbert or Gaussian Processes. We will comment further on this
point later in this section.

For the covariance we have 
\begin{align}
    \cov[f_t,f_t^T]
        &= \mathbb{E}\left[U(t) f_0 f_0^T U(t)^T\right] 
            - \mathbb{E}\left[U(t) f_0\right] \mathbb{E}\left[f_0^T U(t)^T\right]  \nonumber \\
        &\quad + \mathbb{E}\left[U(t) f_0 Y^T V(t)^T\right] 
            - \mathbb{E}\left[U(t) f_0\right] \mathbb{E}\left[Y^T V(t)^T\right] \nonumber \\
        &\quad + \mathbb{E}\left[V(t) Y f_0^T U(t)^T\right]
            - \mathbb{E}\left[V(t) Y\right] \mathbb{E}\left[f_0^T U(t)^T\right] \nonumber \\
    \label{eq:CovAtT}
        &\quad + \mathbb{E}\left[V(t) Y Y^T V(t)^T\right]
            - \mathbb{E}\left[V(t) Y\right] \mathbb{E}\left[Y^T V(t)^T\right] \, .
\end{align}
Note that the first and the fourth lines above yield symmetric matrices, while
the third line is just the transpose of the second, thereby ensuring that the
whole covariance matrix is the sum of three symmetric matrices and therefore is
symmetric, 
\begin{align}
    \label{eq:SumOfCovariances}
    \cov[f_t,f_t^T] = C_t^{(00)} + C_t^{(0Y)} + C_t^{(YY)}\, ,
\end{align}
where
\begin{align}
    \label{eq:C00term}
    C_t^{(00)} 
        &= \mathbb{E}\left[U(t) f_0 f_0^T U(t)^T\right] 
        - \mathbb{E}\left[U(t) f_0\right] \mathbb{E}\left[f_0^T U(t)^T\right]\, ,\\
    C_t^{(0Y)}
        &= \mathbb{E}\left[U(t) f_0 Y^T V(t)^T\right] 
        - \mathbb{E}\left[U(t) f_0\right] \mathbb{E}\left[Y^T V(t)^T\right] \nonumber \\
        \label{eq:C0Yterm}
        &\quad + \mathbb{E}\left[V(t) Y f_0^T U(t)^T\right]
            - \mathbb{E}\left[V(t) Y\right] \mathbb{E}\left[f_0^T U(t)^T\right] \, ,\\
    C_t^{(YY)}
        &= \mathbb{E}\left[V(t) Y Y^T V(t)^T\right]
        - \mathbb{E}\left[V(t) Y\right] \mathbb{E}\left[Y^T V(t)^T\right]\, .
\end{align}

% Onset Lazy L0 ===================================
\begin{figure}[t]
  \centering
  \includegraphics[width=0.9\textwidth]{section_4/evolution_vs_trained_ftref_L0.pdf}
  \caption{Comparison of the trained and analytical evolution at the end of
  training. In each panel, the blue curve remains identical. The analytic
  evolution is obtained using the NTK frozen at $T_{\rm ref}$, which is
  different in each panel. The initial condition is taken from the ensemble at
  $T_{\rm ref}$, then evolved analytically until the end of training}
  \label{fig:OnsetLazyL0}
\end{figure}
% =================================================
% Onset Lazy L1 ===================================
\begin{figure}[t]
  \centering
  \includegraphics[width=0.9\textwidth]{section_4/evolution_vs_trained_ftref_L1.pdf} 
  \caption{Same as Fig.~\ref{fig:OnsetLazyL0}, but for L1 closure test data.}
  \label{fig:OnsetLazyL1}
\end{figure}
% =================================================
% Onset Lazy L2 ===================================
\begin{figure}[t]
  \centering
  \includegraphics[width=0.9\textwidth]{section_4/evolution_vs_trained_ftref_L2.pdf} 
  \caption{Same as Fig.~\ref{fig:OnsetLazyL0}, but for L2 closure test data.}
  \label{fig:OnsetLazyL2}
\end{figure}
% =================================================

% Decomposition L0 ================================
\begin{figure}[t]
  \centering
  \includegraphics[width=0.9\textwidth]{section_4/u_v_decomposition_L0.pdf}
  \caption{Decomposition of the trained field into the $U$ and $V$ components
  at different training time with fixed frozen NTK at $T_{\rm ref}$. The initial
  condition is taken from the ensemble at $T_{\rm ref}$, then evolved
  analytically. L0 data is used.}
  \label{fig:FrefDecompositionL0}
\end{figure}
% =================================================
% Decomposition L1 ================================
\begin{figure}[t]
  \centering
  \includegraphics[width=0.9\textwidth]{section_4/u_v_decomposition_L1.pdf} 
  \caption{Same as Fig.~\ref{fig:FrefDecompositionL0}, but for L1 closure test
  data.}
  \label{fig:FrefDecompositionL1}
\end{figure}
% =================================================
% Decomposition L2 ================================
\begin{figure}[t]
  \centering
  \includegraphics[width=0.9\textwidth]{section_4/u_v_decomposition_L2.pdf} 
  \caption{Same as Fig.~\ref{fig:FrefDecompositionL0}, but for L2 closure test
  data.}
  \label{fig:FrefDecompositionL2}
\end{figure}
% =================================================

\subsection{Results and Crosschecks}
\label{sec:TrainClosure}

The analytical solution in Eq.~\eqref{eq:AnalyticSol} sheds a new light onto the
behaviour of the numerical neural network training. In order to study the
training process, the NNPDF collaboration has successfully developed so-called
{\em closure tests}, which we are going to adopt here. 

A closure test uses synthetic data, generated using a known set of PDFs, to
train the neural network. The PDFs used for generating the data are called here
{\em input}\ PDFs. The results of the training are then compared to the known
input PDFs; the performance of the training algorithm and the NN architecture
are assessed by quantifying the comparison between trained PDFs and input PDFs.
Following the original presentation in Ref.~\cite{NNPDF:2014otw}, we distinguish
three levels of closure tests, which are defined by the complexity of the data
used to train the NNs. We use the standard NNPDF nomenclature and refer to these
three levels as level-0 (L0), level-1 (L1), and level-2 (L2) closure tests, and
we denote the input PDFs used to generate the data as $\fin$.

With the full control over the sought solution at hand, the analytic solution
allows us to perform a number of crosschecks that validate our implementation
and provides insight into the training process. These are discussed in the
following sections.

\subsection{Analytical Results and Crosschecks using L0 data}
\label{sec:AnalyticalChecks}
Let us start by discussing the case of L0 closure tests. In this case, the realization
of the dataset is completely determined by the input PDFs, and the values of the data
are given by
\begin{equation}
    \label{eq:DataL0}
    Y_I = T[\fin]_I
        = \sum_{i=1}^{\nflav} \sum_{\alpha=1}^{\ngrid} \FKtab_{Ii\alpha} \fin_{i\alpha}\, ,
\end{equation}
or equivalently, suppressing the indices,
\begin{equation}
    \label{eq:DataL0NoIndices}
    Y = \FKtab \fin\, .
\end{equation}
Note that using L0 data only affects the second term in
Eq.~\ref{eq:AnalyticSol}\footnote{To be more precise, since the analytical
solution is requires the NTK to be frozen at a certain epoch $T_{\rm ref}$, the
NTK will itself depend on the data used in the training.}. We can then rewrite the
combined term in Eq.~\eqref{eq:DataCorrectedInference} as follows
\begin{align}
  \label{eq:TrainingOnLevelZero}
  \check{U}^\perp(t) f_0 + V(t) Y 
    &= \mathcal{M}(t)\, \FKtabT C_{Y}^{-1} \FKtab\, 
      \left[\fin - f_{0}^\parallel\right]\, .
\end{align}
The subtraction taking place in the square brackets of
Eq.~\eqref{eq:TrainingOnLevelZero} suggests us that the effective function that
the neural network actually sees is not the input function $\fin$ used to
generate the data, but rather the difference between $\fin$ and the component of
the initial function $f_0$ that lies in the subspace spanned by the kernel of
the NTK, \ie\ $f_0^\parallel$. In other words, the parallel component
$f_0^\parallel$, which we remind does not evolve during the analytic training,
acts as a constant ``bias'' in the training process, shifting the effective
input function that the neural network sees. Of course the actual magnitude of
this irreducible noise depends both on how $f_0$ and the kernel of the NTK are
distributed over the ensemble. We will come to this point soon.

Note that the observation above remains true even in the limit of infinite training.
This can be shown using
Interestingly, for $t\to\infty$, we have
\begin{align}
    \label{eq:LevelZeroClosureInfiniteTraining}
    \lim_{t\to\infty} V(t) Y = \finperp + \mathcal{M}_{\infty} M \finpar
\end{align}
and therefore the $V$ component of the trained solution reproduces exactly the
component of the PDF that lies in  the subspace orthogonal to the kernel of
$\Theta$. We compare the asymptotic behaviour of $V(t) Y$ and $\finperp$ in
Fig.~\ref{fig:InfiniteTimeVterm}.

\begin{figure}[t]
  \centering
  \includegraphics[width=\textwidth]{vy_inf_L0.pdf}  
  \caption{Test the $t\to\infty$ limit of the L0 training for different frozen
  NTK. The orange curve represents the projection of the input function $\fin$
  onto the subspace orthogonal to the kernel of the NTK at $T_{\rm ref}$, \ie\
  $\finperp$. The blue curve represents the contribution of the operator $V$,
  computed with the NTK at $T_{\rm ref}$, in the limit of infinite training
  time.}
  \label{fig:InfiniteTimeVterm}
\end{figure}

The second term in the square bracket on the right-hand side of
Eq.~\eqref{eq:TrainingOnLevelZero} is the contribution from the parallel
component at initialisation that does not evolve in the training process. Given
that $f_0$ is almost normally distributed around zero, that term does not
contribute to the central value of the fitted PDF, \ie\ to the average of the
trained solution over replicas. The time evolution of 
\begin{align}
  \label{eq:AverageLevelZeroUcheck}
  \mathbb{E}\left[\mathcal{M}(t)\, \FKtabT C_{Y}^{-1} \FKtab\, 
    f_{0}^\parallel\right]\, ,
\end{align}
is shown in Fig.~\ref{fig:AverageLevelZeroUcheck}.
\begin{figure}[h!]
  \centering
  \includegraphics[width=0.95\textwidth]{Mcal_M_fpar_L2_linear.pdf} 
  \caption{Test of the average of the parallel contribution for different
  epochs. The reference epoch at which the frozen NTK is chosen is $T_{\rm ref}
  = 10000$. L2 data is used in the plot.}
  \label{fig:AverageLevelZeroUcheck}
\end{figure}

\FloatBarrier

\subsubsection{Convergence of the Analytical Solution}
\label{sec:CheckAnalyticalConvergence}

At the beginning of the training process there is clearly no difference between
the analytical solution and the trained solution. By construction both AS and TS
are given by the outcome of the neural network at initialisation, as discussed
in Sect.~\ref{sec:Init}. In the early stages of training AS and TS differ as
expected. Indeed, the analytical solution is computed using the frozen NTK at
$T_{\rm ref}$, while the trained solution evolves with an NTK that is still
changing as shown in Sect.~\ref{sec:NTKPheno}. Since the NTK at $T_{\rm ref}$ is
already aligned with the solution, the AS converges faster to the target
solution, while the TS takes more epochs before starting to evolve in the right
direction. The two solutions are compared for different training times in
Fig.~\ref{fig:xT3_analytical_vs_trained} after $T=500, 1000$ and 10000 epochs.
The plots in the left column correspond to synthetic L0 data, while the ones in
the right column are obtained using L2 data. The analytical solution is obtained
using a frozen NTK at $T_{\rm ref}=20000$. In both cases the analytical solution
agrees with the trained one for $T=10000$.

\begin{itemize}
  \item Trained vs Analytical solution at different training times (grid)
  \item Decomposition of U and V
  \item Error decomposition
\end{itemize}

\subsubsection{Error decomposition}

\subsubsection{Infinite Training Time}
In the limit of infinite training time, the evolution operators $U(t)$ and
$V(t)$ simplify and yield an elegant interpretation of the minimum of the cost
function. For large training times, we have
\begin{align}
    \label{eq:UhatInfty}
    \hat{U}^\perp_{\infty, \alpha\alpha'}
        &= \lim_{t\to\infty}\hat{U}^\perp(t)_{\alpha\alpha'} = 0\, \\
    \label{eq:MOperatorInfty}
    \mathcal{M}_{\infty, \alpha\alpha'} 
        &= \lim_{t\to\infty}\mathcal{M}(t)_{\alpha\alpha'} = \sum_{k,k'\in\perp} \sqrt{\lambda^{(k)}} z^{(k)}_\alpha 
        \left[\sideset{}{'}\sum_{i} w^{(i)}_{k} \frac{1}{h^{(i)}}\, 
        w^{(i)}_{k'}\right] z^{(k')}_{\alpha'} \sqrt{\lambda^{(k')}}\, ,
\end{align}
and explicit expressions for $\check{U}^\perp_{\infty}$ and $V_{\infty}$ are
obtained from $\mathcal{M}_{\infty}$. The term in the square bracket in
Eq.~\eqref{eq:MOperatorInfty} is the spectral decomposition of the pseudoinverse
of $H^\perp$ in $d_\perp$ orthogonal subspace. So, the operator
$\mathcal{M}_{\infty}$ acts as follow on a field $f_{\alpha}$:
\begin{enumerate}
    \item The term on the right of the square bracket computes the coordinate
    $f_k$ introduced in Eq.~\eqref{eq:OrthogonalComponents}. The $f_k$ are a set
    of coordinates for the component $f^\perp$ of the field that evolves during
    training, 
    \begin{align}
        \label{eq:RightOfTheBracket}
        f^\perp = \sum_{k\in\perp} \sqrt{\lambda^{(k)}} f_k\, z^{(k)}\,  .
    \end{align}
    \item The term in the square bracket applies the pseudoinverse to the
    coordinates $f_k$, 
    \begin{align}
        \label{eq:ApplyPseudoInv}
        f'_k = \left(H^\perp\right)^+_{kk'} f_{k'}\, .
    \end{align}
    \item The final term on the left of the square bracket reconstructs the full
    field corresponding to the modified $f'_{k}$,
    \begin{align}
        \label{eq:LeftOfTheBracket}
        f^{'\perp} = \sum_{k\in\perp} \sqrt{\lambda^{(k)}} f'_{k}\, z^{(k)}\, .
    \end{align}
    
\end{enumerate}

As discussed at the end of Sect.~\ref{sec:Lazy} it is convenient to combine the
contributions of $\check{U}^\perp_{\infty}$ and $V_{\infty}$,
\begin{align}
    \label{eq:DataCorrectedInferenceAtInfty}
    \check{U}^{\perp}_{\infty} f_{0} + V_{\infty} Y 
        = \mathcal{M}_{\infty}\; \FKtabT C_Y^{-1} \left[Y - \FKtab f_{0}^{\parallel}\right]\, .
\end{align}
The contribution to the observables from the parallel components of $f$ does not
change during training, therefore that contribution is subtracted from the data
and the orthogonal components of $f$ are adjusted to minimize the $\chi^2$ of
the corrected data. The minimum of the $\chi^2$ in the orthogonal subspace is
found applying $\mathcal{M}_{\infty}$, \ie\ by projecting in the orthogonal
subspace, applying the pseudoinverse and finally recompute the full field as
detailed above.


\subsection{Connection with Gaussian Processes}

The infinite 


\newpage

% Collecting all terms yields a simple (and useful!) expression, \begin{align}
% \label{eq:AnalyticSol} f_{t,\alpha} = U(t)_{\alpha\alpha'} f_{0,\alpha'} +
% V(t)_{\alpha I} Y_{I}\, . \end{align} The two evolution operators $U(t)$ and
% $V(t)$ have lengthy, yet explicit, expressions, which we summarise here or
% move to an appendix: \ac{$U^\parallel$ should not be time-dependent, right?}
% \begin{align} U(t)_{\alpha\alpha'} = \hat{U}^\perp(t)_{\alpha\alpha'} +
% \check{U}^\perp(t)_{\alpha\alpha'} + U^\parallel_{\alpha\alpha'}\, ,
% \end{align} where \begin{align} \hat{U}^\perp(t)_{\alpha\alpha'} &= \sum_i
% Z^{(i)}_{\alpha} e^{-h^{(i)}t} Z^{(i)}_{\alpha'}\, , \\
%     Z^{(i)}_{\alpha} &= \sum_{k\in\perp} \sqrt{\lambda^{(k)}} z^{(k)}_\alpha
%         w^{(i)}_{k}\, , \end{align} and \begin{align}
%         \check{U}^\perp(t)_{\alpha\alpha'} &= \sideset{}{'}\sum_{i}
%         Z^{(i)}_{\alpha} \frac{1}{h^{(i)}} \left(1 - e^{-h^{(i)}t}\right)
%         \tilde{Z}^{(i)}_{\alpha'}\, , \\
%     \tilde{Z}^{(i)}_{\alpha} &= -\sum_{k'\in\perp} \sum_{k''\in\parallel}
%         w^{(i)}_{k'} \sqrt{\lambda^{(k')}} T^{(k') T} C_Y^{-1} T^{(k'')}
%         z^{(k'')}_{\alpha}\, , \end{align} and \begin{align}
%         U^\parallel_{\alpha\alpha'} = \sum_{k\in\parallel} z^{(k)}_\alpha
%         z^{(k)}_{\alpha'} \, , \end{align} and \begin{align} V(t)_{\alpha I} =
%         \sideset{}{'}\sum_{i} Z^{(i)}_{\alpha} \frac{1}{h^{(i)}} \left(1 -
%         e^{-h^{(i)}t}\right) \tilde{T}^{(i)}_{I}\, , \end{align} and
%         \begin{align} \tilde{T}^{(i)}_{I} = \sum_{k'\in\perp} w^{(i)}_{k'}
%         \sqrt{\lambda^{(k')}} T^{(k')}_J \left(C_Y^{-1}\right)_{JI}\, .
%         \end{align}

% \subsection{Behaviour of the solution}
% \label{sec:BehaviourOfSolution}
% The solution in Eq.~\eqref{eq:AnalyticSol} is the main result of this section.
% It shows that the training process can be described as the sum of a linear
% transformation of the initial fields $f_{0,\alpha}$, which are the
% preactivations of the output layer at initialisation, and a linear
% transformation of the data $Y_I$. The two transformations depend on the flow
% time $t$ and are given by the evolution operators $U(t)$ and $V(t)$.
% Eq.~\eqref{eq:AnalyticSol} encodes the information on the central value and the
% variance of the trained fields, and any other quantity that is derived from the
% PDFs. 

% % Deprecated
% %%%%%%%%%%%%%%%%%%%%%%%%%%%%%%%%%%%%%%%%%%%%%%%%%%%%%%%%%%%%%%%%%%%%%% 
% % \subsubsection{Central value of the trained fields} \label{sec:CentralValue}
% % The central values of the trained fields is obtained by taking the expectation
% % value of Eq.~\eqref{eq:AnalyticSol} over the initial fields, which are
% % approximately Gaussian distributed at initialisation, and over the
% % fluctuations of the NTK, \begin{align} \label{eq:MeanValAtT}
% % \bar{f}_{t,\alpha} = \mathbb{E}\left[f_{t,\alpha}\right] =
% % \mathbb{E}\left[U(t)_{\alpha\alpha'} f_{0,\alpha'}\right] +
% % \mathbb{E}\left[V(t)_{\alpha I} Y_I\right] \, . \end{align} Note that the
% % first term on the right-hand side of Eq.~\eqref{eq:MeanValAtT} can only be
% % non-zero because of the correlations between $U(t)$ and $f_0$. In the absence
% % of such correlations, the first term would be given by the product of the
% % expectation values and hence would vanish up to corrections of order
% % $\mathcal{O}(1/n)$, since the expectation value of the fields at
% % initialisation vanishes in the limit of infinitely wide networks. Assuming
% % that the correlations between the initial fields and the evolution operators
% % vanish, we can write \begin{align} \label{eq:MeanUt} \bar{U}(t) &=
% % \mathbb{E}\left[U(t)\right]\, , \\
% %     \label{eq:MeanVt} \bar{V}(t) &= \mathbb{E}\left[V(t)\right]\, ,
% %     \end{align} and \begin{equation} \label{eq:MeanValAtTNoCorr}
% %     \bar{f}_{t,\alpha} = \bar{U}(t)_{\alpha\alpha'} \bar{f}_{0,\alpha'} +
% %     \bar{V}(t)_{\alpha I} Y_I = \bar{V}(t)_{\alpha I} Y_I \, . \end{equation}

% % The second term in Eq.~\eqref{eq:MeanValAtT}, or equivalently
% % Eq.~\eqref{eq:MeanValAtTNoCorr}, explicitly shows the contribution of each
% % data point to the central value of the trained fields at each value of
% % $x_{\alpha}$. It is worthwhile remarking that in this limit, the central value
% % from the set of trained networks is a linear combination of the data points,
% % with coefficients given by the evolution operator $V(t)_{\alpha I}$. In this
% % respect, the trained NNs behave like other well-known linear methods for
% % solving inverse problems, like \eg\ Backus-Gilbert or Gaussian Processes. It
% % is interesting to compare the matrix $V(t)$ with the corresponding linear
% % operators that enter in Backus-Gilbert or Gaussian Processes solutions. 

% \paragraph{Central value at infinite training time.}

% In the limit of infinite training time, the evolution operators $U(t)$ and
% $V(t)$ simplify and yield an elegant interpretation of the minimum of the cost
% function. For large training times, we have
% \begin{align}
%     \label{eq:UhatInfty}
%     \hat{U}^\perp_{\infty, \alpha\alpha'}
%         &= \lim_{t\to\infty}\hat{U}^\perp(t)_{\alpha\alpha'} = 0\, \\
%     \label{eq:MOperatorInfty}
%     \mathcal{M}_{\infty, \alpha\alpha'} 
%         &= \lim_{t\to\infty}\mathcal{M}(t)_{\alpha\alpha'} = \sum_{k,k'\in\perp} \sqrt{\lambda^{(k)}} z^{(k)}_\alpha 
%         \left[\sideset{}{'}\sum_{i} w^{(i)}_{k} \frac{1}{h^{(i)}}\, 
%         w^{(i)}_{k'}\right] z^{(k')}_{\alpha'} \sqrt{\lambda^{(k')}}\, ,
% \end{align}
% and explicit expressions for $\check{U}^\perp_{\infty}$ and $V_{\infty}$ are
% obtained from $\mathcal{M}_{\infty}$. The term in the square bracket in
% Eq.~\eqref{eq:MOperatorInfty} is the spectral decomposition of the pseudoinverse
% of $H^\perp$ in $d_\perp$ orthogonal subspace. So, the operator
% $\mathcal{M}_{\infty}$ acts as follow on a field $f_{\alpha}$:
% \begin{enumerate}
%     \item The term on the right of the square bracket computes the coordinate
%     $f_k$ introduced in Eq.~\eqref{eq:OrthogonalComponents}. The $f_k$ are a set
%     of coordinates for the component $f^\perp$ of the field that evolves during
%     training, 
%     \begin{align}
%         \label{eq:RightOfTheBracket}
%         f^\perp = \sum_{k\in\perp} \sqrt{\lambda^{(k)}} f_k\, z^{(k)}\,  .
%     \end{align}
%     \item The term in the square bracket applies the pseudoinverse to the
%     coordinates $f_k$, 
%     \begin{align}
%         \label{eq:ApplyPseudoInv}
%         f'_k = \left(H^\perp\right)^+_{kk'} f_{k'}\, .
%     \end{align}
%     \item The final term on the left of the square bracket reconstructs the full
%     field corresponding to the modified $f'_{k}$,
%     \begin{align}
%         \label{eq:LeftOfTheBracket}
%         f^{'\perp} = \sum_{k\in\perp} \sqrt{\lambda^{(k)}} f'_{k}\, z^{(k)}\, .
%     \end{align}
    
% \end{enumerate}

% As discussed at the end of Sect.~\ref{sec:Lazy} it is convenient to combine the
% contributions of $\check{U}^\perp_{\infty}$ and $V_{\infty}$,
% \begin{align}
%     \label{eq:DataCorrectedInferenceAtInfty}
%     \check{U}^{\perp}_{\infty} f_{0} + V_{\infty} Y 
%         = \mathcal{M}_{\infty}\; \FKtabT C_Y^{-1} \left[Y - \FKtab f_{0}^{\parallel}\right]\, .
% \end{align}
% The contribution to the observables from the parallel components of $f$ does not
% change during training, therefore that contribution is subtracted from the data
% and the orthogonal components of $f$ are adjusted to minimize the $\chi^2$ of
% the corrected data. The minimum of the $\chi^2$ in the orthogonal subspace is
% found applying $\mathcal{M}_{\infty}$, \ie\ by projecting in the orthogonal
% subspace, applying the pseudoinverse and finally recompute the full field as
% detailed above.
% %%%%%%%%%%%%%%%%%%%%%%%%%%%%%%%%%%%%%%%%%%%%%%%%%%%%%%%%%%%%%%%%%%%%%%

\FloatBarrier
\section{Conclusions}
\label{sec:conclusions}

Our present age is marked by unprecedented advancements in machine learning
techniques, whose applications span many scientific domains -- PDF determination
being one of them. However, we believe that we ought to move from mere witnesses
to active contributions. In fact, it is of paramount importance to understand
how these new techniques behave when applied to complex problems such as PDF
determination.

In this work, we have taken a step forward in this direction. We have
investigated a novel treatment of the learning process in the context of PDF
fitting by exploring the training dynamics in the functional space of the neural
network. In fact, the NTK can be used to unravel complex dynamics obfuscated by
the training algorithms commonly employed in PDF fits. We have shown that the
properties of the NTK are highly entangled with the fitting results, and that a
proper understanding of its structure can provide precious insights on the
learning process. Notably, we have developed, under certain assumptions, an
analytical description of how the neural network evolves during training,
enabling us to better understand the NNPDF methodology and its dependence on the
underlying model architecture. To the best of our knowledge, such description
has not yet been attempted in the context of ill-defined inverse problems.

Yet, this work is far from being conclusive. As a pioneering study, we believe
that the most important contribution of this work has been to open the
discussion on the role of the NTK in PDF fits, identifying a set of relevant
diagnostic metrics that can be explored in future works. Indeed, many aspects
merit further investigations. First of all, we are unable to anticipate the
outcome of a similar analysis when applied to a more complex and realistic
framework, including multiple fitted flavors and real data. Furthermore,
further insights could be gained by exploring different architectures and,
possibly, parametrisations beyond neural networks.

In spite of the simplified framework adopted in the present study, our findings
highlight the complexity and richness of the learning process, confirming how
little is understood regarding the origin of PDF uncertainties. This poses a
significant challenge in light of the improved precision of forthcoming
measurements. We believe that the tools presented here can help address this
gap. 

\section*{Acknowledgements}
LDD is suppoorted by an STFC Consolidated Grant (ST/T000600/1, ST/X000494/1). 

\newpage

% Appendices
\appendix
\appendix
\section{The BCDMS dataset for $T_3$}
\label{app:dataset}

The analysis presented in this work employs a set of synthetic data points
generated using a known underlying law $\fin$ that we seek to reconstruct.
Analogous to Ref.~\cite{Candido:2024hjt}, the pseudo-data are constructed by
convolving $\fin$ with FK tables whose kinematic dependence is specified by
measurements of the structure function $F_2$ on proton and deuterium targets
from the BCDMS collaboration~\cite{Benvenuti:1989fm}. By combining these
measurements, one can construct the observable $F_2^p - F_2^d$ that, at
next-to-next-to-leading order (NNLO) in QCD and under the assumption of
isoscalarity for the deuterium nucleus, provides a clean probe of the
non-singlet triplet PDF combination $T_3 = u^+ - d^+$, where $u^+ = u + \bar{u}$
and $d^+ = d + \bar{d}$. The factorisation formula in Eq.~\eqref{eq:TheoryPred}
then simplifies to 
\begin{equation}
F_2^p - F_2^d = C_{T_3} \otimes T_3,
\end{equation}
where $C_{T_3}$ is the corresponding Wilson coefficient computed in perturbative
QCD, and $\otimes$ is a short-hand notation that denotes the convolution as in
Eq.~\eqref{eq:TheoryPred}. The construction of the FK tables needed to compute
the predictions, as well as the construction of the covariance matrix, are
identical to Ref.~\cite{Candido:2024hjt}, to which the reader is referred for
further details. This yields a total of 333 data points, which then reduce to
248 points after applying the kinematic cuts.

Following the closure test framework developed by the NNPDF
collaboration~\cite{DelDebbio:2021whr,NNPDF:2021njg}, we generate
data with three different levels of noise, labelled as Level 0 (L0), Level 1 (L1)
and Level 2 (L2) data. Furthermore, we use the non-singlet triplet $xT_3$ from the
NNPDF4.0 parton set~\cite{NNPDF:2021njg} as the input law $\fin$. In the following,
we summarise the definition of the different levels of pseudo-data.

\paragraph{Level 0}
The pseudo-data are generated without any experimental noise, \ie, by using
the input function and the FK tables as follows
\begin{equation}
Y_{L0} =  \FKtab \fin.
\end{equation}
In this ideal scenario, the analysis should reproduce the input $\fin$, though
some residual reconstruction error may remain in the kinematic region not
covered by the FK tables. Level 0 assesses the intrinsic bias of the methodology,
as any neural network replica will be trained on the same data points $Y_{L0}$.

\paragraph{Level 1}
In this case, the experimental noise is added on top of the L0 data, by sampling from the
multivariate normal distribution with the full experimental covariance matrix
$C_Y$ provided by the BCDMS collaboration
\begin{equation}
Y_{L1} =  Y_{L0} + \eta, \quad \textrm{where} \quad \eta \sim \mathcal{N}(0, C_Y).
\end{equation}
This case is closer to actual experimental data, where the ``true'' value is
blurred by the presence of noise. Note however that we are not yet propagating
the experimental uncertainties into the uncertainties of the fitted PDF, as the added
noise is fixed over all replicas.

\paragraph{Level 2}
Finally, we generate L2 pseudo-data by adding a different noise realisation to each
replica, sampled from the same multivariate normal distribution
\begin{equation}
Y_{L2}^{(k)} =  Y_{L1} + \xi^{(k)}, \quad \textrm{where} \quad \xi^{(k)} \sim \mathcal{N}(0, C_Y).
\end{equation}
This represents the most realistic scenario where both model and data uncertainties
are present. In this case, each neural network replica will be trained on a
different set of data points $Y_{L2}^{(k)}$.
\section{Dependence on the Architecture of the NTK}
\label{sec:NTKArchDep}

It is interesting to consider what happens to the picture sketched so far as the
architecture of the neural networks is varied. In Fig.~\ref{fig:NTKTimeDiffArch}
we compare the time dependence of the Frobenius norm of the NTK and the
variation of the first three eigenvalues for two different architectures. The
smaller network is the $[28,20]$ used in the standard NNPDF analyses and in this
work, while the large one is a $[100,100]$ network, which is closer to the
infinite-width limit. For illustration purposes we focus in this plot on L1 data
and only three eigenvalues rather than the five we examined above. The
quantitative features are exactly the same for L0 and L2 data, and adding the
fourth and fifth eigenvalues does not add unexpected behaviours compared to what
we observe in Fig.~\ref{fig:NTKTime}. It is interesting to remark that the onset
of lazy training is slower for the larger network. This is to be expected if we
interpret the early stages of training as a phase where the network identifies
the learnable features in the space of functions that it can parametrize. For a
larger network, the space of parametrized functions is larger and the
identification of the physical features takes a larger number of epochs. 
% ===================================
\begin{figure}[t]
  \centering
  \includegraphics[width=0.45\textwidth]{appendix_arch/delta_ntk_arch.pdf}
  \includegraphics[width=0.45\textwidth]{appendix_arch/ntk_eigvals_single_plot_arch.pdf}
  \caption{Comparison of the variation of the NTK during training (left) and the
  first three eigenvalues (right) for two different architectures with sizes
  $[28,20]$ and $[100,100]$ respectively. In both cases, L0 data is used. In the
  left plot, error bands represent the standard deviation over the ensemble of
  replicas. In the right plot, solid lines represent the median over the
  ensemble of networks, while solid bands correspond to 68\% confidence level.}
  \label{fig:NTKTimeDiffArch}
\end{figure}
% ===================================

\FloatBarrier
\section{Cut-off Tolerance of the NTK spectrum}
\label{app:cutoff}

% Bibliography
\bibliographystyle{unsrt}
\bibliography{ntk.bib}

\end{document}