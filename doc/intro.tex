

A first paragraph on precision physics at the LHC and the need for robust determinations of PDFs.

The extraction of PDFs from experimental data is a classic example of an inverse problem,
namely the reconstruction of a function $f(x)$ from a finite set of data points
$Y_I$, where the index $I=1, \ldots, \ndat$.1\footnote{When omitting the data index $I$, we will always
assume $Y \in \mathbb{R}^{\ndat}$.} In particular, for this study, we will focus
on DIS data, which depend
linearly on the function $f(x)$. The theoretical prediction for the data point $Y_I$ is denoted
\begin{equation}
    \label{eq:TheoryPred}
    T_I[f] = \sum_{i=1}^{\nflav} \int dx\, C_{Ii}(x) f_{i}(x)\, ,
\end{equation}
where $C_{Ii}(x)$ is a coefficient function, known to some given order in perturbation theory,
$i$ labels the parton flavor, and $f_i(x)$
is the PDF (or set of PDFs) that we want to determine.

Trying to determine a function $f$ in an infinite dimensional space of solutions with a finite
set of data leads to an ill-defined problem, whose solution will depend on assumptions made.
In particular, the choice of a parametrization for $f$ leads to a bias in the space
of solutions that can be obtained. Together with the fit methodology, the parametrization also
determines the propagation of the error on the data to the error on the fitted solution. Understanding
the bias and the variance of the fitted PDF is therefore a major challenge for precision physics.

Following the ideas highlighted in Refs.~\cite{DelDebbio:2021whr,Candido:2024hjt}, the solution
of the inverse problem is conveniently phrased in
a Bayesian framework. The functions $f_i$ are promoted to stochastic processes; for any grid
of points $x_{\alpha}$, $\alpha=1, \ldots, \ngrid$, the vector $f_{i\alpha}=f_{i}(x_{\alpha})$ is a
vector of $\nflav\times\ngrid$ stochastic variables, for which we introduce a prior distribution $p(f)$.~\footnote{Following
the same convention used for the data, when omitting the grid index $\alpha$, and/or the flavor index $i$, we will always refer to a
vector $f \in \mathbb{R}^{\nflav\times\ngrid}$.}

Any fitting procedure is interpreted as a recipe that yields the posterior distribution
$\tilde{p}(f)$.

In this study, probability distributions are represented by ensembles of i.i.d. replicas.
So, for instance, the prior distribution $p(f)$ is described by an ensemble
\begin{equation}
    \label{eq:RepDef}
    \left\{f^{(k)} \in \mathbb{R}^{\nflav\times\ngrid}; k=1, \ldots, \nreps\right\}\, ,
\end{equation}
drawn from the distribution $p$, so that
\begin{equation}
    \label{eq:ReplicaEnsemble}
    \mathbb{E}_{p}[O(f)] = \frac{1}{\nreps} \sum_{k=1}^{\nreps} O(f^{(k)})\, ,
\end{equation}
for any observable $O$ that is built from the PDFs.

The prior distribution $p(f)$ is defined by initializing a set of replicas using a Glorot normal 
initializer~\cite{glorot2010understanding}.
The result of this initialization is discussed below in Sec.~\ref{sec:Init}.
For each replica, a new set of data $Y^{(k)}$ is generated from an $\ndat$ dimensional Gaussian distribution
centred at the experimental central value $Y$, with the covariance given by the experimental covariance
matrix $C_Y$,
\begin{equation}
    \label{eq:ExpReplicaDistr}
    Y^{(k)} \sim \mathcal{N}\left(Y, C_Y\right)\, .
\end{equation}
Each replica $f^{(k)}$ is trained on its corresponding data set $Y^{(k)}$. We denote the replicas at training time $t$,
$f^{(k)}_{t} \in \mathbb{R}^{\nflav\times\ngrid}$. Stopping the training at time $T$, the posterior probability
distribution is represented by the set of replicas
$\left\{f^{(k)}_{T}\in \mathbb{R}^{\nflav\times\ngrid}; k=1, \ldots, \nreps\right\}$, so that averages over the posterior
distribution are computed as
\begin{equation}
    \label{eq:PostEnsemble}
    \mathbb{E}_{\tilde{p}}[O(f)] = \frac{1}{\nreps} \sum_{k=1}^{\nreps}
        O\left(f^{(k)}_{T}\right)\, .
\end{equation}
All knowledge about the solution of the inverse problem, $f$, is encoded in the posterior
$\tilde{p}$ and is expressed as expectation values of observables $O$ using
Eq.~\eqref{eq:PostEnsemble}.

\FloatBarrier
