\appendix
\section{The BCDMS dataset for $T_3$}
\label{app:dataset}

The analysis presented in this work employs a set of synthetic data points
generated using a known underlying law $\fin$ that we seek to reconstruct.
Analogous to Ref.~\cite{Candido:2024hjt}, the pseudo-data are constructed by
convolving $\fin$ with FK tables whose kinematic dependence is specified by
measurements of the structure function $F_2$ on proton and deuterium targets
from the BCDMS collaboration~\cite{Benvenuti:1989fm}. By combining these
measurements, one can construct the observable $F_2^p - F_2^d$ that, at
next-to-next-to-leading order (NNLO) in QCD and under the assumption of
isoscalarity for the deuterium nucleus, provides a clean probe of the
non-singlet triplet PDF combination $T_3 = u^+ - d^+$, where $u^+ = u + \bar{u}$
and $d^+ = d + \bar{d}$. The factorisation formula in Eq.~\eqref{eq:TheoryPred}
then simplifies to 
\begin{equation}
F_2^p - F_2^d = C_{T_3} \otimes T_3,
\end{equation}
where $C_{T_3}$ is the corresponding Wilson coefficient computed in perturbative
QCD, and $\otimes$ is a short-hand notation that denotes the convolution as in
Eq.~\eqref{eq:TheoryPred}. The construction of the FK tables needed to compute
the predictions, as well as the construction of the covariance matrix, are
identical to Ref.~\cite{Candido:2024hjt}, to which the reader is referred for
further details. This yields a total of 333 data points, which then reduce to
248 points after applying the kinematic cuts.

Following the closure test framework developed by the NNPDF
collaboration~\cite{DelDebbio:2021whr,NNPDF:2021njg}, we generate
data with three different levels of noise, labelled as Level 0 (L0), Level 1 (L1)
and Level 2 (L2) data. Furthermore, we use the non-singlet triplet $xT_3$ from the
NNPDF4.0 parton set~\cite{NNPDF:2021njg} as the input law $\fin$. In the following,
we summarise the definition of the different levels of pseudo-data.

\paragraph{Level 0}
The pseudo-data are generated without any experimental noise, \ie, by using
the input function and the FK tables as follows
\begin{equation}
Y_{L0} =  \FKtab \fin.
\end{equation}
In this ideal scenario, the analysis should reproduce the input $\fin$, though
some residual reconstruction error may remain in the kinematic region not
covered by the FK tables. Level 0 assesses the intrinsic bias of the methodology,
as any neural network replica will be trained on the same data points $Y_{L0}$.

\paragraph{Level 1}
In this case, the experimental noise is added on top of the L0 data, by sampling from the
multivariate normal distribution with the full experimental covariance matrix
$C_Y$ provided by the BCDMS collaboration
\begin{equation}
Y_{L1} =  Y_{L0} + \eta, \quad \textrm{where} \quad \eta \sim \mathcal{N}(0, C_Y).
\end{equation}
This case is closer to actual experimental data, where the ``true'' value is
blurred by the presence of noise. Note however that we are not yet propagating
the experimental uncertainties into the uncertainties of the fitted PDF, as the added
noise is fixed over all replicas.

\paragraph{Level 2}
Finally, we generate L2 pseudo-data by adding a different noise realisation to each
replica, sampled from the same multivariate normal distribution
\begin{equation}
Y_{L2}^{(k)} =  Y_{L1} + \xi^{(k)}, \quad \textrm{where} \quad \xi^{(k)} \sim \mathcal{N}(0, C_Y).
\end{equation}
This represents the most realistic scenario where both model and data uncertainties
are present. In this case, each neural network replica will be trained on a
different set of data points $Y_{L2}^{(k)}$.