% ===================================
\begin{figure}[t!]
  \centering
  \includegraphics[width=0.65\textwidth]{plots/xT3_2.png}
  \caption{Comparison of the trained solution at the end of training (eot) and analytical solution.
  The boundary condition of the analytical solution is the trained function at $T_{\rm ref} < T_{\rm eot}$,
  and then evolved to $T_{\rm eot}$. The NTK is chosen at $T_{\rm ref}$ too../
  \ac{These plots will be modified (font size, etc...) to match the other figures.}}
  \label{fig:xT3_analytical}
\end{figure}
% ===================================

% ===================================
\begin{figure}[t!]
  \centering
  \includegraphics[width=0.65\textwidth]{plots/xT3_u_v_contribution_small_t.pdf} \\
  \includegraphics[width=0.65\textwidth]{plots/xT3_u_v_contribution_eot.pdf}
  \caption{Contribution of the $U$ and $V$ terms to the solution. The top panel shows this breakdown
  at early stages of the analytical training ($t=0.001$); the bottom panel shows the contributions at the end
  of training (eot) (same as Fig.~\ref{fig:xT3_analytical}).
  \ac{These plots will be modified (font size, etc...) to match the other figures.}}
\end{figure}
% ===================================
