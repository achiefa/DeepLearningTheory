The analytical solution in \eqref{eq:AnalyticSol} can
shed a new light onto the behaviour of the numerical neural network training.
In order to study the training process, the NNPDF collaboration has successfully developed so-called
{\em closure tests}. A closure test uses synthetic data, generated using a known
set of PDFs, to train the neural network. The PDFs used for genrating the data are called here {\em input}\
PDFs. The results of the training are then compared to the
known input PDFs; the performance of the training algorithm and the NN
architecture are assessed by quantifying the comparison between trained PDFs and input PDFs.
Following the original presentation in Ref.~\cite{NNPDF:2014otw}, we distinguish three
levels of closure tests, which are defined by the complexity of the data used to train the NNs.
We use the standard NNPDF nomenclature and refer to these three levels as level-0 (L0), level-1 (L1),
and level-2 (L2) closure tests and we denote the input PDFs used to generate the data as $\fin$.

Let us start by discussing the case of L0 tests. In this case, the data are given by
\begin{equation}
    \label{eq:DataL0}
    Y_I = T[\fin]_I
        = \sum_{i=1}^{\nflav} \sum_{\alpha=1}^{\ngrid} \FKtab_{Ii\alpha} \fin_{i\alpha}\, ,
\end{equation}
or equivalently, suppressing the indices,
\begin{equation}
    \label{eq:DataL0NoIndices}
    Y = \FKtab \fin\, .
\end{equation}
Using L0 data in the analytical expression for the trained network in Eq.~\eqref{eq:AnalyticSol} allows
a simplification of the second term,
\begin{align}
    \label{eq:L0ClosureTrained}
    V(t) Y = \sumprime_{i} Z^{(i)} \left(1 - e^{-h^{(i)}t}\right) 
        \sum_{k\in\perp} w^{(i)}_k \fin_k\, .
\end{align}
Interestingly, for $t\to\infty$, 
\begin{align}
    \label{eq:L0ClosureInfiniteTraining}
    \lim_{t\to\infty} V(t) Y = \finperp\, ,
\end{align}
and therefore the $V$ component of the trained solution reproduces exactly the 
component of the PDF that lies in  the subspace orthogonal to the kernel of $\Theta$.


Distance between the numerical minimization and the analytical formula as a function of
training time $t$ 

\ldd{Compute distances between numerical and analytical solutions using the standard NNPDF definition of the
distance. Same as the ones that are below, but using L0 data}


\newpage



\begin{figure}[t!]
  \label{fig:xT3_exp_val}
  \centering
  \includegraphics[width=0.65\textwidth]{plots/xT3_exp_val_early.pdf} \\
  \includegraphics[width=0.65\textwidth]{plots/xT3_exp_val_eot.pdf}
  \caption{Expectation value of the product (blue) and product of the
  expectation value for the $U$ contribution. Plots obtained using $t_{\rm ref}
  = 30000$, while $f_0$ is an ensemble of networks at initialisation
  (\textit{i.e.} $\mathbb{E}[f_0]=0$). \ac{These plots will be modified (font
  size, etc...) to match the other figures.}} \ldd{Error bars needed}
\end{figure}
% ===================================


\begin{figure}[h]
    \centering
    \includegraphics[width=0.48\textwidth]{plots/pdf_plot_init.pdf}
    \includegraphics[width=0.48\textwidth]{plots/distance_plot_init.pdf}
    \caption{PDF comparison (left) and PDF distance (right) of the trained
    solution at the end of training (eot) and analytical solution. The frozen NTK
    is chosen at $T_{\rm ref} = 20000$ and the initial function $f_0$ is a
    different ensemble of networks at initialisation. The analytical solution is
    evolved for the equivalent of $T_{\rm tot}$ epochs. \ac{Plots obtained with L2
    data}}
    \label{fig:xT3_analytical_init}
  \end{figure}
  % ===================================
  
  % ===================================
  \begin{figure}[h]
    \centering
    \includegraphics[width=0.48\textwidth]{plots/pdf_plot_ref_20000.pdf}
    \includegraphics[width=0.48\textwidth]{plots/distance_plot_ref_20000.pdf}
    \caption{PDF comparison (left) and PDF distance (right) of the trained
    solution at the end of training (eot) and analytical solution. The frozen NTK
    is chosen at $T_{\rm ref} = 20000$ and the initial function $f_0$ is the
    trained function at $T_{\rm ref}$. The analytical solution is evolved for the
    equivalent of $T_{\rm tot} - T_{\rm ref}$ epochs. \ac{Plots obtained with L2
    data}}
    \label{fig:xT3_analytical_ref}
  \end{figure}
  % ===================================
  
  % ===================================
  \begin{figure}[h]
    \centering
    \includegraphics[width=0.48\textwidth]{plots/pdf_plot_ref_0.pdf}
    \includegraphics[width=0.48\textwidth]{plots/distance_plot_ref_0.pdf}
    \caption{Same as Fig.~\ref{fig:xT3_analytical_ref}, but for $T_{\rm ref} = 0$.
    \ac{Plots obtained with L2 data}}
    \label{fig:xT3_analytical_ref_0}
  \end{figure}
  % ===================================

  % ===================================
\begin{figure}[t!]
    \centering
    \includegraphics[width=0.65\textwidth]{plots/xT3_u_v_contribution_small_t.pdf}
    \\
    \includegraphics[width=0.65\textwidth]{plots/xT3_u_v_contribution_eot.pdf}
    \caption{Contribution of the $U$ and $V$ terms to the solution. The top panel
    shows this breakdown at early stages of the analytical training ($t=0.001$);
    the bottom panel shows the contributions at the end of training (eot). 
    These plots have been obtained by taking the
    $t_{\rm ref} = 30000$ and $f_0 = f_{t_{\rm ref}}$ using L2 data. \ac{These
    plots will be modified (font size, etc...) to match the other figures.}}
  \end{figure}
  % ===================================


\FloatBarrier
