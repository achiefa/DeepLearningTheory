\section{Cut-off Tolerance of the NTK spectrum}
\label{app:cutoff}

\begin{figure}[h!]
  \centering
  \includegraphics[width=0.6\textwidth]{appendix_cutoff/cutoff.pdf}
  \caption{Visualisation of the cut-off tolerance for the NTK spectrum used in
    Section~\ref{sec:LazyTraining}. Eigenvalues (orange solid line) and singular values (blue dashed
    line) are shown, together with the cut-off tolerance (black) chosen as described
    in the text. We also show the corresponding numerical noise (grey shaded area)
    which occurs at small eigenvalues (or singular values).}
  \label{fig:cutoff}
\end{figure}

We determine the effective rank of the NTK by identifying eigenvalues that are
numerically significant. We classify eigenvalues $\lambda_i$ as numerically zero
if
\begin{equation}
  \lambda_i < \epsilon_{\rm tol} \cdot \lambda_{\rm max} \,,
\end{equation}
where $\epsilon_{\rm tol}$ is the relative tolerance and $\lambda_{\rm max}$ is
the largest eigenvalue of the NTK. Throughout this work, we use $\epsilon_{\rm
tol} = 10^{-7}$, which is close to machine precision for single-precision
floating point numbers. This choice is illustrated in Fig.~\ref{fig:cutoff},
where we show the NTK spectrum at initialization for the NNPDF-like architecture
discussed in the main text (orange solid line). Together with the eigenvalues,
we also show the singular values of the NTK (dashed blue line) to illustrate
that the two spectra are identical until numerical noise sets in. The cut-off
tolerance, indicated by the black horizontal line, is chosen to be slightly
above the numerical noise level (grey shaded area) to ensure that only
numerically significant eigenvalues are retained.
